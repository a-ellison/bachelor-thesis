\section{Tighter GSD security}

\subsection{Runtime in Theorem~\ref{theorem:sdgsd-security}} \label{sec:sdgsd-security-runtime}

For completeness, we provide a more precise expression of the runtime $\tilde{t}$ in Theorem~\ref{theorem:sdgsd-security}. For appropriately chosen constants we have
\[
	\tilde{t} = t \begin{aligned}[t]
		- & \mathcal{O}
		\begin{aligned}[t]
			\big(\rho \cdot t_{\mathrm{sample}} \cdot \ms & + (\gamma + \eta \cdot t_{\mathrm{sample}}) \cdot \mdh \\ + & N \cdot ((\rho + \eta) \cdot t_{\mathrm{sample}} + \mdh \cdot t_{\mathrm{op}} + t_{\dhies.\gen})  \\ + & N^2 \cdot t_{\dhies.\enc}\big), \\
		\end{aligned}
	\end{aligned}
\]
where the various variables denote the following
\begin{itemize}
	\item $t_{\mathrm{sample}}$: time to sample a uniform bit
	\item $t_{\dhies.\enc}$: time to encrypt $s \in \{0, 1\}^\rho$ with $\dhies$
	\item $t_{\dhies.\gen}$: runtime of $\dhies.\gen(1^\eta)$ (which is strictly greater than the runtime of $\dhies.\gen(1^\eta, r)$ for input randomness $r$)
	\item $t_{\mathrm{op}}$: time to perform the group operation in a group output by $\mathcal{G}(1^\eta)$
	\item $\gamma$: maximum length of any query to $\hdh$
\end{itemize}

\subsection{Proof of Lemma~\ref{lemma:mis-eav-from-eav}} \label{sec:mis-eav-from-eav-proof}

\begin{proof}[of Lemma~\ref{lemma:mis-eav-from-eav}] Note that since the message space is finite, the time to encrypt a message is bounded. As outlined before Lemma~\ref{lemma:mis-eav-from-eav}, the lemma follows from a simple hybrid argument. Let $q$ a function in $\kappa$, let $\kappa$ arbitrary and let $\adv$ an arbitrary MIS-EAV adversary running in time $\tilde{t}(\kappa)$ and making at most $q(\kappa)$ queries. Define the sequence of hybrid games $G_0, \ldots, G_q$ where in the game $G_i$ the first $i$ encryptions given to the adversary encrypt $m_1$ and all remaining encryptions encrypt $m_0$. We will write
	\[
		\pr{0 \from \adv \mid G_i}
	\]
	for the probability of $\adv$ outputting $0$ when playing the hybrid game $G_i$.

	Let $i \in [q]$. Construct an EAV adversary $\adv'$ that behaves as follows:
	\begin{enumerate}[1.]
		\item $\adv'$ runs $\adv$ and gets $q, m_0, m_1$.
		\item $\adv'$ outputs the messages $m_0, m_1$ and gets a ciphertext $c$ from the challenger.
		\item $\adv'$ gives the ciphertexts $c_1, \ldots, c_q$ to $\adv$ where
		      \[
			      c_j = \begin{cases}
				      \Pi.\enc_{k_j}(m_1) & i < j \\
				      c                   & i = j \\
				      \Pi.\enc_{k_j}(m_0) & i > j
			      \end{cases}
		      \]
		      and $k_j \from \Pi.\gen(1^\kappa) \; \forall j$.
		\item $\adv'$ outputs whatever bit $\adv$ outputs.
	\end{enumerate}
	Now consider the value of the bit $b$ sampled in $\game{\Pi}{\kappa}{EAV}(\adv')$. If $b = 0$, then the first $i - 1$ ciphertexts that $\adv$ received were encryptions of $m_1$ and the remaining ciphertexts were encryptions of $m_0$, where all encryptions were under keys sampled independently with $\Pi.\gen$. Thus, from the view of $\adv$ everything followed the same distribution as in the game $G_{i - 1}$ and
	\[
		\pr{0 \from \adv' \mid b = 0} = \pr{0 \from \adv \mid G_{i - 1}}.
	\]
	Analogously, in the case $b = 1$ the first $i$ ciphertexts received by $\adv$ were encryptions of $m_1$ and the rest encryptions of $m_0$, so
	\[
		\pr{0 \from \adv' \mid b = 1} = \pr{0 \from \adv \mid G_{i}}.
	\]
	Then
	\begin{align} \label{eq:eav-to-mis-eav-hybrid-distinguisher}
		\begin{split}
			\begin{split}
				\pr{0 \from \adv \mid G_{i - 1}} - & \pr{0 \from \adv \mid G_{i}} \\
				& = \pr{0 \from \adv' \mid b = 0} - \pr{0 \from \adv' \mid b = 1}
			\end{split} \\
			& \stackrel{\mathclap{\eqref{eq:advantage-equality}}}{=} \advantage{\Pi}{\kappa}{EAV}(\adv')                                         \\
			& \le \epsilon
		\end{split}
	\end{align}
	by $(t, \epsilon)$-EAV security of $\Pi$ since $\adv'$ runs in time $\tilde{t} + \mathcal{O}(q \cdot (t_{\gen} + t_{\enc})) = t$.

	Now let $b$ be the bit sampled in the MIS-EAV game. Notice that
	\[
		\pr{0 \from \adv \mid b = 0} = \pr{0 \from \adv \mid G_0}
	\]
	and
	\[
		\pr{0 \from \adv \mid b = 1} = \pr{0 \from \adv \mid G_q}.
	\]
	Then
	\begin{align*}
		\advantage{\Pi}{\kappa}{MIS\text{-}EAV}(\adv) & \stackrel{\mathclap{\eqref{eq:advantage-equality}}}{=} \; \pr{0 \from \adv \mid b = 0} - \pr{0 \from \adv \mid b = 1} \\
		                                              & = \pr{0 \from \adv \mid G_0} - \pr{0 \from \adv \mid G_q}                                                             \\
		                                              & = \sum_{i = 1}^{q} \pr{0 \from \adv \mid G_{i - 1}} - \pr{0 \from \adv \mid G_i}                                      \\
		                                              & \stackrel{\mathclap{\eqref{eq:eav-to-mis-eav-hybrid-distinguisher}}}{\le} q \cdot \epsilon.
	\end{align*}
\end{proof}

\subsection{Proof of Lemma~\ref{lemma:eav-reduction}} \label{sec:eav-reduction-proof}

\begin{proof}[of Lemma~\ref{lemma:eav-reduction}]
	As already motivated after Lemma~\ref{lemma:eav-reduction}, we construct an MIS-EAV adversary $\adv'$ to derive the bound. $\adv'$ behaves as follows:
	\begin{enumerate}[1.]
		\item $\adv'$ runs $\adv$ to get $n$ and $D$ and initializes the GSD graph, seeds and the set of edges and corrupted nodes as in step \ref{def:sd-gsd-game-step-init} of the SD-GSD game.
		\item \label{eav-reduction-mis-eav-adversary-step-init} $\adv'$ samples $w \from [n], s \from \{0, 1\}^\rho$ and gives $\delta$ and the messages $s_w, s$ to the challenger. Let $c_1, \ldots, c_\delta$ the encryptions it gets back.
		\item $\adv'$ faithfully simulates the SD-GSD game to $\adv$ with the following exception: Whenever $\adv$ makes a query of the form $\operatorname{encrypt}(u, w)$ for any $u$, $\adv'$ replies with $\langle g^x, c_i \rangle$ where $x \from [q]$ and $i$ is the index of the next ciphertext (from step \ref{eav-reduction-mis-eav-adversary-step-init} of the MIS-EAV game) not yet used.

		      All random oracle queries are simulated by sampling the output of the oracle u.a.r. for new queries and using the value first sampled for repeated queries.

		      During the simulation $\adv'$ also pays attention to the following:
		      \begin{itemize}
			      \item If any of the following events occur, $\adv'$ aborts the simulation and outputs 1:
			            \begin{itemize}
				            \item $\adv$ queries $\hdh$ for a hidden Diffie-Hellman key
				            \item $\adv$ queries $\hgen$ or $\hdep$ for a hidden seed that is not $s_w$
				            \item $\adv$ queries $\operatorname{corrupt}(u)$ for some node $u$ such that $w$ is no longer safe
			            \end{itemize}
			      \item If $\adv$ queries $\hgen(s_w)$ or $\hdep(s_w)$, $\adv'$ aborts the simulation and outputs 0. This is the only point at which $\adv'$ outputs 0.
		      \end{itemize}

		      If the simulation arrives at the point where $\adv$ outputs its guess (step \ref{def:sd-gsd-game-step-output} of the SD-GSD game), then $\adv'$ outputs 1.
	\end{enumerate}

	The advantage of $\adv'$ is given by
	\begin{equation} \label{eq:lemma-eav-reduction-advantage}
		\advantage{\Pi_s}{\eta}{MIS\text{-}EAV}(\adv') \stackrel{\eqref{eq:advantage-equality}}{=}  \pr{0 \from \adv' \mid b = 0} - \pr{0 \from \adv' \mid b = 1},
	\end{equation}
	where $b$ is the bit sampled by the MIS-EAV challenger.


	First, we will show that
	\begin{equation} \label{eq:lemma-eav-reduction-b=0}
		\pr{0 \from \adv' \mid b = 0} \ge \frac{\pr{\qs \land \overline{\fdh}\,}}{N}.
	\end{equation}
	Let $E = \qs \land \overline{\fdh}$. In the following, while showing \eqref{eq:lemma-eav-reduction-b=0} we will implicitly assume that $b = 0$ when referring to an execution of $\game{\Pi_s}{\eta}{MIS\text{-}EAV}(\adv')$.
	On a high level \eqref{eq:lemma-eav-reduction-b=0} holds because as long as the game has not been aborted the encryptions $\adv$ receives from $\adv'$ are indistinguishable from what it would get in the real SD-GSD game and we get a factor $\frac{1}{N}$ from guessing the node that triggered $E$. However, showing this requires a few steps.

	Consider a modification of the SD-GSD game $G_1$ where the game is aborted whenever one of the following events occurs, where for all these events $\adv'$ would also abort the simulation:
	\begin{itemize}
		\item $\adv$ queries $\hdh$ for a hidden Diffie-Hellman key
		\item $\adv$ queries $\hgen$ or $\hdep$ for a hidden seed
	\end{itemize}
	(Since we are not interested in the output of the game we can define \emph{aborting the game} as the game ending with output 0.) The game $G_1$ is something between the real SD-GSD game and what $\adv'$ simulates to $\adv$. The only difference in when $G_1$ aborts compared to the game simulated by $\adv'$ is that we are not paying attention to some specific node $w$ remaining safe. Aborting the game in this way does not alter the probability of $\adv$ triggering the event $E$ in $G_1$, since in either case when the game is aborted either $E$ or $\overline{E}$ is already known to hold:
	\begin{itemize}
		\item If $\adv$ queries $\hdh$ for a hidden Diffie-Hellman key, then it triggers $\qdh$ and $\qs$ has not been triggered before since this would have caused the game to be aborted. Thus, $\adv$ triggered $\fdh$ and $\qs \land \overline{\fdh}$ cannot hold in this execution of the game.
		\item If $\adv$ queries $\hgen$ or $\hdep$ for a hidden seed, then this triggers $\qs$. Moreover, $\overline{\fdh}$ also holds at this moment since the game would have aborted earlier if $\qdh$ had already been triggered. Thus, $\qs \land \overline{\fdh}$ holds.
	\end{itemize}
	Let $E_1$ the same event as $E$ in the game $G_1$. As argued above we have
	\begin{equation} \label{eq:lemma-eav-reduction-E=E_1}
		\pr{E_1} = \pr{E}.
	\end{equation}


	Now consider a game $G_2$ which is a modification of the game $G_1$ where at the beginning of the game $w_2 \from [n]$ is sampled and the game also aborts if $\adv$ queries $\operatorname{corrupt}(u)$ for some node $u$ such that $w_2$ is no longer safe, just as in the game simulated by $\adv'$. The game $G_2$ is again something between the game $G_1$ and what $\adv'$ simulates to $\adv$. We also modify $G_1$ such that it also samples $w_1 \from [n]$ at the beginning of the game. This does not change the fact that \eqref{eq:lemma-eav-reduction-E=E_1} holds as the sampling of $w_1$ has no effect on the execution of the game.

	Let $E_2$ and $E'$ the events corresponding to $E$ in the game $G_2$ and the game simulated by $\adv'$, respectively. We further introduce a new random variable $W$ to analyze each game where
	\[
		W = \begin{cases}
			0 & \overline{E}                       \\
			x & E \text{ was triggered at node } x
		\end{cases}
	\]
	(if $x$ is not unique we choose the node with the lowest identifier).
	Let $W_1$, $W_2$ and $W'$ be the corresponding random variables in game $G_1$, game $G_2$ and the game simulated by $\adv'$. Consider the probability $\pr{W_1 = w_1 \mid E_1}$. The node $w_1$ is sampled independently and does not affect the execution of the game. Therefore, in an execution where $E_1$ occurs and the GSD graph has size $n$ (so $W_1 \in [n]$), we correctly guess $W_1 = w_1$ with probability exactly $\frac{1}{n} \ge \frac{1}{N}$. Thus
	\[
		\pr{W_1 = w_1 \mid E_1} \ge \frac{1}{N}
	\]
	and combining this with \eqref{eq:lemma-eav-reduction-E=E_1} we get
	\begin{align} \label{eq:lemma-eav-reduction-detect-E1-G1}
		\begin{split}
			\pr{W_1 = w_1} & = \pr{W_1 = w_1 \land E_1} \\
			& = \pr{W_1 = w_1 \mid E_1} \cdot \pr{E_1} \\
			& \ge \frac{1}{N} \cdot \pr{E}.
		\end{split}
	\end{align}

	Analogously to the argument used to justify \eqref{eq:lemma-eav-reduction-E=E_1}, we can argue that
	\begin{equation} \label{eq:lemma-eav-reduction-G1-vs-G2}
		\pr{W_1 = w_1} = \pr{W_2 = w_2}.
	\end{equation}
	The only difference from $G_1$ to $G_2$ is that $G_2$ aborts when $w_2$ is no longer safe. But if $w_2$ is no longer safe then we know that $W_2 \neq w_2$ (if $W_2 = w_2$ the game would have already aborted when $w_2$'s seed was queried while it was safe). Thus, \eqref{eq:lemma-eav-reduction-G1-vs-G2} indeed holds.

	We now show an analogous result comparing the game $G_2$ to the game simulated by $\adv'$:
	\begin{equation} \label{eq:lemma-eav-reduction-G2-vs-simulation}
		\pr{W_2 = w_2} = \pr{W' = w}.
	\end{equation}
	Consider how $G_2$ differs from the game simulated by $\adv'$. Both games abort at exactly the same events (this is easy to see). They only differ in how $\adv'$ answers queries $\operatorname{encrypt}(u, w)$ for any $u$. In $G_2$ such a query is answered with a ciphertext $\langle g^x, c \rangle$ where $x \from [q], c \from \Pi_s.\enc_k(s_w)$ and $k = \hdh(pk_u^x)$. $\adv'$ answers such a query with $\langle g^{x'}, c' \rangle$ where $x' \from [q], c' \from \Pi_s.\enc_{k'}(s_w)$ and $k' \from \{0, 1\}^\eta$. Now notice that as long as the game $G_2$ is ongoing, $pk_u^{x}$ is a hidden Diffie-Hellman key and $\adv$ has not queried $pk_u^{x}$ to $\hdh$. If it had, then the game would have already aborted. Therefore, from $\adv$'s view $k$ follows the same distribution as $k'$. Thus, overall the game $G_2$ and the game simulated by $\adv'$ are indistinguishable to $\adv$ and \eqref{eq:lemma-eav-reduction-G2-vs-simulation} holds.

	Finally, notice that if the event $W' = w$ occurred, then $\adv'$ outputs $0$. Then we have
	\begin{align*}
		\pr{0 \from \adv' \mid b = 0} & \ge \pr{W' = w}                                                              \\
		                              & \stackrel{\eqref{eq:lemma-eav-reduction-G2-vs-simulation}}{=} \pr{W_2 = w_2} \\
		                              & \stackrel{\eqref{eq:lemma-eav-reduction-G1-vs-G2}}{=}  \pr{W_1 = w_1}        \\
		                              & \stackrel{\eqref{eq:lemma-eav-reduction-detect-E1-G1}}{\ge} \frac{\pr{E}}{N} \\
		                              & = \frac{\pr{\qs \land \overline{\fdh}\,}}{N},
	\end{align*}
	as promised.

	Second, returning to \eqref{eq:lemma-eav-reduction-advantage}, we can more easily show that $\pr{0 \from \adv' \mid b = 1}$ is negligible. In the SD-GSD game simulated to $\adv$ during an execution of $\game{\Pi_s}{\eta}{MIS\text{-}EAV}(\adv')$ with $b = 1$, the seed $s_w$ is a random variable independent of any information given to $\adv$:
	\begin{itemize}
		\item the game aborts when $w$ becomes unsafe, so $s_w$ cannot be learned by querying $\operatorname{corrupt}(w)$ or by querying $\hdep(s_p)$ for an unsafe node $p$ where $(p, w)$ is a seed dependency
		\item querying $\hdep(s_p)$ for a safe node $p$ where $(p, w)$ is a seed dependency results in the game being aborted and by virtue of $\hdep$ being a random oracle, from $\adv$'s perspective $s_w$ follows the same distribution regardless of whether there is a seed dependency $(p, w)$ or not
		\item with $b = 1$ queries $\operatorname{encrypt}(u, w)$ yield encryptions of $s$ instead of $s_w$
	\end{itemize}
	Therefore, for any seed $s'$ that $\adv$ queries to $\hgen$ or $\hdep$ we have
	\[
		\pr{s_w = s'} = \frac{1}{2^\rho}.
	\]
	Thus, by a union bound we have
	\begin{equation} \label{eq:lemma-eav-reduction-b=1}
		\pr{0 \from \adv' \mid b = 1} \le \frac{\ms}{2^\rho}.
	\end{equation}

	Combining \eqref{eq:lemma-eav-reduction-advantage}, \eqref{eq:lemma-eav-reduction-b=0} and \eqref{eq:lemma-eav-reduction-b=1} we get
	\begin{align} \label{eq:lemma-eav-reduction-advantage-lower-bound}
		\begin{split}
			\advantage{\Pi_s}{\eta}{MIS\text{-}EAV}(\adv') & = \pr{0 \from \adv' \mid b = 0} - \pr{0 \from \adv' \mid b = 1}           \\
			& \ge \frac{\pr{\qs \land \overline{\fdh}\,}}{N} - \frac{\ms}{2^\rho}.
		\end{split}
	\end{align}
	Furthermore, going through the details yields that $\adv'$ runs in time
	\begin{align*}
		\begin{split}
			t_{\adv'} \coloneqq \tilde{t} + \mathcal{O}\big(\rho \cdot t_{\mathrm{sample}} \cdot \ms & + (\gamma + \eta \cdot t_{\mathrm{sample}}) \cdot \mdh  \\
			& + N \cdot (\rho \cdot t_{\mathrm{sample}} + t_{\dhies.\gen})  \\
			& +  N^2 \cdot t_{\dhies.\enc}\big)
		\end{split}
	\end{align*}
	(note that $t_{\adv'}$ is a constant).
	Using that $\delta \le N, t_{\Pi_s.\gen} \le \mathcal{O}(\eta \cdot t_{\mathrm{sample}})$, $t_{\Pi_s.\enc} \le t_{\dhies.\enc}$ (as encrypting with $\dhies$ involves an encryption with $\Pi_s$), the definition of $\tilde{t}$, with appropriately chosen constants we have
	\[
		t_{\adv'} \le t - \mathcal{O}(\delta \cdot (t_{\Pi_s.\gen} + t_{\Pi_s.\enc})).
	\]
	By Lemma~\ref{lemma:mis-eav-from-eav} $\Pi_s$ is $(t - \mathcal{O}(\delta \cdot (t_{\Pi_s.\gen} + t_{\Pi_s.\enc})), \delta \cdot \eeav, \delta)$-MIS-EAV-secure, so
	\begin{equation} \label{eq:lemma-eav-reduction-advantage-upper-bound}
		\advantage{\Pi_s}{\eta}{MIS\text{-}EAV}(\adv') = \delta \cdot \eeav.
	\end{equation}

	Finally, if we now combine \eqref{eq:lemma-eav-reduction-advantage-lower-bound} and \eqref{eq:lemma-eav-reduction-advantage-upper-bound} we get
	\begin{align*}
		\frac{\pr{\qs \land \overline{\fdh}\,}}{N} - \frac{\ms}{2^\rho} & \le \delta \cdot \eeav                                       \\
		                                                                & \iff                                                         \\
		\pr{\qs \land \overline{\fdh}\,}                                & \le \delta \cdot N \cdot \eeav + \frac{\ms \cdot N}{2^\rho},
	\end{align*}
	as was to prove.
\end{proof}


\subsection{Tighter MIS-EAV security for certain schemes} \label{sec:tighter-mis-eav-security}

In our reduction from MIS-EAV security to EAV security (Lemma~\ref{lemma:mis-eav-from-eav}) we applied a general hybrid argument. It is also tempting to try a more direct approach. The EAV and MIS-EAV games seem less far apart than IND-CPA for single and multiple encryptions: All additional encryptions in the MIS-EAV game encrypt the same message, with the only difference being that each encryption is performed using a fresh key. If only we could take a single encryption $c \from \enc_k(m)$ and from it produce several encryptions $c_i \from \enc_{k_i}(m)$ for $k_i \from \gen(1^\kappa)$ (without knowing $k$ or $m$), then the additional encryptions would leak no new information to the adversary, and we would have a tight bound on MIS-EAV security from EAV security. There is a simple EAV secure scheme that achieves the above property: the one-time pad. Given an encryption $c = k \oplus m$, we can just sample $k' \from \{0, 1\}^\kappa$ and compute the ciphertext $c' = c \oplus k' = (k \oplus k') \oplus m$, an encryption of $m$ under the uniformly random key $k \oplus k'$. In the following, we formalize this property of a private-key encryption scheme and use it to prove the desired bound on MIS-EAV security.

\begin{definition}[Key-rerandomizability] \label{def:key-rerandomizability}
	Let $\kappa$ denote the security parameter and let $\Pi = (\gen, \enc, \dec)$ a private-key encryption scheme. $\Pi$ is \emph{key-rerandomizable} if there exists a probabilistic polynomial-time algorithm $\operatorname{ReRan}$ achieving the following: Let $\kappa, k \from \gen(1^\kappa)$, $m$ in the message space and $c \from \enc_k(m)$ be arbitrary but fixed.\footnote{Here we are quantifying over all possible keys $k$ and ciphertexts $c$ that can be output by $\gen(1^\kappa)$ and $\enc_k(m)$.} Then the distribution over ciphertexts as defined by computing $c' \from \operatorname{ReRan}(1^\kappa, c)$ is identical to the distribution over ciphertexts resulting from the process of first sampling $k' \from \gen(1^\kappa)$ and then computing a ciphertext $c' \from \enc_{k'}(m)$.
\end{definition}

\paragraph{Example} As outlined above, the one-time pad is an example of a key-rerandomizable encryption scheme.

\question{Is there a key-rerandomizable IND-CPA secure scheme? If yes, this would imply a key-rerandomizable AE scheme using the encrypt-then-authenticate paradigm, since a rerandomized tag can easily produced for the ciphertext by sampling a fresh MAC key.}

The key idea underlying the proof of the following Lemma was already provided at the beginning of this section.

\begin{lemma} \label{lemma:mis-eav-from-eav-with-key-rerandomizability}
	Let $\Pi$ a key-rerandomizable private-key encryption scheme with finite message space. Let $\operatorname{ReRan}$ the corresponding algorithm to rerandomize ciphertexts and $t_{\operatorname{ReRan}}$ an upper bound for the runtime of $\operatorname{ReRan}$. If $\Pi$ is $(t, \epsilon)$-EAV-secure, then for all $q \in \N$, $\Pi$ is $(\tilde{t}, \epsilon, q)$-MIS-EAV-secure with $\tilde{t} = t - \mathcal{O}(q \cdot t_{\operatorname{ReRan}})$.
\end{lemma}

\begin{proof}
	Note that since the message space and thus the ciphertext space is finite, the runtime of $\operatorname{ReRan}$ is indeed bounded. Let $\kappa$ arbitrary. Let $\adv$ an MIS-EAV adversary running in time $\tilde{t}(\kappa)$ and making at most $q(\kappa)$ queries. We construct an EAV adversary $\adv'$ that behaves as follows:
	\begin{enumerate}[1.]
		\item $\adv'$ runs $\adv$ to get the number of queries $q$ and messages $m_0, m_1$.
		\item $\adv'$ gives $m_0, m_1$ to the challenger and receives the ciphertext $c_1$.
		\item $\adv'$ computes ciphertexts $c_2 \from \operatorname{ReRan}(1^\kappa, c_1), \ldots, c_q \from \operatorname{ReRan}(1^\kappa, c_1)$ (with independent runs of $\operatorname{ReRan}$).
		\item $\adv'$ gives the ciphertexts $c_1, \ldots, c_q$ to $\adv$.
		\item $\adv'$ outputs whatever bit $\adv$ outputs.
	\end{enumerate}
	We apply the properties of $\operatorname{ReRan}$ given in Definition~\ref{def:key-rerandomizability} to show that the game simulated to $\adv$ is distributed identically to the MIS-EAV game. For this we need only show that the ciphertexts $c_1, \ldots, c_q$ given to $\adv$ in the simulation are distributed identically to the ciphertexts $c_1', \ldots, c_q'$ that $\adv$ would get in the real MIS-EAV game. It is immediate that $c_1$ is distributed identically to $c_1'$. Now let $i \in \{2, \ldots, q\}$. By Definition~\ref{def:key-rerandomizability} $\operatorname{ReRan}(c)$ outputs a ciphertext encrypting $m_b$ (where $b$ is the bit chosen by the EAV challenger) distributed identically to a ciphertext encrypting $m_b$ output by the MIS-EAV challenger. Thus, indeed for any $i$, $c_i$ is distributed identically to $c_i'$ and the claim holds. Therefore
	\begin{equation} \label{eq:lemma-key-rerandomizability-advantage-same}
		\advantage{\Pi}{\kappa}{MIS\text{-}EAV}(\adv) = \advantage{\Pi}{\kappa}{EAV}(\adv').
	\end{equation}

	Because $\adv'$ is an EAV adversary running in time $\tilde{t} + \mathcal{O}(q \cdot t_{\operatorname{ReRan}}) = t$ we know that
	\[
		\advantage{\Pi}{\kappa}{EAV}(\adv') \le \epsilon(\kappa),
	\]
	which together with \eqref{eq:lemma-key-rerandomizability-advantage-same} concludes the proof.
\end{proof}

By assuming a key-rerandomizable encryption scheme and applying Lemma~\ref{lemma:mis-eav-from-eav-with-key-rerandomizability} instead of the hybrid argument (Lemma~\ref{lemma:mis-eav-from-eav}) in the proof of Lemma~\ref{lemma:eav-reduction}, we can drop the $\delta$ factor in the bound. This also allows us to drop the $\delta$ factor in Theorem~\ref{theorem:sdgsd-security}.

\begin{corollary}
	Recall the setting of Theorem~\ref{theorem:sdgsd-security}. If the private-key encryption scheme $\Pi_s$ is additionally key-rerandomizable, then the bound in Lemma~\ref{lemma:eav-reduction} can be improved to
	\[
		\pr{\qs \land \overline{\fdh}\,} \le N \cdot \eeav + \frac{\ms \cdot N}{2^\rho}
	\]
	and the bound $\tilde{\epsilon}$ on the success probability of an SD-GSD adversary thus improved to
	\[
		\tilde{\epsilon} = 2 \cdot N \cdot (\eeav + \eddh) + \frac{2 \cdot \mdh \cdot N^2}{q} + \frac{\ms \cdot N}{2^{\rho - 1}} + \frac{N^2}{2^{\rho - 3}}
	\]
	(with appropriate changes to the runtime $\tilde{t}$).
\end{corollary}

\subsection{Proof of Lemma~\ref{lemma:dh-reduction}} \label{sec:dh-reduction-proof}

\begin{proof}[of Lemma~\ref{lemma:dh-reduction}]
	As outlined after Lemma~\ref{lemma:dh-reduction} we use $\adv$ to construct a DDH adversary $\adv'$:
	\begin{enumerate}[1.]
		\item $\adv'$ gets $h_1, h_2$ and $k$ from the DDH challenger.
		\item $\adv'$ runs $\adv$ to get $n$ and $D$, samples $a \from [n]$ and initializes the GSD graph, seeds and the set of edges and corrupted nodes as in step \ref{def:sd-gsd-game-step-init} of the SD-GSD game, with the sole exception that $pk_a = h_1$ (as opposed to setting it to the public key output by $\dhies.\gen(1^\eta, \hgen(s_a))$).
		\item $\adv'$ faithfully simulates the SD-GSD game to $\adv$ with the following exception: For the $j$-th query $\operatorname{encrypt}(a, u_j)$ made by $\adv$, $\adv'$ replies with $\langle h_2 \cdot g^{r_j}, \Pi_s.\enc_{k_j}(s_{u_j}) \rangle$ where $r_j \from [q]$, $k_j \from \{0, 1\}^\eta$. $\adv'$ also computes and stores $R_j = \left(pk_a^{r_j}\right)^{-1}$.

		      All random oracle queries are simulated by sampling the output of the oracle u.a.r. for new queries and using the value first sampled for repeated queries.

		      During the simulation $\adv'$ also pays attention to the following:
		      \begin{itemize}
			      \item If any of the following events occur, $\adv'$ aborts the simulation and outputs 1:
			            \begin{itemize}
				            \item $\adv$ queries $\hdh$ for a hidden Diffie-Hellman key on an encryption edge $(u, v) \in E$ with $u \neq a$
				            \item $\adv$ queries $\hgen$ or $\hdep$ for a hidden seed
				            \item $\adv$ queries $\operatorname{corrupt}(u)$ for some node $u$ such that $a$ is no longer safe
			            \end{itemize}
			      \item If $\adv$ queries $q_i$ to $\hdh$ such that $q_i \cdot R_j = k$ for some $j$, $\adv'$ aborts the simulation and outputs 0. This is the only point at which $\adv'$ outputs 0.
		      \end{itemize}

		      If the simulation arrives to the point where $\adv$ outputs its guess (step \ref{def:sd-gsd-game-step-output} of the SD-GSD game), then $\adv'$ outputs 1.
	\end{enumerate}

	The advantage of $\adv'$ is given by
	\begin{equation} \label{eq:lemma-dh-reduction-advantage}
		\advantage{\mathcal{G}}{\eta}{DDH}(\adv') \stackrel{\eqref{eq:advantage-equality}}{=} \pr{0 \from \adv' \mid b = 0} - \pr{0 \from \adv' \mid b = 1},
	\end{equation}
	where $b$ is the bit sampled by the DDH challenger.

	First, we will show that
	\begin{equation} \label{eq:lemma-dh-reduction-b=0}
		\pr{0 \from \adv' \mid b = 0} \ge \frac{\pr{\fdh}}{N}.
	\end{equation}
	This part of the proof proceeds very similarly to the proof of Lemma~\ref{lemma:eav-reduction} and we will be a bit more concise. We focus on executions of $\game{\mathcal{G}}{\eta}{DDH}(\adv')$ with $b = 0$. Let the games $G_1, G_2$ be defined as in Lemma~\ref{lemma:eav-reduction}, where we denote the node sampled at the beginning of each game by $a_1, a_2$, respectively (as opposed to $w_1, w_2$). Let $E = \fdh$ and let $E_1, E_2$ and $E'$ be the analogous events in $G_1, G_2$ and the game simulated by $\adv'$ (note that in this latter game, the group elements $pk_a^{\log_g(h_2) + r_j}$ are also hidden Diffie-Hellman keys). Finally, we introduce the random variable
	\[
		A = \begin{cases}
			0 & \overline{\fdh}                                                                    \\
			x & \text{$\fdh$ holds and $\qdh$ was triggered on an encryption edge with source $x$}
		\end{cases}
	\]
	(if $x$ is not unique we choose the node with the lowest identifier) and let $A_1, A_2$ and $A'$ denote the corresponding random variables in game $G_1$, game $G_2$ and the game simulated by $\adv'$.

	Just as argued in Lemma~\ref{lemma:eav-reduction},
	\begin{equation} \label{eq:lemma-dh-reduction-E=E_1}
		\pr{E_1} = \pr{E}
	\end{equation}
	holds, since whenever $G_1$ aborts, it is already decided whether $\fdh$ holds:
	\begin{itemize}
		\item If the game was aborted when $\adv$ queried a hidden Diffie-Hellman key, then $\fdh$ holds.
		\item If the game was aborted when $\adv$ queried $\hgen$ or $\hdep$ for a hidden seed, $\fdh$ does not hold.
	\end{itemize}

	Next, the inequality
	\[
		\pr{A_1 = a_1 \mid E_1} \ge \frac{1}{N}
	\]
	and therefore also
	\begin{equation} \label{eq:lemma-dh-reduction-detect-E1-G1}
		\pr{A_1 = a_1} \ge \frac{1}{N} \cdot \pr{E}
	\end{equation}
	hold for the same reason that
	\[
		\pr{W_1 = w_1 \mid E_1} \ge \frac{1}{N}
	\]
	and \eqref{eq:lemma-eav-reduction-detect-E1-G1} held in Lemma~\ref{lemma:eav-reduction}.

	Then, the equality
	\begin{equation} \label{eq:lemma-dh-reduction-G1-vs-G2}
		\pr{A_1 = a_1} = \pr{A_2 = a_2}
	\end{equation}
	holds again due to the fact that when $G_2$ aborts because $a_2$ is no longer safe, we know that $A_2 \neq a_2$.

	Finally, we need to argue that
	\begin{equation} \label{eq:lemma-dh-reduction-G2-vs-simulation}
		\pr{A_2 = a_2} = \pr{A' = a}.
	\end{equation}
	Consider how $G_2$ differs from the game simulated by $\adv'$. As in Lemma~\ref{lemma:eav-reduction}, both games abort at exactly the same events (note that if $q_i \cdot R_j = k$ holds and $\adv'$ outputs $0$, then $q_i = k \cdot R_j^{-1} = k \cdot pk_a^{r_j} = h_1^{\log_g(h_2)} \cdot pk_a^{r_j} = pk_a^{\log_g(h_2) + r_j}$, a hidden Diffie-Hellman key). The game simulated by $\adv'$ differs in two aspects:
	\begin{enumerate}[(i)]
		\item $\adv'$ sets $pk_a$ to $h_1$ and not to the public key output by $\dhies.\gen(1^\eta, \hgen(s_a))$
		\item $\adv'$ answers queries $\operatorname{encrypt}(a, u)$ differently
	\end{enumerate}
	Note that as long as the game $G_2$ is ongoing, $\adv$ has not queried $\hgen$ for $s_a$ or $\hdh$ for a hidden Diffie-Hellman key. Both differences are therefore indistinguishable:
	\begin{enumerate}[(i)]
		\item By Definition~\ref{def:public-key-encryption}, the output of $\dhies.\gen(1^\eta, r)$ with $r \from \{0, 1\}^\rho$ follows the same distribution as the output of $\dhies.\gen(1^\eta)$. The former process is behind the distribution of $pk_a$ as viewed from $\adv$ in $G_2$, as $\adv$ has not queried $\hgen(s_a)$, and the latter process is behind the distribution of $pk_a$ in the game simulated by $\adv'$, as the DDH challenger generates a public key with the same distribution as $\dhies.\gen(1^\eta)$. Since both processes follow the same distribution, $pk_a$ follows the same in $G_2$ and the game simulated by $\adv'$ from $\adv$'s perspecive.
		\item In $G_2$ a query $\operatorname{encrypt}(a, u)$ is answered with $\langle g^z, c \rangle$ where $z \from [q], c \from \Pi_s.\enc_k(s_u)$ and $k = \hdh(pk_a^z)$. $\adv'$ answers such a query with $\langle g^{\log_g(h_1) + r}, c' \rangle$ where $r \from [q], c' \from \Pi_s.\enc_{k'}(s_u)$ and $k' \from \{0, 1\}^\eta$. First, $\log_g(h_1) + r$ follows the same distribution as $z$. Second, $pk_a^z$ is a hidden Diffie-Hellman key and from $\adv$'s view $k$ follows the same distribution as $k'$.
	\end{enumerate}
	Thus \eqref{eq:lemma-dh-reduction-G2-vs-simulation} indeed holds.

	Now, again analogous to Lemma~\ref{lemma:eav-reduction} if the event $A' = a$ occurred, then $\adv'$ outputs $0$ and
	\begin{align*}
		\pr{0 \from \adv' \mid b = 0} & \ge \pr{A' = a}                                                                             \\
		                              & \stackrel{\mathclap{\eqref{eq:lemma-dh-reduction-G2-vs-simulation}}}{=} \;\; \pr{A_2 = a_2} \\
		                              & \stackrel{\mathclap{\eqref{eq:lemma-dh-reduction-G1-vs-G2}}}{=} \;\; \pr{A_1 = a_1}         \\
		                              & \stackrel{\mathclap{\eqref{eq:lemma-dh-reduction-detect-E1-G1}}}{\ge} \;\; \frac{\pr{E}}{N} \\
		                              & = \frac{\pr{\fdh}}{N}.
	\end{align*}

	Second, we will show that $\pr{0 \from \adv' \mid b = 1}$ is negligible. When $b = 1$ in $\game{\mathcal{G}}{\eta}{DDH}(\adv')$, $k$ is a uniformly random group element independent of any information given to $\adv$, in particular of $q_i \cdot R_j$ for any $i, j$. Thus for any $i, j$,
	\[
		\pr{q_i \cdot R_j = k} = \frac{1}{q}.
	\]
	Thus, by a union bound and using that $i \in [\mdh], 1 \le j \le N - 1 \le N$ ($j$ is bounded by the maximum outdegree) we have
	\begin{equation} \label{eq:lemma-dh-reduction-b=1}
		\pr{0 \from \adv' \mid b = 1} \le \frac{\mdh \cdot N}{q}.
	\end{equation}

	Combining \eqref{eq:lemma-dh-reduction-advantage}, \eqref{eq:lemma-dh-reduction-b=0} and \eqref{eq:lemma-dh-reduction-b=1} we get
	\begin{align} \label{eq:lemma-dh-reduction-advantage-lower-bound}
		\begin{split}
			\advantage{\mathcal{G}}{\eta}{DDH}(\adv') \ge \frac{\pr{\fdh}}{N} - \frac{\mdh \cdot N}{q}.
		\end{split}
	\end{align}

	Furthermore, going through the details yields that $\adv'$ runs in time
	\begin{align*}
		\begin{split}
			t_{\adv'} \coloneqq \tilde{t} + \mathcal{O}\big(\rho \cdot t_{\mathrm{sample}} \cdot \ms & + (\gamma + \eta \cdot t_{\mathrm{sample}}) \cdot \mdh \\
			& + N \cdot ((\rho + \eta) \cdot t_{\mathrm{sample}} + \mdh \cdot t_{\mathrm{op}} + t_{\dhies.\gen})  \\
			& +  N^2 \cdot t_{\dhies.\enc}\big).
		\end{split}
	\end{align*}
	Then using the definition of $\tilde{t}$, with appropriately chosen constants we have $t_{\adv'} \le t$. So by virtue of the DDH problem being $(t, \eddh)$-hard relative to $\mathcal{G}$
	\[
		\advantage{\mathcal{G}}{\eta}{DDH}(\adv') \le \eddh
	\]
	and if we combine this with \eqref{eq:lemma-dh-reduction-advantage-lower-bound} we get
	\begin{align*}
		\frac{\pr{\fdh}}{N} - \frac{\mdh \cdot N}{q} & \le \eddh                                     \\
		                                             & \iff                                          \\
		\pr{\fdh}                                    & \le N \cdot \eddh + \frac{\mdh \cdot N^2}{q},
	\end{align*}
	concluding the proof.
\end{proof}
