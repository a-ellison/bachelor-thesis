\section{Basic cryptographic definitions} \label{sec:preliminaries-appendix}

\subsection{Notation}

We will use the following notation throughout:
\begin{itemize}
	\item We write $x \from S$ to say that $x$ is sampled u.a.r.\ from the finite set $S$
	\item For $n \in \mathbb{N} \setminus \{0\}$, $[n] = \{1, \ldots, n\}$, and for $a, b \in \N$ s.t.\ $a \le b$, $[a, b] = \{a, a + 1, \ldots, b\}$
	\item If $\mathbb{G}$ is a cyclic group of order $q$ and $g$ a generator, then
	      \begin{itemize}
		      \item We write the group operation in $\mathbb{G}$ multiplicatively
		      \item $h^{-1}$ denotes the inverse of $h \in \mathbb{G}$
		      \item $\log_g(h)$ denotes the unique $x \in [q]$ such that $g^x = h$
	      \end{itemize}
	\item We write $b \from \adv$ to denote the event that an adversary $\adv$ outputs the bit $b$ when playing a game where it must output a bit in the end
	\item For $a, b \in \{0, 1\}^n$, $a \oplus b$ denotes the XOR of $a$ and $b$
	\item We will stick to using $\kappa$ as the security parameter of private-key encryption schemes and $\eta$ as the parameter of public-key encryption schemes
	\item For a function $f$ in the security parameter $\eta$ (or $\kappa$) we will often omit writing $\eta$ as an argument and simply write $f$ to refer to $f(\eta)$
	\item For an encryption scheme $\Pi$ that contains an algorithm $X$, we may refer to $\Pi$'s implementation of $X$ by $\Pi.X$. (E.g. if $\Pi$ is an encryption scheme we can refer to its key-generation algorithm by $\Pi.X$)
\end{itemize}

\subsection{Encryption schemes}

\subsubsection{Private-key encryption}

\begin{definition}[Private-key encryption {\cite[Definition 3.7]{introduction-to-modern-cryptography}}]
	Let $\kappa$ denote the security parameter. A \emph{private-key encryption scheme} $\Pi$ consists of three probabilistic polynomial-time algorithms $(\gen, \enc, \dec)$ such that:
	\begin{enumerate}[1.]
		\item The \emph{key-generation algorithm} $\gen$ takes as input $1^\kappa$ (in unary) and outputs a key $k$. We will write $k \from \gen(1^\kappa)$.
		\item The \emph{encryption algorithm} $\enc$ takes as input a key $k$ and a message $m \in \{0, 1\}^*$, or $m \in \{0, 1\}^{\le l(\kappa)}$ for some function $l$ if the message space is finite, and outputs a ciphertext $c$. We write this as $c \from \enc_k(m)$.
		\item The deterministic \emph{decryption algorithm} $\dec$ takes as input a key $k$ and a ciphertext $c$, and outputs a message $m$ or $\bot$ (denoting an error). We write this as $m = \dec_{k}(c)$.
	\end{enumerate}

	It is required that for every $\kappa$, every key $k$ output by $\gen$, and every message $m$, it holds that $\pr{\dec_k(\enc_k(m)) = m} = 1$ (where the probability is over the randomness of $\enc_k$).
\end{definition}

\subsubsection{Public-key encryption}

In the following definition we will be more explicit about the randomness used by the algorithm $\gen$, as we will require a way to provide the randomness as input.

\begin{definition}[Public-key encryption {\cite[Definition 12.1]{introduction-to-modern-cryptography}}] \label{def:public-key-encryption}
	Let $\eta$ denote the security parameter.
	A \emph{public-key encryption scheme} $\Pi$ consists of three probabilistic polynomial-time algorithms $(\gen, \enc, \dec)$ such that:
	\begin{enumerate}[1.]
		\item The \emph{key-generation algorithm} $\gen$ takes as input $1^\eta$ (in unary) and outputs a pair of keys $(pk, sk)$ (a public and private key). We will write $(pk, sk) \from \gen(1^\eta)$.

		      The public key defines a message space $\mathcal{M}_{pk}$.

		      The algorithm samples $\rho(\eta)$ uniformly random bits to make randomized decisions for some function $\rho$ polynomial in $\eta$. The sequence of random bits $r \in \{0, 1\}^{\rho(\eta)}$ to be used by the algorithm may also be provided as input. We write this as $(pk, sk) = \gen(1^\eta, r)$ to emphasize the fact that the output is deterministic. The distribution over key pairs output by sampling $r \from \{0, 1\}^{\rho(\eta)}$ and running $\gen(1^\eta, r)$ is identical to the distribution over key pairs output by running $\gen(1^\eta)$.


		\item The \emph{encryption algorithm} $\enc$ takes as input a public key $pk$ and a message $m \in \mathcal{M}_{pk}$, and outputs a ciphertext $c$. We write this as $c \from \enc_{pk}(m)$.
		\item The deterministic \emph{decryption algorithm} $\dec$ takes as input a private key $sk$ and a ciphertext $c$, and outputs a message $m$ or $\bot$ (denoting an error). We write this as $m = \dec_{sk}(c)$.
	\end{enumerate}

	It is required that for every $\eta$, every key $(pk, sk)$ output by $\gen$, and every message $m$, it holds that $\pr{\dec_{sk}(\enc_{pk}(m)) = m} = 1$ (where the probability is over the randomness of $\enc_{pk}$).
\end{definition}

\subsection{Security definitions}

\begin{definition}[The IND-CPA game]
	Let $\kappa$ denote the security parameter and let $\Pi$ a private-key encryption scheme. Define the game $\game{\Pi}{\kappa}{IND-CPA}(\adv)$ for an adversary $\adv$:
	\begin{enumerate}[1.]
		\item A key $k \from \gen(1^\kappa)$ is generated.
		\item The adversary $\adv$ is given oracle access to $\Pi.\enc_k$, and outputs a pair of messages $m_0, m_1$ of the same length.
		\item A bit $b \from \{0, 1\}$ is sampled and $\adv$ is given a ciphertext $c \from \enc_k(m_b)$. ($\adv$ continues to have oracle access to $\Pi.\enc_k$.)
		\item $\adv$ outputs a bit $b'$. The output of the game is defined to be $1$ if $b' = b$, and $0$ otherwise.
	\end{enumerate}
\end{definition}

\begin{definition}[IND-CPA security {\cite[Definition 3.21]{introduction-to-modern-cryptography}}]
	For functions $t, \epsilon$ in the security parameter $\kappa$, a private-key encryption scheme $\Pi$ is \emph{$(t, \epsilon)$-IND-CPA-secure} if for all $\kappa$, for any adversary $\adv$ running in time $t(\kappa)$ we have
	\begin{align*}
		\advantage{\Pi}{\kappa}{IND-CPA}(\adv) \coloneqq 2 \cdot \left(\pr{\game{\Pi}{\kappa}{IND-CPA}(\adv) = 1} - \frac{1}{2}\right) \le \epsilon(\kappa).
	\end{align*}
\end{definition}

We will make use of a weaker form of security called ``indistinguishability in the presence of an eavesdropper'' \cite{introduction-to-modern-cryptography} and will refer to it as ``EAV security''. It is identical to IND-CPA security with the sole exception that the adversary does not have access to an encryption oracle.

\begin{definition}[The EAV game]
	Let $\kappa$ denote the security parameter and let $\Pi$ a private-key encryption scheme. Define the game $\game{\Pi}{\kappa}{EAV}(\adv)$ for an adversary $\adv$:
	\begin{enumerate}[1.]
		\item A key $k \from \gen(1^\kappa)$ is generated.
		\item The adversary $\adv$ outputs a pair of messages $m_0, m_1$ of the same length.
		\item A bit $b \from \{0, 1\}$ is sampled and $\adv$ is given a ciphertext $c \from \enc_k(m_b)$.
		\item $\adv$ outputs a bit $b'$. The output of the game is defined to be $1$ if $b' = b$, and $0$ otherwise.
	\end{enumerate}
\end{definition}

\begin{definition}[EAV security {\cite[Definition 3.8]{introduction-to-modern-cryptography}}]
	A private-key encryption scheme $\Pi$ is \emph{$(t, \epsilon)$-EAV-secure} if for all $\kappa$, for any adversary $\adv$ running in time $t(\kappa)$ we have
	\begin{align*}
		\advantage{\Pi}{\kappa}{EAV}(\adv) \coloneqq 2 \cdot \left(\pr{\game{\Pi}{\kappa}{EAV}(\adv) = 1} - \frac{1}{2}\right) \le \epsilon(\kappa).
	\end{align*}
\end{definition}


\begin{lemma}
	Let $\Pi$ a private-key encryption scheme. If $\Pi$ is $(t, \epsilon)$-IND-CPA-secure, then $\Pi$ is $(t, \epsilon)$-EAV-secure.
\end{lemma}
\begin{proof}
	This follows immediately from the fact that any EAV adversary is also an IND-CPA adversary.
\end{proof}

When analyzing the advantage of an adversary we may make use of the following well-known equality.

\begin{lemma}
	Let $X$ a Bernoulli random variable and $b \from \{0, 1\}$ (where $X$ and $b$ are not necessarily independent). Then for $x \in \{0, 1\}$
	\[
		2 \cdot \left(\pr{X = b} - \frac{1}{2}\right) = \pr{X = x \mid b = x} - \pr{X = x \mid b = 1 - x}.
	\]
	In particular, if $\adv$ is an adversary with output in $\{0, 1\}$ playing a game where a bit $b \from \{0, 1\}$ is sampled, then for $x \in \{0, 1\}$
	\begin{equation} \label{eq:advantage-equality}
		2 \cdot \left(\pr{b \from \adv} - \frac{1}{2}\right) = \pr{x \from \adv \mid b = x} - \pr{x \from \adv \mid b = 1 - x}.
	\end{equation}
\end{lemma}
\begin{proof}
	Let $x \in \{0, 1\}$. We have
	\begin{align*}
		2 \cdot \left(\pr{X = b} - \frac{1}{2}\right) & = 2 \cdot \left(\pr{X = x \mid b = x} \cdot \frac{1}{2} + \pr{X = 1 - x \mid b = 1 - x} \cdot \frac{1}{2} - \frac{1}{2}\right) \\
		                                              & = \pr{X = x \mid b = x} + \pr{X = 1 - x \mid b = 1 - x} - 1                                                                    \\
		                                              & = \pr{X = x \mid b = x} - (1 - \pr{X = 1 - x \mid b = 1 - x})                                                                  \\
		                                              & = \pr{X = x \mid b = x} - \pr{X = x \mid b = 1 - x}.
	\end{align*}
\end{proof}

\subsection{The Decisional Diffie-Hellman problem}

\begin{definition}[Group-generation algorithm {\cite[Section 9.3.2]{introduction-to-modern-cryptography}}]
	A \emph{group-generation algorithm} $\mathcal{G}$ is a probabilistic polynomial-time algorithm that takes as input $1^\eta$ and outputs $(\mathbb{G}, q, g)$, where $\mathbb{G}$ is (a description of) a cyclic group with order $q$ and $g \in \mathbb{G}$ is a generator. A group element is represented as a bit-string of length at most $\gamma(\eta)$. We write $(\mathbb{G}, q, g) \from \mathcal{G}(1^\eta)$.
\end{definition}

\begin{definition}[The Decisional Diffie-Hellman (DDH) problem {\cite[Section 9.3.2]{introduction-to-modern-cryptography}}]
	Let $\mathcal{G}$ a group-generation algorithm.
	Define the game $\game{\mathcal{G}}{\eta}{DDH}(\adv)$ for an adversary $\adv$:
	\begin{enumerate}[1.]
		\item $\mathcal{G}(1^\eta)$ is run to obtain $(\mathbb{G}, q, g)$, and exponents $x, y \from [q]$ and a bit $b \from \{0, 1\}$ are sampled.
		\item The adversary $\adv$ is given $\mathbb{G}$, $q$, $g$, $h_1 \coloneqq g^x, h_2 \coloneqq g^y$ and
		      \[
			      k = \begin{cases}
				      g^{x \cdot y} & b = 0 \\
				      \tilde{k}     & b = 1
			      \end{cases}
		      \]
		      where $\tilde{k} \from \mathbb{G}$.
		\item $\adv$ outputs a bit $b'$. The output of the game is defined to be $1$ if $b' = b$, and $0$ otherwise.
	\end{enumerate}
\end{definition}

\begin{definition}[Hardness of the DDH problem {\cite[Definition 9.64]{introduction-to-modern-cryptography}}]
	The DDH problem is \emph{$(t, \epsilon)$-hard relative to} $\mathcal{G}$ if for all $\eta$, for any adversary $\adv$ running in time $t(\eta)$ we have
	\begin{align*}
		\advantage{\mathcal{G}}{\eta}{DDH}(\adv) \coloneqq 2 \cdot \left(\pr{\game{\mathcal{G}}{\eta}{DDH}(\adv) = 1} - \frac{1}{2}\right) \le \epsilon(\eta).
	\end{align*}
\end{definition}

\subsection{The Random Oracle Model} \label{sec:rom}

We will work in the commonly used Random Oracle Model (ROM) to prove our results. We refer the reader to \cite[Chapter 6.5]{introduction-to-modern-cryptography} for an informal overview of the ROM and to \cite{rom} for the original work that introduced the model. The ROM introduces the concept of a \emph{random oracle}. If a function $H : A \to B$ is modelled as a random oracle, then certain assumptions are made about what an adversary $\adv$ knows about $H$ and how it interacts with it:
\begin{itemize}
	\item From $\adv$'s perspective, $H$ is a black-box function. The only way for $\adv$ to interact with $H$ is for it to provide a value $a \in A$ and get back $H(a)$, and this is the only way for $\adv$ to learn $H(a)$. We say that $\adv$ \emph{queries} $H(a)$ or that $\adv$ \emph{queries $H$ for $a$}. This well-defined interface of $\adv$ to $H$ implies that a reduction can extract the queries that $\adv$ makes to $H$.
	\item From $\adv$'s perspective, $H$ is a random variable, sampled u.a.r.\ from the set of all functions from $A$ to $B$. Thus, if $\adv$ queries $H$ for some $a \in A$ that it has not queried before, the value $H(a)$ is a random variable uniformly distributed in $B$ from $\adv$'s perspective.
\end{itemize}
We do not rely on the property known as ``programmability'' in this work.
