\section{Preliminaries}

Our definitions were adapted from \cite{introduction-to-modern-cryptography}. A summary of our notation, some simple lemmas, definitions for private-key and public-key encryption, IND-CPA security, EAV security, group-generation algorithms and the Decisional Diffie-Hellman (DDH) problem, and an explanation of the Random Oracle Model (ROM) can be found in section \ref{sec:preliminaries-appendix} of the appendix.

\subsection{DHIES}

In the following definition we will refer to ``key-derivation functions''. This is only meant as a hint to the reader. We do not provide a definition here, as we will always model such a function as a random oracle (see Section~\ref{sec:rom}).

\begin{definition}[DHIES {\cite[Construction 12.19]{introduction-to-modern-cryptography}}]
	Let $\mathcal{G}$ a group-generation algorithm (for Diffie-Hellman groups). Let $\Pi_s$ a private-key encryption scheme where $\Pi_s.\gen(1^\eta)$ samples a key u.a.r.\ from $\{0, 1\}^\eta$. Let $\fhdh = \{\hdh^{(\eta)} \mid \eta \in \N \}$ a family of key-derivation functions where $\hdh^{(\eta)} \colon \{0, 1\}^* \to \{0, 1\}^{\eta}$. We write $\hdh \coloneqq \hdh^{(\eta)}$ when $\eta$ is clear from the context. Define the algorithms $\gen, \enc$ and $\dec$ as follows:
	\begin{itemize}
		\item $\gen$: on input $1^\eta$ run $\mathcal{G}(1^\eta)$ to obtain $(\mathbb{G}, q, g)$. Sample $x \from [q]$ and set $h_1 \coloneqq g^x$. Set $pk \coloneqq \langle \mathbb{G}, q, g, h_1 \rangle$ and $sk \coloneqq \langle \mathbb{G}, q, g, x \rangle$, and output $(pk, sk)$.

		      The message space is the message space of $\Pi_s$.
		\item $\enc$: on input a public key $\langle \mathbb{G}, q, g, h_1 \rangle$ and a message $m$, sample $y \from [q]$, set $h_2 \coloneqq g^y, k \coloneqq \hdh(h_1^y)$\footnote{Where for $h \in \mathbb{G}$, $\hdh(h)$ denotes the output of $\hdh$ with the binary representation of $h$ given as input.}, compute $c' \from \Pi_s.\enc_k(m)$ and output the ciphertext $\langle h_2, c' \rangle$.
		\item $\dec$: on input a private key $\langle \mathbb{G}, q, g, x\rangle$ and a ciphertext $\langle h_2, c' \rangle$, compute $k \coloneqq H(h_2^x)$ and output $\Pi_s.\dec_k(c')$. If the ciphertext is not of the right form or $\Pi_s.\dec$ outputs $\bot$, output $\bot$.
	\end{itemize}
	The public-key encryption scheme $\dhies \coloneqq (\gen, \enc, \dec)$ is called the Diffie-Hellman Integrated Encryption Scheme (DHIES).

	When using the DHIES scheme later on, we will set $pk \coloneqq h $ and $sk \coloneqq x$ in $\gen$ for simplicity. In practice $\mathbb{G}, q, g$ and $\hdh$ will be known.
\end{definition}

Under the DDH assumption (i.e. the assumption that the DDH problem is hard relative to $\mathcal{G}$), using DHIES with an EAV secure private-key scheme gives an IND-CPA secure public-key encryption scheme in the ROM, as proven in \cite[Theorem~12.12]{introduction-to-modern-cryptography}.
