\section{Preliminaries}

\subsection{Notation}

We will use the following notation throughout:
\begin{itemize}
	\item We write $x \from S$ to say that $x$ is sampled u.a.r.\ from the finite set $S$
	\item For $n \in \mathbb{N} \setminus \{0\}$, $[n] = \{1, \ldots, n\}$, and for $a, b \in \N$ s.t.\ $a \le b$, $[a, b] = \{a, a + 1, \ldots, b\}$
	\item If $\mathbb{G}$ is a cyclic group of order $q$ and $g$ a generator, then
	      \begin{itemize}
		      \item We write the group operation in $\mathbb{G}$ multiplicatively
		      \item $h^{-1}$ denotes the inverse of $h \in \mathbb{G}$
		      \item $\log_g(h)$ denotes the unique $x \in [q]$ such that $g^x = h$
	      \end{itemize}
	\item We write $b \from \adv$ to denote the event that an adversary $\adv$ outputs the bit $b$ when playing a game where it must output a bit in the end
	\item For $a, b \in \{0, 1\}^n$, $a \oplus b$ denotes the XOR of $a$ and $b$
	\item We will stick to using $\kappa$ as the security parameter of private-key encryption schemes and $\eta$ as the parameter of public-key encryption schemes
	\item For a function $f$ in the security parameter $\eta$ (or $\kappa$) we will often omit writing $\eta$ as an argument and simply write $f$ to refer to $f(\eta)$
	\item For an encryption scheme $\Pi$ that contains an algorithm $X$, we may refer to $\Pi$'s implementation of $X$ by $\Pi.X$. (E.g. if $\Pi$ is an encryption scheme we can refer to its key-generation algorithm by $\Pi.X$)
\end{itemize}


\subsection{Basic definitions}

Our definitions were adapted from \cite{introduction-to-modern-cryptography}. We will make use of some well-known concepts including private-key and public-key encryption, IND-CPA security, EAV security and the Random Oracle Model (ROM), and of some simple lemmas. The corresponding definitions and lemmas can be found in section \ref{sec:preliminaries-appendix} of the appendix.

\begin{definition}[Group-generation algorithm {\cite[Section 9.3.2]{introduction-to-modern-cryptography}}] \label{def:group-generation-algorithm}
	Let $\eta$ denote the security parameter. A \emph{group-generation algorithm} $\mathcal{G}$ is a probabilistic polynomial-time algorithm that takes as input $1^\eta$ and outputs $(\mathbb{G}, q, g)$, where $\mathbb{G}$ is (a description of) a cyclic group, $q$ is the order of the group with $q \ge 2^\eta$ and $g \in \mathbb{G}$ is a generator. A group element is represented as a bit-string of length at most $\gamma(\eta)$. We write $(\mathbb{G}, q, g) \from \mathcal{G}(1^\eta)$.
\end{definition}

\begin{definition}[The Decisional Diffie-Hellman (DDH) problem]
	Let $\eta$ denote the security parameter and let $\mathcal{G}$ a group-generation algorithm.
	Define the game $\game{\mathcal{G}}{\eta}{DDH}(\adv)$ for an adversary $\adv$:
	\begin{enumerate}[1.]
		\item $\mathcal{G}(1^\eta)$ is run to obtain $(\mathbb{G}, q, g)$, and exponents $x, y \from [q]$ and a bit $b \from \{0, 1\}$ are sampled.
		\item The adversary $\adv$ is given $\mathbb{G}$, $q$, $g$, $h_1 \coloneqq g^x, h_2 \coloneqq g^y$ and
		      \[
			      k = \begin{cases}
				      g^{x \cdot y} & b = 0 \\
				      \tilde{k}     & b = 1
			      \end{cases}
		      \]
		      where $\tilde{k} \from \mathbb{G}$.
		\item $\adv$ outputs a bit $b'$. The output of the game is defined to be $1$ if $b' = b$, and $0$ otherwise.
	\end{enumerate}
\end{definition}

\begin{definition}[Hardness of the DDH problem {\cite[Definition 9.64]{introduction-to-modern-cryptography}}]
	The DDH problem is \emph{$(t, \epsilon)$-hard relative to} $\mathcal{G}$ if for all $\eta$, for any adversary $\adv$ running in time $t(\eta)$ we have
	\begin{align*}
		\advantage{\mathcal{G}}{\eta}{DDH}(\adv) \coloneqq 2 \cdot \left(\pr{\game{\mathcal{G}}{\eta}{DDH}(\adv) = 1} - \frac{1}{2}\right) \le \epsilon(\eta).
	\end{align*}
\end{definition}

In the following definition we will refer to ``key-derivation functions''. This is only meant as a hint to the reader. We do not provide a definition here, as we will always model such a function as a random oracle (see Section~\vref{sec:rom}).

\begin{definition}[DHIES {\cite[Construction 12.19]{introduction-to-modern-cryptography}}]
	Let $\eta$ denote the security parameter. Let $\mathcal{G}$ a group-generation algorithm. Let $\Pi_s$ a private-key encryption scheme where $\Pi_s.\gen(1^\eta)$ samples a key u.a.r.\ from $\{0, 1\}^\eta$. Let $\fhdh = \{\hdh^{(\eta)} \mid \eta \in \N \}$ a family of key-derivation functions where $\hdh^{(\eta)} \colon \{0, 1\}^* \to \{0, 1\}^{\eta}$. We write $\hdh \coloneqq \hdh^{(\eta)}$ when $\eta$ is clear from the context. Define the algorithms $\gen, \enc$ and $\dec$ as follows:
	\begin{itemize}
		\item $\gen$: on input $1^\eta$ run $\mathcal{G}(1^\eta)$ to obtain $(\mathbb{G}, q, g)$. Sample $x \from [q]$ and set $h_1 \coloneqq g^x$. Set $pk \coloneqq \langle \mathbb{G}, q, g, h_1 \rangle$ and $sk \coloneqq \langle \mathbb{G}, q, g, x \rangle$, and output $(pk, sk)$.

		      The message space is the message space of $\Pi_s$.
		\item $\enc$: on input a public key $\langle \mathbb{G}, q, g, h_1 \rangle$ and a message $m$, sample $y \from [q]$, set $h_2 \coloneqq g^y, k \coloneqq \hdh(h_1^y)$\footnote{Where for $h \in \mathbb{G}$, $\hdh(h)$ denotes the output of $\hdh$ with the binary representation of $h$ given as input.}, compute $c' \from \Pi_s.\enc_k(m)$ and output the ciphertext $\langle h_2, c' \rangle$.
		\item $\dec$: on input a private key $\langle \mathbb{G}, q, g, x, \hdh \rangle$ and a ciphertext $\langle h_2, c' \rangle$, compute $k \coloneqq H(h_2^x)$ and output $\Pi_s.\dec_k(c')$. If the ciphertext is not of the right form or $\Pi_s.\dec$ outputs $\bot$, output $\bot$.
	\end{itemize}
	The public-key encryption scheme $\dhies \coloneqq (\gen, \enc, \dec)$ is called the Diffie-Hellman Integrated Encryption Scheme (DHIES).

	When using the DHIES scheme later on, we will set $pk \coloneqq h $ and $sk \coloneqq x$ in $\gen$ for simplicity. In practice $\mathbb{G}, q, g$ and $\hdh$ will be known.
\end{definition}

Under the DDH assumption (i.e. the assumption that the DDH problem is hard relative to $\mathcal{G}$), using DHIES with an EAV secure private-key scheme gives an IND-CPA secure public-key encryption scheme in the ROM, as proven in \cite[Theorem~12.12]{introduction-to-modern-cryptography}.
