\section{Our concrete result for TreeKEM}

The following theorem, stated informally here, is our main practical result.

\begin{theorem}[Informal] \label{theorem:treekem-security-informal}
	If the DHIES scheme is used in TreeKEM, the private-key encryption scheme in DHIES is $(t, \eeav)$-EAV-secure and the DDH problem is $(t, \eddh)$-hard in the Diffie-Hellman group, then for all $c, p, u$, TreeKEM is $(\tilde{t}, \tilde{\epsilon}, c, p, u)$-secure in the ROM with $\tilde{t} \approx t$ and
	\begin{align*}
		\begin{split}
			\tilde{\epsilon} ={} & 2 \cdot u \cdot (3 \cdot c \cdot \log(u) + p) \cdot \eeav \\
			& + 2 \cdot (3 \cdot c \cdot \log(u) + p) \cdot \eddh \\
			& + \mathrm{negl}(\eta)
		\end{split}
	\end{align*}
	where
	\begin{itemize}
		\item $c$ is the number of commits created
		\item $p$ is the number of add or update proposals created
		\item $u$ is the maximum number of users
	\end{itemize}
\end{theorem}

In Section~\ref{sec:treekem-security} of the appendix we restate the above theorem formally as Theorem~\ref{theorem:treekem-security}. To this end, we also provide formal definitions for the syntax and security of propose and commit CGKA schemes and a high-level description of how to instantiate (the essence of) the TreeKEM protocol with our definitions.

\subsection{Interpreting the result}

In the following we will go through some concrete examples to see what level of security our security proof guarantees for TreeKEM with different parameter choices. We will look at the \texttt{MLS\_128\_DHKEMX25519\_AES128GCM\_SHA256\_Ed25519} cipher suite \cite[Section~17.1]{rfc9420} for 128-bit parameters, which uses Curve25519 as the Diffie-Hellman group and AES with a 128-bit key size for private-key encryption. We will assume that both Curve25519 and AES have a 128-bit security level, so for example we can set $(t, \eeav) = (t, \eddh) = (2^{64}, 2^{-64})$.

For 256-bit parameters, we will look at the \texttt{MLS\_256\_DHKEMP521\_AES256GCM\_SHA512\_P521} cipher suite, which uses curve P-521 and 256-bit AES. We will assume that P-521 and 256-bit AES have a 256-bit security level and set $(t, \eeav) = (t, \eddh) = (2^{128}, 2^{-128})$.

\subsubsection{Large groups with hourly commits and frequent updates}

In this example we consider a large group of about 10'000 users, existing for 5 years and making one commit every hour. Then $u \le 2^{14}$ and $c \le 2^{16}$. In this example we assume that a significant fraction of the users will want to update with every commit. Then, assuming that add proposals are relatively rare, we can bound $p \le c \cdot u = 2^{30}$. This implies $3 \cdot c \cdot \log(u) + p \le 2^{31}$, dominated by $p$.

Then with 128-bit parameters we get
\[
	\et \le 2^{46} \cdot \eeav + 2^{32} \cdot \eddh + \mathrm{negl} \le \frac{1}{2^{17}}
\]
with the $\eeav$ term dominating the result. This only gives a security level of $\tilde{t}/\et \ge 2^{81}$. Since private-key encryption is relatively cheap, using 256-bit AES would have a small impact on the performance and would increase the security level to 95 bits (with the $\eddh$ term now dominating). Finally, using full 256-bit parameters yields 209 bits of security for TreeKEM.

The previous best result in \cite[Theorem 3]{ttkem} proved a bound that implies
\[
	\et \le 2 \cdot (3 \cdot c \cdot \log(u) + p)^2 \cdot \epsilon + \mathrm{negl}
\]
where $\epsilon$ is the IND-CPA security of the underlying public-key encryption scheme. If we assume that DHIES has an $x$-bit security level as a public-key encryption scheme with $x$-bit parameters, the result implies 64 bits of security with 128-bit parameters (with no change when using 256-bit AES) and 192 bits with 256-bit parameters.

\subsubsection{Few updates}

In this example we use the same number of users and commits, but assume that the number of proposals is small such that $p \le 3 \cdot c \cdot \log(u)$. In this case we have $3 \cdot c \cdot \log(u) + p \le 2^{22}$. Then our result guarantees 90 bits of security with 128-bit parameters, 104 bits with 256-bit AES and 218 bits with 256-bit parameters.

In constrast, the bound in \cite{ttkem} implies 82 bits with 128-bit parameters and 210 bits with 256-bit parameters.

\subsubsection{Very large groups with one commit every minute and frequent updates} In this example we consider more extreme values for $c$ and $u$ to highlight the gap between our result and the one proven in \cite{ttkem}. We assume a group of about 1 million users, existing for 50 years making one commit every minute. Furthermore, we again will again bound $p$ by $c \cdot u$. This means that $u \le 2^{20}$, $c \le 2^{25}$ and $3 \cdot c \cdot \log(u) + p \le 2^{46}$.

These values imply a 61 bits of security with 128-bit parameters, 81 bits with 256-bit AES and 189 bits with 256-bit parameters using our result. The result in \cite{ttkem} implies 34 bits of security with 128-bit parameters and 162 bits with 256-bit parameters.
