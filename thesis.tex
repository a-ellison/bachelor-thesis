%% (Master) Thesis template
% Template version used: v1.4
%
% Largely adapted from Adrian Nievergelt's template for the ADPS
% (lecture notes) project.


\documentclass{llncs}

%% Packages
%% ========

%% LaTeX Font encoding -- DO NOT CHANGE
\usepackage[OT1]{fontenc}

%% Babel provides support for languages.  'english' uses British
%% English hyphenation and text snippets like "Figure" and
%% "Theorem". Use the option 'ngerman' if your document is in German.
%% Use 'american' for American English.  Note that if you change this,
%% the next LaTeX run may show spurious errors.  Simply run it again.
%% If they persist, remove the .aux file and try again.
\usepackage[english]{babel}

%% Input encoding 'utf8'. In some cases you might need 'utf8x' for
%% extra symbols. Not all editors, especially on Windows, are UTF-8
%% capable, so you may want to use 'latin1' instead.
\usepackage[utf8]{inputenc}

%% The AMS-LaTeX extensions for mathematical typesetting.  Do not
%% remove.
\usepackage{amsmath,amssymb,amsfonts,mathrsfs}

%% LaTeX' own graphics handling
\usepackage{graphicx}

%% This allows you to add .pdf files. It is used to add the
%% declaration of originality.
\usepackage{pdfpages}

%% Some more packages that you may want to use.  Have a look at the
%% file, and consult the package docs for each.
\input{config/extrapackages}

%% Helpful macros.
%% Special characters for number sets, e.g. real or complex numbers.
\newcommand{\C}{\mathbb{C}}
\newcommand{\K}{\mathbb{K}}
\newcommand{\N}{\mathbb{N}}
\newcommand{\Q}{\mathbb{Q}}
\newcommand{\R}{\mathbb{R}}
\newcommand{\Z}{\mathbb{Z}}
\newcommand{\X}{\mathbb{X}}

%% Fixed/scaling delimiter examples (see mathtools documentation)
\DeclarePairedDelimiter\abs{\lvert}{\rvert}
\DeclarePairedDelimiter\norm{\lVert}{\rVert}
\DeclarePairedDelimiter\bracks{\lbrack}{\rbrack}

%% Use the alternative epsilon per default and define the old one as \oldepsilon
\let\oldepsilon\epsilon
\renewcommand{\epsilon}{\ensuremath\varepsilon}

%% Also set the alternate phi as default.
\let\oldphi\phi
\renewcommand{\phi}{\ensuremath{\varphi}}

%% Custom commands

\newcommand{\question}[1]{\textcolor{magenta}{\textbf{Q:} #1}}
\newcommand{\todo}[1]{\textcolor{red}{\textbf{TODO:} #1}}
\newcommand{\from}{\leftarrow}
\newcommand{\gen}{\mathrm{Gen}}
\newcommand{\enc}{\mathrm{Enc}}
\newcommand{\dec}{\mathrm{Dec}}
\newcommand{\hgen}{H_\mathrm{gen}}
\newcommand{\fhgen}{\mathcal{H}_\mathrm{gen}}
\newcommand{\qgen}{Q_\mathrm{gen}}
\newcommand{\hdep}{H_\mathrm{dep}}
\newcommand{\fhdep}{\mathcal{H}_\mathrm{dep}}
\newcommand{\qdep}{Q_\mathrm{dep}}
\newcommand{\hdh}{H_\mathrm{DH}}
\newcommand{\fdh}{F_\mathrm{DH}}
\newcommand{\qdh}{Q_\mathrm{DH}}
\newcommand{\mdh}{m_\mathrm{DH}}
\newcommand{\qs}{Q_\mathrm{s}}
\newcommand{\ms}{m_\mathrm{s}}

% adversary
\newcommand{\adv}{\mathcal{A}}
\newcommand{\wins}{\mathcal{A} \text{ wins}}

\newcommand{\sdgsdgame}[2][\mathcal{A}]{\mathrm{Game}_{#1, #2}^{\mathrm{SD-GSD}}}
\newcommand{\pr}[1]{\Pr\bracks*{#1}}

% schemes
\newcommand{\dhies}{\Pi_{\mathrm{DH}}}
\newcommand{\treekem}{\Sigma_{\mathrm{TK}}}

%% ===============


%% Make document internal hyperlinks wherever possible. (TOC, references)
%% This MUST be loaded after varioref, which is loaded in 'extrapackages'
%% above.  We just load it last to be safe.
\usepackage[bookmarks,bookmarksopen,bookmarksdepth=2,linkcolor=black,colorlinks=true,citecolor=black,filecolor=black]{hyperref}


%% Document information
%% ====================

\title{Tighter provable security for TreeKEM}
\author{}
\institute{}

\begin{document}

\maketitle

\pagestyle{plain}

\begin{abstract}
	The Messaging Layer Security (MLS) protocol, recently standardized in RFC 9420, aims to provide efficient asynchronous group key establishment with strong security guarantees. The main component of MLS, which is the source of its key efficiency and security properties, is a protocol called TreeKEM. Given that a major vision for the MLS protocol is for it to become the new standard for messaging applications like WhatsApp, Facebook Messenger, Signal, etc., it has the potential to be used by a huge number of users. Thus, it is important to better understand the security of MLS and hence also of TreeKEM. In work by Klein et.\ al, TreeKEM was proven adaptively secure in the Random Oracle Model (ROM) with a polynomial loss in security by proving a result about the security of an arbitrary IND-CPA secure public-key encryption scheme in a public-key version of the Generalized Selective Decryption (GSD) security game. \\


	In this work, we prove a tighter bound for the security of TreeKEM that implies meaningful security guarantees in practice. We follow the approach of Klein et.\ al and first introduce a modified version of the public-key GSD game better suited for analyzing TreeKEM. We then provide a simple and detailed proof of security for a specific encryption scheme, the DHIES scheme (currently the only standardized scheme in MLS), in this game in the ROM and achieve a smaller security loss compared to the result from Klein et.\ al. Finally, we state the result on the security of TreeKEM implied by this bound, and give an interpretation of the result with protocol parameters used in practice.

	\keywords{Messaging Layer Security \and TreeKEM \and Secure Messaging \and Group Key-Agreement \and Adaptive Security \and DHIES}
\end{abstract}


\chapter{Introduction}

We all rely on messaging applications like WhatsApp, Facebook Messenger, Signal, etc.\ in our daily lives and take it for granted that our messages will be transmitted securely, for some definition of ``secure''. A common security feature expected from the protocol employed in a messaging application and known also to the general public is end-to-end encryption, i.e.\ that only the end users of a messaging session can read the messages being sent and the service provider or any party with access to the communication channel learns nothing of their contents. Another straightforward feature is that the protocol should work in an asynchronous setting: we would like to send messages even when the recipient is offline, and we expect them to receive the messsage once they come online. For this we must rely on a delivery service to store and deliver the messages. Of course also this delivery service should learn nothing about the contents of the messages.

There are two more advanced security features expected from messaging protocols today, both related to security in case a user is compromised:
\begin{itemize}
	\item forward secrecy (FS): the compromise should not reveal the contents of old messages
	\item post-compromise security (PCS): after the user recovers from the compromise, new messages are secure once again
\end{itemize}
As a user may well not know that they have been compromised, ensuring PCS requires regularly updating the key material used for encryption (in a way that the information leaked in a compromise \emph{before} the update does not suffice to compute encryption keys used \emph{after} the update). The more often the key material is updated, the stronger the level of PCS that is achieved. Thus, updating the key material should be an efficient operation.

For messaging between two users, the Double Ratchet protocol \cite{double-ratchet}, the main component of the so-called Signal Protocol, is a widely adopted solution used by major messaging applications such as Signal, WhatsApp, Facebook Messenger and more. It is well studied and achieves all of the above security guarantees \cite{double-ratchet-analysis}. For messaging in a group of more than two users, a straightforward solution is to maintain 1:1 communication channels using the Double Ratchet protocol between every pair of users and send messages to the group by sending them to every member individually. This achieves very strong security guarantees, but requires a number of encryption operations linear in the group size to send a message.

Another common solution is to use sender keys \cite{sender-keys}: every user creates a symmetric key, their \emph{sender key}, and distributes this sender key to every other user using 1:1 channels as before. A user sending a message then derives a symmetric encryption key for the message from their sender key, while continually updating their sender key (with each sent message) to provide FS. However, achieving PCS is costly: if a user is compromised, the sender keys of all users are leaked and recovering from the compromise requires each user to send a new sender key to every other user over the respective 1:1 channels, resulting in a number of operations linear in the group size per user and a quadratic number of operations in total. Moreover, dynamic group membership introduces additional complexity:
\begin{itemize}
	\item adding a new member involves the new member sharing their sender key with all other group members
	\item removing a member requires distributing new sender keys in the group, just like recovering from a compromise
\end{itemize}
The Messaging Layer Security (MLS) protocol, recently standardized in \cite{rfc9420}, proposes a solution for group messaging with better efficiency and the same strong security guarantees as for the two-party case. Updating key material and adding or removing members can be achieved with a logarithmic number of operations (although the complexity may still degrade to linear in certain scenarios). At the core of MLS is a fairly recent primitive called a \emph{continuous group key agreement} (CGKA) scheme \cite{rtreekem} (this primitive was introduced only \emph{after} the first draft of the MLS protocol). In essence, a CGKA scheme enables a group of users to agree on a \emph{group key}, which they can then use to derive symmetric message encryption keys. This key must be indistinguishable from a random key for anyone outside the group eavesdropping on all communication. However, a CGKA scheme must also achieve FS and PCS, and support dynamic group membership. Hence, it must provide mechanisms for members to update their key material, add new users to the group and remove members from the group. Moreover, the scheme must work in the asynchronous setting with an untrusted service to deliver protocol messages.

The CGKA scheme used in the MLS protocol is called TreeKEM (initially proposed in \cite{treekem}) and the majority of the literature on MLS is dedicated to analyzing TreeKEM or proposing better CGKA schemes as in \cite{ttkem,rtreekem,insider-security,modular-group-messaging}. The TreeKEM protocol has undergone multiple changes since its inception. In this work we refer to the version documented in RFC 9420. TreeKEM, as adopted from its predecessors, maintains a binary tree where every node in the tree has some associated secrets, every member of the group is associated with a leaf and the group key is derived from the root of the tree. Every member can compute the group key from their view of the tree. The group key can be updated and members added/removed with a number of operations logarithmic in the group size.

Given that the vision for the MLS protocol is for it to become the new standard for messaging protocols and that it has support from several large companies \cite{google-mls,mls-support}, it has the potential to be used by a huge number of users. Thus, understanding the security of MLS and hence also of TreeKEM is of great importance. This means having formal security guarantees about the security provided by TreeKEM (based on appropriate hardness assumptions). The first important step in this direction was the conception of the CGKA primitive and the accompanying definitions of security introduced in different works (for example \cite{rtreekem,ttkem}). Such definitions clarify what kind of adversaries we can provide security against and thus what kind of security one should expect from the scheme when using it in practice. Moreover, proofs of (reasonably tight) security under these definitions show what level of security we should expect from the scheme and serve as a guide to implementors on what values to choose for the security parameter. Proofs also provide strong justification that there are no flaws in the overall design of the scheme.

One choice that can be made when defining the security of a CGKA scheme is whether the adversary is modeled as \emph{selective} or \emph{adaptive}. In the former case, the adversary must provide all the interactions it will have with the protocol and when it will attempt to break the scheme at the beginning of the security game, while in the latter case the adversary can make its decisions based on responses from previous interactions. Clearly, the adaptive setting is much closer to how an attack would unfold in practice, so it is desirable to prove security against adaptive adversaries. However, achieving this without too much of a blow-up in the security loss is a challenge since one often resorts to guessing actions performed by the adversary.

The Generalized Selective Decryption (GSD) security game \cite{gsd} was introduced precisely to analyze adaptive security for protocols based on a graph-like structure (as is the case with TreeKEM). It was initially defined for the private-key setting and later adapted to the public-key setting in \cite{ttkem}. The work in \cite{ttkem} proved a polynomial bound for the adaptive security of the public-key GSD game in the so-called Random Oracle Model (ROM) for an arbitrary IND-CPA secure public-key encryption scheme. This result implies a polynomial bound for the adaptive security of TreeKEM as a CGKA scheme as outlined in \cite[Theorem 4]{ttkem} and subsequently proved in more detail in \cite[Theorem 12]{modular-group-messaging}.

In this work, we formally prove the adaptive security of a specific public-key encryption scheme, the DHIES scheme, in a modified version of the public-key GSD game, adapted to better model TreeKEM, in the ROM. Focusing on the DHIES scheme allows us to achieve a tighter bound than the one in \cite{ttkem}. Moreover, we define the syntax and security of propose and commit CGKA schemes, provide a high-level description of how the TreeKEM protocol can be instantiated with our definitions and state a result in the ROM relating the security of a public-key encryption scheme in our modified GSD game with the security of TreeKEM as a CGKA scheme when instantiated with this public-key encryption scheme.

\section{Contributions}

We present the following main contributions:
\begin{itemize}
	\item A simple definition of an adaptation of the public-key GSD game better suited to model TreeKEM.
	\item A simple and detailed proof of (adaptive) security of the DHIES scheme with respect to our GSD definition. We achieve a tighter bound than the one proven so far.
	\item Simple and clear definitions of the syntax and security of propose and commit CGKA schemes. We explain our definition in detail and briefly discuss the correctness of CGKA schemes.
	\item A high-level description of the TreeKEM protocol and how it can be instantiated with respect to our CGKA definition. Finally, we state a result relating the security of our GSD game and the security of TreeKEM, and provide an outline of the proof.
\end{itemize}

\section{Technical overview}

\subsection{The GSD game}

In the GSD security game, given an encryption scheme a graph, the \emph{GSD graph}, is constructed by the challenger where every node in the graph is associated with a symmetric key in the private-key setting, or a public/private key pair in the public-key setting. The adversary can then request encryptions of a node's (secret) key under the (public) key of another node. In the public-key setting, such an \emph{encryption query} also reveals the node's public key. This creates an \emph{encryption edge} in the graph, directed from the node whose (public) key was used for encryption to the node whose key was encrypted. The adversary can also corrupt any node, which reveals its (secret) key and allows the adversary to compute the (secret) key of any other node reachable from the corrupted node in the graph by performing decryptions along the path to the other node. At the end of the game the adversary chooses a node to be challenged on, the \emph{challenge node}. A coin is then tossed and the adversary is given either the (secret) key of the challenge node or a uniformly random (secret) key and it must guess which scenario it is in. The possible choices for the challenge node must of course be restricted to nodes whose keys were not compromised through a corruption, meaning that the challenge node should never be reachable from a corrupted node in the graph. Further restrictions are also necessary which we do not go into here. Figure~\ref{fig:gsd-example} illustrates what an example GSD graph may look like.

\begin{figure}
	\begin{center}
		\includegraphics{figures/gsd-example}
	\end{center}
	\caption{An illustration of the GSD graph for an instance of the GSD game. The challenge node is $v$. The node $c$ was corrupted, resulting in all nodes reachable from it being compromised, as marked with red color.} \label{fig:gsd-example}
\end{figure}

\subsection{The TreeKEM protocol} \label{sec:treekem-overview}

\subsubsection{Propose and commit syntax}

As a CGKA scheme, TreeKEM must support operations for updating the key material of a group member, adding a new user and removing a member. The syntax for these operations has changed over time. In the current version of MLS, the protocol uses so-called \emph{proposals} and \emph{commits}. Whenever a user would like have their key material updated (by someone else), add a new user or remove a group member, they create a corresponding \emph{update}, \emph{add} or \emph{remove proposal}, respectively, and share this proposal with the group. Any group member can then create a \emph{commit} to apply a set of proposals, create a new group key and update their key material in the process. The commit object includes (encrypted) information such that every group member can update their view of the group and compute the new group key.

\subsubsection{TreeKEM dynamics}

As already outlined briefly, TreeKEM uses a full binary tree to model the group. Every user, associated with a leaf in the tree, maintains a synchronized view of the tree, though different users will know more about different parts of the tree. The group key is derived from the root of the tree. Every node $n$ in the tree has an associated key pair $(pk_n, sk_n)$ output by $\Pi.\gen$ where $\Pi$ is a public-key encryption scheme. All public keys are known to all users. Let the \emph{direct path} of a leaf be the path from the leaf's first parent to the root. Every user at a leaf knows the secret key of their leaf and, in the usual case, the secret keys of all nodes on their direct path, though we will see exceptions to this rule later. To illustrate the scheme and how commit operations are performed, we will consider of a group with users $A, B, \ldots, G$ and $H$, as depicted in Figure~\ref{fig:treekem-tree}. In the following, we will use these labels for the users both to refer to the users themselves and to their nodes in the tree.

\begin{figure}
	\begin{center}
		\includegraphics{figures/treekem-tree}
	\end{center}
	\caption{Illustration of a group with users 8 users in the TreeKEM protocol.}\label{fig:treekem-tree}
\end{figure}

\paragraph{Simple commits} The idea behind this tree structure is that it allows for a user creating a commit with a new group key to share the new group key with the group using only a few encryptions, while still updating all the secrets the user knew in the tree in order to recover from a possible compromise (recall that in a PC-CGKA scheme a commit also updates the committer's key material).
To illustrate how a commit is performed and how the new group key is computed, say user $A$ performs a commit. First we only consider commits without any proposals. TreeKEM specifies two hash functions $\hgen, \hdep \colon \{0, 1\}^{\rho(\eta)} \to \{0, 1\}^{\rho(\eta)}$ where $\rho(\eta)$ gives the number of bits of randomness used by $\Pi.\gen(1^\eta)$. Let $d = 3$ the depth of user $A$. $A$ will replace all the $d + 1$ nodes on their path to the root (including their leaf) with new nodes $A, p_1, \ldots, p_d$. Although it would be more accurate to say that $A$ just replaces the information stored in the original nodes, and this view makes more sense when implementing the protocol, it will become convenient later to say that $A$ creates new nodes.
The key pairs for the new nodes are sampled as follows. For the leaf node $A$, user $A$ simply samples a key pair by running $\Pi.\gen(1^\eta)$. For the remaining nodes, they first sample $s_1 \from \{0, 1\}^{\rho(\eta)}$ and compute the key pair of the first parent $p_1$ as $\Pi.\gen(1^\eta, \hgen(s_1))$. For $i \in \{2, \ldots, d\}$ they then compute $s_i \coloneqq \hdep(s_{i - 1})$ and set the key pair of $p_i$ to be $\Pi.\gen(1^\eta, \hgen(s_i))$. The new group key is $k \coloneqq \hdep(s_{d})$.

User $A$ only needs to share (encryptions of) the seeds $s_i$ for the other users to update their view of the tree and compute the new group key:
\begin{itemize}
	\item To share the group key with user $B$, $A$ computes the ciphertext $c_1 \coloneqq \Pi.\enc_{pk_B}(s_1)$. $B$ can then compute the seed $s_1$, then use that to compute the seeds $s_2, \ldots, s_d$, the key pairs of all new nodes on their path to the root and the group key $k$.
	\item To share the new group key with users $C$ and $D$, $A$ computes the ciphertext $c_2 \coloneqq \Pi.\enc_{pk_X}(s_2)$, where $X$ is the parent of the nodes $C$ and $D$. Both $C$ and $D$ know the secret key $sk_X$ of their parent and can decrypt $c_2$.
	\item To share the new group key with users $E, F, G$ and $H$, $A$ computes the ciphertext $c_3 \coloneqq \Pi.\enc_{pk_Y}(s_3)$, where $Y$ is the right child of the root node. Again, all users under $Y$ know $sk_Y$ and can thus decrypt $c_3$.
\end{itemize}
The commit $c$ that $A$ shares with all users includes the ciphertexts $c_1, c_2$ and $c_3$ and the public keys of all new nodes. Figure~\ref{fig:treekem-simple-update} illustrates the commit performed by $A$.

\begin{figure}
	\begin{center}
		\includegraphics{figures/treekem-simple-update}
	\end{center}
	\caption{The commit by user $A$ described in the text. Dashed directed edges illustrate the fact that the target is related to the source via the hash function \question{Should I call this a PRF instead?} $\hdep$. The solid directed edges illustrate the fact that the seed of the target node is encrypted to the public key of the source node.}\label{fig:treekem-simple-update}
\end{figure}

The nodes $B, X$ and $Y$ form the \emph{copath} of $A$: the copath of a node $v$ consists of the sibling of each node on $v$'s path to the root, excluding the root itself. In the ideal case as above, a node performing a commit only has to compute one encryption for each node on its copath, i.e. logarithmically many encryptions in the total number of users.

\paragraph{Remove and update proposals} Things look a bit different if the commit contains a remove proposal. Say user $A$ creates a commit that contains a remove proposal for user $E$. We could just let $A$ replace the nodes on $E$'s direct path $E$ and not encrypt anything for $E$. However, if $A$ were compromised while replacing $E$'s direct path and performed another commit to update their key material after the compromise, the information leaked in the compromise could still be used to compute the new group key, as it includes the secret keys of the nodes on $E$'s direct path.
Instead, $E$'s leaf and all nodes on $E$'s direct path are replaced by \emph{blank} nodes: nodes with no associated key pair. Now $A$ has to encrypt the secret $s_3$ directly to $F$ and to the parent node of $G$ and $H$ in the commit removing user $E$. See Figure~\ref{fig:treekem-remove}. A blank leaf node can be populated with a new user. A blank node that is not a leaf will be replaced by a non-blank node once some user in the node's subtree performs a commit. Blank nodes are also useful to represent the nodes of a subtree with no users.

\begin{figure}
	\begin{center}
		\includegraphics{figures/treekem-remove}
	\end{center}
	\caption{The commit by user $A$ removing user $E$ described in the text. Nodes with a dashed border represent the (new) blank nodes.}\label{fig:treekem-remove}
\end{figure}

Creating a commit with an update proposal for user a $u$ is analogous. The update proposal simply contains the public key of user $u$'s new leaf, while $u$ stored the corresponding secret key locally when creating the proposal. Because we don't want the committer to know the secret keys along $u$'s direct path, we must again replace these nodes with blank ones and encrypt to the non-blank nodes below directly.

\paragraph{Add proposals} Adding a user introduces one new but similar complication. Consider the same group as in Figure~\ref{fig:treekem-tree}, but with the leaf of user $H$ blank. Now say user $A$ would like to add user $H$ to the group. Although we would want $H$ to know all secret keys on their direct path, $A$ can only provide the secrets of their lowest common ancestor, which is the root node in this case.
In such a situation where a non-blank node $n$ has a leaf $l$ below it where $l$ does not know $n$'s secret key, we say that $l$ is \emph{unmerged} relative to $n$. Every non-blank, non-leaf node stores its list of unmerged leaves and whenever one encrypts to a node, one should also encrypt to all its unmerged leaves. A user's leaf becomes ``merged'' as the nodes on their direct path are replaced and they are provided the seeds to compute the secret keys of the new nodes.
Note that for any non-leaf node $n$, any one of its descendants $d$ and any unmerged leaf of $n$ that is a descendant of $d$, this leaf must also be an unmerged leaf of $d$: every commit that replaces $d$ also replaces the node $n$ and if a user at a leaf learns the secret key of the new node for $d$ through its seed, they also learn the seed of and therefore the secret key of the new node for $n$. Figure~\ref{fig:treekem-add} shows a commit by user $A$ adding $H$, followed by another commit by user $E$.

\begin{figure}
	\centering
	\begin{subfigure}[b]{\textwidth}
		\centering
		\includegraphics[width=\textwidth]{figures/treekem-add-1}
		\caption{User $A$ adds user $H$. As a small detail: the encryption for user $H$ is computed using $H$'s init key instead of the public key of their leaf.}
		\label{fig:treekem-A-add-H}
	\end{subfigure}
	\begin{subfigure}[b]{\textwidth}
		\centering
		\includegraphics[width=\textwidth]{figures/treekem-add-2}
		\caption{Commit by user $E$. $H$ is now ``merged'' relative to node $Y'$.}
	\end{subfigure}
	\caption{A commit adding user $H$ and another commit by a user $E$ as described in the text. Orange edges illustrate the fact that the target leaf is unmerged relative to the source node. In \subref{fig:treekem-A-add-H}, $H$ is also unmerged relative to $Y$'s right child, but this information is redundant as it follows from $H$ being unmerged relative to $Y$.}
	\label{fig:treekem-add}
\end{figure}

\paragraph{Resolution} We have now seen that when performing a commit, one must pay attention to blank nodes and unmerged leaves when providing encryptions. Instead of only providing encryptions for each node on the copath as in the ideal case, in the general case for each node $n$ on the copath one must provide an encryption for every node in the \emph{resolution} of $n$. The resolution of a non-blank node is the node itself and the set of all its unmerged leaves. The resolution of a blank leaf is the empty set and the resolution of a blank, non-leaf node is the union of the resolutions of its two children.

\paragraph{Key packages and welcome messages} To encrypt to an existing group member it is clear that we can just use the public key in their leaf. But how do we encrypt to a new user? Before a user joins any group, they publish a \emph{key package}: this contains (among other things) the public key, their so-called \emph{init key}, to be used to encrypt information to the user when they join the group and the public key that should be associated with the user's leaf. The key package is included (or referenced) in the add proposal for the new user. Along with the seed of the committer's and the new user's lowest common ancestor in the tree, the new user must also be given the (public) state of the tree. This information is provided to the new user, encrypted with their init key, by the committer in a \emph{welcome message}.


\section{Preliminaries}

\subsection{Notation}

We will use the following notation throughout:
\begin{itemize}
	\item We write $x \from S$ to say that $x$ is sampled u.a.r.\ from the finite set $S$
	\item For $n \in \mathbb{N} \setminus \{0\}$, $[n] = \{1, \ldots, n\}$, and for $a, b \in \N$ s.t.\ $a \le b$, $[a, b] = \{a, a + 1, \ldots, b\}$
	\item If $\mathbb{G}$ is a cyclic group of order $q$ and $g$ a generator, then
	      \begin{itemize}
		      \item We write the group operation in $\mathbb{G}$ multiplicatively
		      \item $h^{-1}$ denotes the inverse of $h \in \mathbb{G}$
		      \item $\log_g(h)$ denotes the unique $x \in [q]$ such that $g^x = h$
	      \end{itemize}
	\item We write $b \from \adv$ to denote the event that an adversary $\adv$ outputs the bit $b$ when playing a game where it must output a bit in the end
	\item For $a, b \in \{0, 1\}^n$, $a \oplus b$ denotes the XOR of $a$ and $b$
	\item We will stick to using $\kappa$ as the security parameter of private-key encryption schemes and $\eta$ as the parameter of public-key encryption schemes
	\item For a function $f$ in the security parameter $\eta$ (or $\kappa$) we will often omit writing $\eta$ as an argument and simply write $f$ to refer to $f(\eta)$
	\item For an encryption scheme $\Pi$ that contains an algorithm $X$, we may refer to $\Pi$'s implementation of $X$ by $\Pi.X$. (E.g. if $\Pi$ is an encryption scheme we can refer to its key-generation algorithm by $\Pi.X$)
\end{itemize}


\subsection{Basic definitions}

Our definitions were adapted from \cite{introduction-to-modern-cryptography}. We will make use of some well-known concepts including private-key and public-key encryption, IND-CPA security, EAV security and the Random Oracle Model (ROM), and of some simple lemmas. The corresponding definitions and lemmas can be found in section \ref{sec:preliminaries-appendix} of the appendix.

\begin{definition}[Group-generation algorithm {\cite[Section 9.3.2]{introduction-to-modern-cryptography}}] \label{def:group-generation-algorithm}
	Let $\eta$ denote the security parameter. A \emph{group-generation algorithm} $\mathcal{G}$ is a probabilistic polynomial-time algorithm that takes as input $1^\eta$ and outputs $(\mathbb{G}, q, g)$, where $\mathbb{G}$ is (a description of) a cyclic group, $q$ is the order of the group with $q \ge 2^\eta$ and $g \in \mathbb{G}$ is a generator. A group element is represented as a bit-string of length at most $\gamma(\eta)$. We write $(\mathbb{G}, q, g) \from \mathcal{G}(1^\eta)$.
\end{definition}

\begin{definition}[The Decisional Diffie-Hellman (DDH) problem]
	Let $\eta$ denote the security parameter and let $\mathcal{G}$ a group-generation algorithm.
	Define the game $\game{\mathcal{G}}{\eta}{DDH}(\adv)$ for an adversary $\adv$:
	\begin{enumerate}[1.]
		\item $\mathcal{G}(1^\eta)$ is run to obtain $(\mathbb{G}, q, g)$, and exponents $x, y \from [q]$ and a bit $b \from \{0, 1\}$ are sampled.
		\item The adversary $\adv$ is given $\mathbb{G}$, $q$, $g$, $h_1 \coloneqq g^x, h_2 \coloneqq g^y$ and
		      \[
			      k = \begin{cases}
				      g^{x \cdot y} & b = 0 \\
				      \tilde{k}     & b = 1
			      \end{cases}
		      \]
		      where $\tilde{k} \from \mathbb{G}$.
		\item $\adv$ outputs a bit $b'$. The output of the game is defined to be $1$ if $b' = b$, and $0$ otherwise.
	\end{enumerate}
\end{definition}

\begin{definition}[Hardness of the DDH problem {\cite[Definition 9.64]{introduction-to-modern-cryptography}}]
	The DDH problem is \emph{$(t, \epsilon)$-hard relative to} $\mathcal{G}$ if for all $\eta$, for any adversary $\adv$ running in time $t(\eta)$ we have
	\begin{align*}
		\advantage{\mathcal{G}}{\eta}{DDH}(\adv) \coloneqq 2 \cdot \left(\pr{\game{\mathcal{G}}{\eta}{DDH}(\adv) = 1} - \frac{1}{2}\right) \le \epsilon(\eta).
	\end{align*}
\end{definition}

In the following definition we will refer to ``key-derivation functions''. This is only meant as a hint to the reader. We do not provide a definition here, as we will always model such a function as a random oracle (see Section~\vref{sec:rom}).

\begin{definition}[DHIES {\cite[Construction 12.19]{introduction-to-modern-cryptography}}]
	Let $\eta$ denote the security parameter. Let $\mathcal{G}$ a group-generation algorithm. Let $\Pi_s$ a private-key encryption scheme where $\Pi_s.\gen(1^\eta)$ samples a key u.a.r.\ from $\{0, 1\}^\eta$. Let $\fhdh = \{\hdh^{(\eta)} \mid \eta \in \N \}$ a family of key-derivation functions where $\hdh^{(\eta)} \colon \{0, 1\}^* \to \{0, 1\}^{\eta}$. We write $\hdh \coloneqq \hdh^{(\eta)}$ when $\eta$ is clear from the context. Define the algorithms $\gen, \enc$ and $\dec$ as follows:
	\begin{itemize}
		\item $\gen$: on input $1^\eta$ run $\mathcal{G}(1^\eta)$ to obtain $(\mathbb{G}, q, g)$. Sample $x \from [q]$ and set $h_1 \coloneqq g^x$. Set $pk \coloneqq \langle \mathbb{G}, q, g, h_1 \rangle$ and $sk \coloneqq \langle \mathbb{G}, q, g, x \rangle$, and output $(pk, sk)$.

		      The message space is the message space of $\Pi_s$.
		\item $\enc$: on input a public key $\langle \mathbb{G}, q, g, h_1 \rangle$ and a message $m$, sample $y \from [q]$, set $h_2 \coloneqq g^y, k \coloneqq \hdh(h_1^y)$\footnote{Where for $h \in \mathbb{G}$, $\hdh(h)$ denotes the output of $\hdh$ with the binary representation of $h$ given as input.}, compute $c' \from \Pi_s.\enc_k(m)$ and output the ciphertext $\langle h_2, c' \rangle$.
		\item $\dec$: on input a private key $\langle \mathbb{G}, q, g, x, \hdh \rangle$ and a ciphertext $\langle h_2, c' \rangle$, compute $k \coloneqq H(h_2^x)$ and output $\Pi_s.\dec_k(c')$. If the ciphertext is not of the right form or $\Pi_s.\dec$ outputs $\bot$, output $\bot$.
	\end{itemize}
	The public-key encryption scheme $\dhies \coloneqq (\gen, \enc, \dec)$ is called the Diffie-Hellman Integrated Encryption Scheme (DHIES).

	When using the DHIES scheme later on, we will set $pk \coloneqq h $ and $sk \coloneqq x$ in $\gen$ for simplicity. In practice $\mathbb{G}, q, g$ and $\hdh$ will be known.
\end{definition}

Under the DDH assumption (i.e. the assumption that the DDH problem is hard relative to $\mathcal{G}$), using DHIES with an EAV secure private-key scheme gives an IND-CPA secure public-key encryption scheme in the ROM, as proven in \cite[Theorem~12.12]{introduction-to-modern-cryptography}.


\section{Our concrete result for TreeKEM}

The following theorem describing the security of TreeKEM, stated informally here, is our main practical result. It bounds the advantage of any adversary creating (at most) $c$ commits and $p$ add or update proposals in a group with at most $u$ users in distinguishing the group key of any uncompromised commit from a random bit string.

\begin{theorem}[Informal] \label{theorem:treekem-security-informal}
	If the DHIES scheme is used in TreeKEM, the private-key encryption scheme in DHIES is $(t, \eeav)$-EAV-secure and the DDH problem is $(t, \eddh)$-hard in the Diffie-Hellman group, then for all $c, p, u$, TreeKEM is $(\tilde{t}, \tilde{\epsilon}, c, p, u)$-secure in the ROM with $\tilde{t} \approx t$ and
	\begin{align*}
		\begin{split}
			\tilde{\epsilon} ={} & 2 \cdot u \cdot (3 \cdot c \cdot \log(u) + p) \cdot \eeav \\
			& + 2 \cdot (3 \cdot c \cdot \log(u) + p) \cdot \eddh \\
			& + \mathrm{negl}(\eta)
		\end{split}
	\end{align*}
	where
	\begin{itemize}
		\item $c$ is the number of commits created
		\item $p$ is the number of add or update proposals created
		\item $u$ is the maximum number of users
	\end{itemize}
\end{theorem}

In Section~\ref{sec:treekem-security} of the appendix we restate the above theorem formally as Theorem~\ref{theorem:treekem-security}. To this end, we also provide formal definitions for the syntax and security of propose and commit CGKA schemes and a high-level description of how to instantiate (the essence of) the TreeKEM protocol with our definitions.

\subsection{Interpreting the result}

In the following, we will go through some concrete examples to see what level of security our proof guarantees for TreeKEM with different parameter choices. We will look at the \texttt{MLS\_128\_DHKEMX25519\_AES128GCM\_SHA256\_Ed25519} cipher suite \cite[Section~17.1]{rfc9420} for 128-bit parameters, which uses Curve25519 as the Diffie-Hellman group and AES with a 128-bit key size for private-key encryption. We will assume that both Curve25519 and AES have a 128-bit security level and we will set $(t, \eeav) = (t, \eddh) = (2^{48}, 2^{-80})$.

For 256-bit parameters, we will look at the \texttt{MLS\_256\_DHKEMP521\_AES256GCM\_SHA512\_P521} cipher suite, which uses curve P-521 and 256-bit AES. We will assume that P-521 and 256-bit AES have a 256-bit security level and set $(t, \eeav) = (t, \eddh) = (2^{128}, 2^{-128})$.

\subsubsection{Large groups with hourly commits and frequent updates}

In this example we consider a large group of about 10'000 users, existing for 5 years and making one commit every hour. Then $u \le 2^{14}$ and $c \le 2^{16}$. We also assume that a significant fraction of the users will want to update with every commit. Then, assuming that add proposals are relatively rare, we can bound $p \le c \cdot u = 2^{30}$. This implies $3 \cdot c \cdot \log(u) + p \le 2^{31}$, dominated by $p$.

Then with 128-bit parameters we get
\[
	\et \le 2^{46} \cdot \eeav + 2^{32} \cdot \eddh + \mathrm{negl} \le \frac{1}{2^{33}}
\]
with the $\eeav$ term dominating the result. This only gives a security level of $\tilde{t}/\et \ge 2^{81}$. Since private-key encryption is relatively cheap, using 256-bit AES would have a small impact on the performance and would increase the security level to 95 bits (with the $\eddh$ term now dominating). Finally, using full 256-bit parameters yields 209 bits of security for TreeKEM.

The previous best result in \cite[Theorem 3]{ttkem} proved the bound
\[
	\et \le 2 \cdot (3 \cdot c \cdot \log(u) + p)^2 \cdot \epsilon + \mathrm{negl}
\]
where $\epsilon$ is the IND-CPA security of the underlying public-key encryption scheme. If we assume that DHIES has an $x$-bit security level as a public-key encryption scheme with $x$-bit parameters, the result implies 64 bits of security with 128-bit parameters (with no change when using 256-bit AES) and 192 bits with 256-bit parameters.

\subsubsection{Few updates}

In this example we use the same number of users and commits, but assume that the number of proposals is small such that $p \le c \cdot \log(u)$. In this case, we have $3 \cdot c \cdot \log(u) + p \le 2^{22}$. Then our result guarantees 90 bits of security with 128-bit parameters, 104 bits with 256-bit AES and 218 bits with 256-bit parameters.

In contrast, the bound in \cite{ttkem} implies 82 bits with 128-bit parameters and 210 bits with 256-bit parameters.

\subsubsection{Very large groups with one commit every minute and frequent updates} In this example we consider more extreme values for $c$ and $u$ to highlight the gap between our result and the one in \cite{ttkem}. We assume a group of about 1 million users, existing for 50 years and making one commit every minute. Furthermore, will again use $p \le c \cdot u$. This means that $u \le 2^{20}$, $c \le 2^{25}$ and $3 \cdot c \cdot \log(u) + p \le 2^{46}$.

These values imply a 61 bits of security with 128-bit parameters, 81 bits with 256-bit AES and 189 bits with 256-bit parameters using our result. The result in \cite{ttkem} implies 34 bits of security with 128-bit parameters and 162 bits with 256-bit parameters.


\section{Tighter GSD security} \label{sec:tighter-gsd-security}

In this section, we present the main theoretical result behind Theorem~\ref{theorem:treekem-security-informal} in Theorem~\ref{theorem:sdgsd-security}. To this end, we first define a modified version of the public-key GSD game from \cite{ttkem} and then proceed to prove Theorem~\ref{theorem:sdgsd-security} in detail.

\subsection{Seeded GSD with Dependencies} \label{sec:sd-gsd-game}

The GSD game defined here is inspired by the definition of the public-key GSD game (Definition 7) and the proof of Theorem 3 in \cite{ttkem}. We have already motivated the differences in Section~\ref{sec:gsd-intro}. We will call our adapted game \emph{Seeded GSD with Dependencies} (SD-GSD). A very similar definition appears in \cite{modular-group-messaging}, providing essentially the same abstraction over TreeKEM and also allowing for an adversary to provide the randomness used for encryption and key generation. However, our definition has one notable difference: when asking to be challenged on a node with seed $s$, the adversary must distinguish $\hdep(s)$ from random as opposed to $s$. This stays true to how the group key is computed in TreeKEM and has the significant advantage of greatly simplifying our proof. On the other hand, the security implied by our definition is weaker (at least in the ROM), as it only guarantees that an adversary cannot compute the seed of the challenge node, whereas the other definitions guarantee that this seed cannot be distinguished from random.

\begin{definition}[The SD-GSD game] \label{def:sd-gsd-game}
	Let $\Pi = (\gen, \enc, \dec)$ a public-key encryption scheme, where $\gen(1^\eta)$ uses $\rho(\eta)$ random bits and $\{0, 1\}^{\rho(\eta)}$ is a subset of the message space.
	Let $\fhgen = \{ \hgen^{(\eta)} \mid \eta \in \N\}, \fhdep = \{ \hdh^{(\eta)} \mid \eta \in \N\}$ families of functions with $\hgen^{(\eta)}, \hdep^{(\eta)} \colon \{0, 1\}^{\rho(\eta)} \to \{0, 1\}^{\rho(\eta)}$. We will write $\hgen \coloneqq \hgen^{(\eta)}, \hdep \coloneqq \hdep^{(\eta)}$ and $\rho \coloneqq \rho(\eta)$ if $\eta$ is clear from the context.
	Define the game $\game{(\Pi, \fhgen, \fhdep)}{\eta}{SD\text{-}GSD}(\adv)$ for an adversary $\adv$:
	\begin{enumerate}[1.]
		\item \label{def:sd-gsd-game-step-init} The adversary $\adv$ outputs $n \in \N$ and a list of dependencies $D = \{(a_{i}, b_{i})\}_{i=1}^m \subseteq [n]^2$. For each $v \in [n]$:
		      \begin{enumerate}[(i)]
			      \item \begin{itemize}
				            \item \textbf{Case $v = b_i$ for some $i$ ($v$ is the target of some dependency):} set $s_v = \hdep(s_{a_i})$.
				            \item \textbf{Otherwise:} sample $s_v \from \{0, 1\}^\rho$.
			            \end{itemize}
			            We call $s_v$ the \emph{seed} of the node $v$ and a tuple $(a, b) \in D$ a \emph{seed dependency}.
			      \item Compute $(pk_v, sk_v) = \gen(1^\eta, \hgen(s_v))$.
		      \end{enumerate}
		      Set $\mathcal{C} = E = \varnothing$. We call the directed graph $([n], E)$ a \emph{GSD graph} of \emph{size} $n$.
		\item $\adv$ may adaptively make the following queries:
		      \begin{itemize}
			      \item $\mathrm{reveal}(v)$ for $v \in [n]$: $\adv$ is given $pk_v$.
			      \item $\mathrm{encrypt}(u, v)$ for $u, v \in [n], u \neq v, (u, v) \notin E$: $(u, v)$ is added to $E$ and $\adv$ is given $c \from \enc_{pk_u}(s_v)$.
			      \item $\mathrm{corrupt}(v)$ for $v \in [n], v \notin \mathcal{C}$: $\adv$ is given $s_v$ and $v$ is added to $\mathcal{C}$. We call such a node $v \in \mathcal{C}$ \emph{corrupted}. All nodes not reachable from any corrupted node in the graph $([n], E \cup D)$ are \emph{safe} (while all other nodes are \emph{unsafe}) and we call their seeds \emph{hidden} (even if an unsafe node happens to have the same seed).
		      \end{itemize}
		\item \label{def:sd-gsd-game-step-challenge} $\adv$ outputs a node $v \in [n]$. We call $v$ the \emph{challenge node}. A bit $b \from \{0, 1\}$ is sampled and $\adv$ is given
		      \[
			      r = \begin{cases}
				      \hdep(s_v) & b = 0 \\
				      s          & b = 1
			      \end{cases},
		      \]
		      where $s \from \{0, 1\}^\rho$.\footnote{Note that (in general) $r$ is not a hidden seed, as (with overwhelming probability) it is not the seed of any node.} $\adv$ may continue to do queries as before.
		\item \label{def:sd-gsd-game-step-output} $\adv$ outputs a bit $b'$. The output of the game is defined to be $1$ if $b' = b$, and $0$ otherwise.
	\end{enumerate}

	We require an adversary playing the above game to adhere to the following:
	\begin{itemize}
		\item The challenge node is safe
		\item The challenge node is not the source of a seed dependency\footnote{Otherwise the adversary could learn the value of $\hdep$ on the seed of the challenge node by creating a seed dependency with the challenge node as the source and corrupting the target node.}
		\item Every node is the source and target of at most one seed dependency\footnote{If a node were the source node of multiple seed dependencies, then corrupting one target node would reveal the seeds of all target nodes. Additionally, the computation of seeds is not well-defined if a node is the target of multiple dependencies.}
		\item The graph $([n], E \cup D)$ is acyclic and without self-loops
	\end{itemize}
\end{definition}

It is interesting to note that previous definitions of the GSD game also included the following restrictions to the adversary:
\begin{itemize}
	\item The challenge node always remains a sink in $([n], E)$
	\item $\operatorname{reveal}$ is never queried on the challenge node
\end{itemize}
Our proof in the ROM does not require these restrictions. If $\hdep$ is modelled as a random oracle, then an encryption edge outgoing from the challenge node, or knowing its public key gives no advantage to $\adv$, as by the assumption of $\hdep$ being a random oracle the only way to learn information about $\hdep(s)$ is by querying $s$.

\begin{definition}[SD-GSD security]
	Let $\Pi, \fhgen$ and $\fhdep$ as in Definition~\ref{def:sd-gsd-game} above and let $t, \epsilon, N, \delta$ functions in $\eta$.
	The triple $(\Pi, \hgen, \hdep)$ is \emph{$(t, \epsilon, N, \delta)$-SD-GSD-secure} if for all $\eta$, for any adversary $\adv$ constructing a GSD graph of size at most $N(\eta)$ and indegree at most $\delta(\eta)$ and running in time $t(\eta)$ we have
	\begin{align*}
		\advantage{(\Pi, \fhgen, \fhdep)}{\eta}{SD\text{-}GSD}(\adv) \coloneqq 2 \cdot \left(\pr{\game{(\Pi, \fhgen, \fhdep)}{\eta}{SD\text{-}GSD}(\adv) = 1} - \frac{1}{2}\right) \le \epsilon(\eta).
	\end{align*}
\end{definition}

Since in this work we are interested in SD-GSD security for the case where $\hgen$ and $\hdep$ are modelled as random oracles and our focus is on the encryption scheme being used, we introduce the following definition for convenience.

\begin{definition}[SD-GSD security in the ROM]
	A public-key encryption scheme $\Pi$ is \emph{$(t, \epsilon, N, \delta)$-SD-GSD-secure in the ROM} if the triple $(\Pi, \fhgen, \fhdep)$ is $(t, \epsilon, N, \delta)$-SD-GSD-secure when $\hgen$ and $\hdep$ are modelled as random oracles. For security parameter $\eta$ and an adversary $\adv$, we write $\game{\Pi}{\eta}{SD\text{-}GSD}(\adv)$ to denote the game where $\hgen$ and $\hdep$ are modelled as random oracles and $\advantage{\Pi}{\eta}{SD\text{-}GSD}(\adv)$ for $\adv$'s advantage in this game.
\end{definition}

\subsection{Proving SD-GSD security for DHIES in the ROM}

\begin{theorem} \label{theorem:sdgsd-security}
	Let $\dhies$ denote the DHIES scheme instantiated with a group-generation algorithm $\mathcal{G}$ and a private-key encryption scheme $\Pi_s$. If $\Pi_s$ is $(t, \eeav)$-EAV-secure, the DDH problem is $(t, \eddh)$-hard relative to $\mathcal{G}$ and the function $\hdh$ in $\dhies$ is modelled as a random oracle, then for any $\delta, N$ with $\delta \le N$, $\dhies$ is $(\tilde{t}, \tilde{\epsilon}, N, \delta)$-SD-GSD-secure in the ROM with\footnote{Note that in the following equality we have omitted writing the argument $\eta$ to the various functions and are implying that the equality holds for all $\eta$.}
	\[
		\tilde{\epsilon} = 2 \cdot \delta \cdot N \cdot \eeav + 2 \cdot N \cdot \eddh + \frac{2 \cdot \mdh \cdot N^2}{q} + \frac{\ms \cdot N}{2^{\rho - 1}},
	\]
	where $\ms$ is an upper bound on the number of queries made to either $\hgen$ or $\hdep$, $\mdh$ is an upper bound on the number of queries made to $\hdh$, $q$ is a lower bound on the size of the group output by $\mathcal{G}$ and $\rho$ is the number of random bits sampled by $\dhies.\gen$, and with $\tilde{t} \approx t$.\footnote{We provide a more precise expression of the runtime in Section~\ref{sec:sdgsd-security-runtime} of the appendix.}
\end{theorem}

In contrast, the result in \cite{ttkem} achieves a security loss in $\mathcal{O}(N^2)$ and reduces to the IND-CPA security of the public-key encryption scheme.

For ease of exposition, we will assume that $\mathcal{G}(1^\eta)$ is deterministic, as is the case in practice. We will therefore set $pk \coloneqq h_1, sk \coloneqq x$ in $\dhies.\gen$, as $\mathbb{G}, q, g$ are implied by $\eta$. The results nevertheless hold also for the general case.

\paragraph{Intuition}
Consider an arbitrary SD-GSD adversary $\adv$. For an execution of $\game{\dhies}{\eta}{SD\text{-}GSD}(\adv)$ we say ``\emph{$\adv$ wins}" to denote the event $\game{\dhies}{\eta}{SD\text{-}GSD}(\adv) = 1$.
As usual with random oracles we proceed by a case distinction on whether they were queried on some interesting value. Accordingly, let $Q_{x}$ denote the event that $\adv$ queries $H_{x}$ on a hidden seed for $x \in \{\mathrm{gen}, \mathrm{dep}\}$. Then we can write
\begin{align} \label{eq:theorem-sd-gsd-security-win-cases}
	\begin{split}
		\pr{\wins} & = \pr{\wins \land \qdep} + \pr{\wins \land \overline{\qdep} \,} \\
		& \le \pr{\wins \land \qdep} + \pr{\wins \mid \overline{\qdep} \,} \\
		& \stackrel{\mathclap{(\dagger)}}{=}  \pr{\wins \land \qdep} + \frac{1}{2}         \\
		& \le \pr{\qdep} + \frac{1}{2} \\
		& \le \pr{\qs} + \frac{1}{2},
	\end{split}
\end{align}
where $\qs \coloneqq \qgen \cup \qdep$ ($\mathrm{s}$ for \emph{seed}). Step $(\dagger)$ intuitively holds because without having queried $\hdep$ for any hidden seed, in particular the seed $s_v$ of the challenge node $v$, $\hdep(s_v)$ is a uniformly random value from $\adv$'s perspective. Therefore, it can do no better than guessing to distinguish $\hdep(s_v)$ from $s \from \{0, 1\}^\rho$. \todo{correct this note based on fix to the argument.}

The heart of the proof is to bound $\pr{\qs}$. When the adversary first triggers $\qs$ by querying the seed of some safe node $w$, (with overwhelming probability $w$ will be the only node with this seed and) it can only have learned the seed through encryptions
$c_1 \from \dhies.\enc_{pk_{u_1}}(s_w), \ldots, c_d \from \dhies.\enc_{pk_{u_d}}(s_w)$
where $(u_1, w), \ldots, (u_d, w)$ are edges in the GSD graph (obtained through corresponding queries $\operatorname{encrypt}(u_1, w), \ldots, \operatorname{encrypt}(u_d, w)$). The only other potential source of information about $s_w$ would be a seed dependency $(p, w)$, but this tells $\adv$ nothing: Since $w$ is safe, $p$ would also be safe and $\hdep(s_p)$ cannot have been queried due to the assumption that $w$ was the first node to trigger $\qs$.
Without having queried $\hdep(s_p)$, by virtue of $\hdep$ being a random oracle $s_w$ has the same distribution as a seed without a dependency from $\adv$'s perspective (uniformly random). See Figure~\ref{fig:gsd-qs-triggered} for an illustration of node $w$ in the GSD graph.

\begin{figure}
	\begin{center}
		\includegraphics{figures/gsd-qs-triggered}
	\end{center}
	\caption{Illustration of the GSD graph when $\qs$ is triggered at a node $w$. The dashed edge represents a seed dependency $(p, w)$ and the remaining edges represent encryption queries $c_i \from \operatorname{encrypt}(u_i, w)$.}
	\label{fig:gsd-qs-triggered}
\end{figure}

The proof in \cite{ttkem} simply argued that this is not too likely if these encryptions were made with an IND-CPA secure scheme. In the context of the DHIES scheme we can say more about these encryptions and achieve a better reduction loss.
Let $(\mathbb{G}, q, g) = \mathcal{G}(1^\eta)$.
Let $x_i = \log_g(pk_{u_i})$. Each encryption $c_i$ is a tuple of the form $\langle g^{y_i}, \Pi_s.\enc_{k_i}(s_w) \rangle$ where $y_i \from [q]$ and $k_i = \hdh\left(g^{x_i \cdot y_i}\right)$. Now we can again do a case distinction on whether $\hdh$ was queried for (the encoding of) some group element $g^{x_j \cdot y_j}$ or not:
\begin{enumerate}[(i)]
	\item \label{qs-triggered-case-1} If such a query was made, then $\adv$ solved the Diffie-Hellman challenge $(g^{x_j}, g^{y_j})$. (Remember that we assumed that $w$ is the first node for which $\qs$ is triggered and as before if $w$ is safe, then so are the nodes $u_i$. Thus the adversary has not learned the exponent $x_i$ through querying $\hgen(s_{u_i})$ for any $i$.)
	\item \label{qs-triggered-case-2} If no such query was made, then from $\adv$'s perspective all the $k_i$ are independent, uniformly random keys and it still was able to learn $s_w$ from the EAV secure encryptions $\Pi_s.\enc_{k_1}(s_w), \ldots, \Pi_s.\enc_{k_d}(s_w)$.
\end{enumerate}
We can bound the probability of either of these events occurring using the hardness of the DDH problem relative to $\mathcal{G}$ and EAV security of $\Pi_s$, respectively.

To this end, we call a group element $h \in \mathbb{G}$ a \emph{hidden Diffie-Hellman key} if $h = pk_u^{y_{u, v}}$, where $(u, v)$ is an edge in the GSD graph, $u$ is safe and $y_{u, v}$ is the exponent chosen in the DHIES encryption of $s_v$ (i.e. $\adv$ was given a ciphertext of the form $\langle g^{y_{u, v}}, \ldots\rangle$ when it queried $\operatorname{encrypt}(u, v)$). Now analogously to above let $\qdh$ the event that $\adv$ queries $\hdh$ on a hidden Diffie-Hellman key, and let $\fdh$ the event that $\adv$ triggers $\qdh$ when $\qs$ has not (yet) been triggered.
Then we can split the event $\qs$ into two cases as motivated above:
\begin{align*}
	\pr{\qs} & = \pr{\qs \land \fdh} + \pr{\qs \land \overline{\fdh}\,}.
\end{align*}
We bound $\pr{\qs \land \fdh}$ and $\pr{\qs \land \overline{\fdh} \,}$ in Lemma~\ref{lemma:dh-reduction} and Lemma~\ref{lemma:eav-reduction}, respectively.\footnote{To be precise, the event $\qs \land \fdh$ is a superset of case \ref{qs-triggered-case-1} above. However, the argument applied in Lemma~\ref{lemma:dh-reduction} gives the same bound for either event and this more general event has the advantage of being simpler.}
Overall this gives us a bound on the advantage of $\adv$ using \eqref{eq:theorem-sd-gsd-security-win-cases}.

\begin{proof}[of Theorem~\ref{theorem:sdgsd-security}]
	Let $\delta, N$ functions in $\eta$ (mapping to $\N$) with $\delta \le N$.
	Let $\eta$ arbitrary and let $\adv$ an arbitrary SD-GSD adversary constructing a GSD graph of size at most $N(\eta)$ and indegree at most $\delta(\eta)$, making at most $\ms(\eta)$ queries to $\hgen$ or $\hdep$ and at most $\mdh(\eta)$ queries to $\hdh$, each of length at most $\gamma(\eta)$, and running in time $\tilde{t}(\eta)$. We will use the events defined above.

	We first justify step $(\dagger)$ in \eqref{eq:theorem-sd-gsd-security-win-cases}. Note that by the rules imposed on the adversary in the SD-GSD game, the challenge node $v$ is safe and its seed is thus indeed hidden. If $\qdep$ does not hold, then $\adv$ has not queried $\hdep$ for $s_v$ and, by virtue of $\hdep$ being a random oracle, $\hdep(s_v)$ is a uniformly distributed value in $\{0, 1\}^\rho$ from $\adv$'s perspective. The value $s$ follows the same distribution. Thus, $\adv$ behaves the same when given either $r = s$ or $r = \hdep(s_v)$ and
	\begin{align} \label{eq:theorem-sd-gsd-security-not-qdep-behavior}
		\begin{split}
			\pr{1 \from \adv \mid \overline{\qdep}, b = 1} & = \pr{1 \from \adv \mid \overline{\qdep}, r = s}          \\
			& = \pr{1 \from \adv \mid \overline{\qdep}, r = \hdep(s_v)} \\
			& = \pr{1 \from \adv \mid \overline{\qdep}, b = 0}.
		\end{split}
	\end{align}
	Therefore
	\begin{align*}
		\pr{\wins \mid \overline{\qdep} \,} & = \begin{aligned}[t]
			                                         & \pr{1 \from \adv \mid \overline{\qdep}, b = 1} \cdot \frac{1}{2}   \\
			                                         & + \pr{0 \from \adv \mid \overline{\qdep}, b = 0} \cdot \frac{1}{2}
		                                        \end{aligned}                                                                              \\
		                                    & \stackrel{\mathclap{\eqref{eq:theorem-sd-gsd-security-not-qdep-behavior}}}{=} \begin{aligned}[t]
			                                                                                                                     & \pr{1 \from \adv \mid \overline{\qdep}, b = 0} \cdot \frac{1}{2}   \\
			                                                                                                                     & + \pr{0 \from \adv \mid \overline{\qdep}, b = 0} \cdot \frac{1}{2}
		                                                                                                                    \end{aligned} \\
		                                    & = \frac{1}{2}.
	\end{align*}

	By Lemma~\ref{lemma:dh-reduction} we have\footnote{Note that we are again omitting the argument $\eta$ from the functions on the right-hand side ($N, \eddh$ and $\mdh$ in this case).}
	\[
		\pr{\qs \land \fdh} \le N \cdot \eddh + \frac{\mdh \cdot N^2}{q}.
	\]
	and by Lemma~\ref{lemma:eav-reduction} we have
	\[
		\pr{\qs \land \overline{\fdh}\,} \le \delta \cdot N \cdot \eeav + \frac{\ms \cdot N}{2^\rho},
	\]
	so we know that
	\begin{equation} \label{eq:theorem-sd-gsd-security-qs-bound}
		\pr{\qs} \le N \cdot \eddh + \delta \cdot N \cdot \eeav + \frac{\mdh \cdot N^2}{q} + \frac{\ms \cdot N}{2^\rho} = \tilde{\epsilon}(\eta) / 2.
	\end{equation}
	Then
	\begin{align*}
		\advantage{\Pi}{\eta}{SD\text{-}GSD}(\adv) & = 2 \cdot \left(\pr{\wins} - \frac{1}{2}\right)                                                   \\
		                                           & \stackrel{\mathclap{\eqref{eq:theorem-sd-gsd-security-win-cases}}}{\le} \; 2 \cdot \pr{\qs}       \\
		                                           & \stackrel{\mathclap{\eqref{eq:theorem-sd-gsd-security-qs-bound}}}{\le} \; \tilde{\epsilon}(\eta).
	\end{align*}
\end{proof}

\subsubsection{Security against multiple challenges} \label{sec:sd-gsd-multi-challenge-security}
As noted already in the introduction, it is straightforward to prove the same result for an SD-GSD game with multiple challenge queries. In this SD-GSD game, the adversary would have access to an additional \emph{challenge} oracle to perform a challenge on any node as in step \ref{def:sd-gsd-game-step-challenge} of the original game, with the bit $b$ sampled once at the beginning of the game, and the seed $s$ sampled fresh for every query. Of course, the adversary must ensure that the restrictions that apply to the challenge node in the original game apply to all challenge nodes in this new game.

To prove security in this multi-challenge SD-GSD game, the same case distinction can be performed as in \eqref{eq:theorem-sd-gsd-security-win-cases} and, concerning step $(\dagger)$, it can be shown through a sequence of statistically indistinguishable hybrids that the adversary has no advantage if they did not trigger $\qdep$.
\todo{adapt based on how the argument changes}

\subsubsection{Reducing to EAV security}

Recall case \ref{qs-triggered-case-2} in the high-level discussion of Theorem~\ref{theorem:sdgsd-security}: the adversary $\adv$ was able to learn the seed $s_w$ given the EAV secure encryptions $\Pi_s.\enc_{k_1}(s_w), \ldots, \Pi_s.\enc_{k_d}(s_w)$. We can see $\adv$ as an adversary in a security game where $\adv$ is given $d$ EAV secure encryptions $c_1 \from \Pi_s.\enc_{k_1}(m), \ldots, c_d \from \Pi_s.\enc_{k_d}(m)$ of a message $m$ with $k_i \from \Pi_s.\gen(1^\eta)$ and must compute $m$. If we can prove that beating such a game is hard, then we can bound the probability of $\adv$ actually learning $s_w$ in this way.

This is exactly how we proceed in this section. Instead of asking the adversary to compute an encrypted message $m$, we turn to a more familiar decisional formulation as in the IND-CPA game (where the adversary may choose a pair $m_0, m_1$ and must distinguish whether the $d$ ciphertexts encrypt $m_0$ or $m_1$). We call this security notion \emph{EAV security under multiple (M) independent (I) encryptions of a single (S) pair of messages} (MIS-EAV).

\begin{definition}[The MIS-EAV game]
	Let $\kappa$ denote the security parameter and let $\Pi$ a private-key encryption scheme. Define the game $\game{\Pi}{\kappa}{MIS\text{-}EAV}(\adv)$ for an adversary $\adv$:
	\begin{enumerate}[1.]
		\item The adversary $\adv$ outputs $q \in \N$ and a pair of messages $m_0, m_1$ of the same length. We refer to $q$ as the number of \emph{queries} made by $\adv$.
		\item A bit $b \from \{0, 1\}$ is sampled. For each $i \in [q]$, $\adv$ is given an encryption $c_i \from \Pi.\enc_{k_i}(m_b)$ where $k_i \from \Pi.\gen(1^\kappa)$ is generated independently of the other keys.
		\item $\adv$ outputs a bit $b'$. The output of the game is defined to be $1$ if $b' = b$, and $0$ otherwise.
	\end{enumerate}
\end{definition}

\begin{definition}[MIS-EAV security]
	A private-key encryption scheme $\Pi$ is \emph{$(t, \epsilon, q)$-MIS-EAV-secure} if for all $\kappa$, for any adversary $\adv$ making at most $q(\kappa)$ queries and running in time $t(\kappa)$ we have
	\begin{align*}
		\advantage{\Pi}{\kappa}{MIS\text{-}EAV}(\adv) \coloneqq 2 \cdot \left(\pr{\game{\Pi}{\kappa}{MIS-EAV}(\adv) = 1} - \frac{1}{2}\right) \le \epsilon(\kappa).
	\end{align*}
\end{definition}

Similar to how IND-CPA security for a single encryption query implies IND-CPA security for $q$ queries with a security loss of $q$ by a standard hybrid argument, one can show that EAV security implies MIS-EAV security with the same loss. To see why, recall the hybrid argument for IND-CPA security (as discussed in e.g. \cite[Theorem 12.6]{introduction-to-modern-cryptography}): We define the sequence of hybrid games $G_0, \ldots, G_q$ where in the game $G_i$ the first $i$ encryption queries encrypt the second message and the remaining $q - i$ queries encrypt the first message. Then given an IND-CPA adversary $\adv$ for multiple encryptions, an IND-CPA adversary $\adv'$ is constructed to bound
\[
	\abs*{\pr{\adv \text{ outputs } 0 \text{ in game } G_{i - 1}} - \pr{\adv \text{ outputs } 0 \text{ in game } G_{i}}}
\]
for arbitrary $i$.
The adversary $\adv'$ simulates $G_{i - 1}$ or $G_{i}$ to $\adv$ depending on whether the ciphertext received from the (single-query) IND-CPA challenger, which gets passed on as the response to the $i$-th query, encrypts the first or the second message from the $i$-th pair of messages. $\adv'$ then uses the encryption oracle to pass on the right encryptions to $\adv$ for all other queries. Now notice that if we wanted to simulate to an MIS-EAV adversary we would not need access to an encryption oracle since for the MIS-EAV security game all the other encryptions can easily be generated by $\adv'$ sampling the new keys itself.

The argument would of course also work without restricting the adversary to a single pair of messages. However, we will make use of this restriction to provide a tighter reduction for a certain class of schemes in the appendix.

\begin{lemma} \label{lemma:mis-eav-from-eav}
	Let $\Pi$ a private-key encryption scheme with finite message space. Let $t_{\gen}, t_{\enc}$ functions in $\kappa$ that upper bound the runtime of $\Pi.\gen$ and $\Pi.\enc$, respectively. If $\Pi$ is $(t, \epsilon)$-EAV-secure, then for any function $q$, $\Pi$ is $(\tilde{t}, q \cdot \epsilon, q)$-MIS-EAV-secure with $\tilde{t} = t - \mathcal{O}(q \cdot (t_\gen + t_\enc))$.
\end{lemma}

The details of the proof can be found in Section~\ref{sec:mis-eav-from-eav-proof} of the appendix.

\begin{lemma} \label{lemma:eav-reduction}
	Recall the assumptions, variables and events from the statement and proof of Theorem~\ref{theorem:sdgsd-security}. In particular, assume that $\Pi_s$ is $(t, \eeav)$-EAV-secure. Let $\eta$ arbitrary and let $\adv$ an SD-GSD adversary constructing a GSD graph of size at most $N(\eta)$ and indegree at most $\delta(\eta)$, making at most $\ms(\eta)$ queries to $\hgen$ or $\hdep$ and at most $\mdh(\eta)$ queries to $\hdh$, and running in time $\tilde{t}(\eta)$. Then
	\[
		\pr{\qs \land \overline{\fdh}\,} \le \delta \cdot N \cdot \eeav + \frac{\ms \cdot N}{2^\rho}.
	\]
\end{lemma}

\paragraph{Intuition} By Lemma~\ref{lemma:mis-eav-from-eav} we know that $\Pi_s$ is MIS-EAV secure. Continuing the high-level argument before the proof of Theorem~\ref{theorem:sdgsd-security}, consider the first moment that $\adv$ triggers $\qs \land \overline{\fdh}$ by querying the seed of some safe node $w$.  As intended, it follows from the definition of the event $\fdh$ that from $\adv$'s perspective all DHIES ciphertexts it got from queries $\operatorname{encrypt}(u, w)$ for any $u$ contain encryptions of $s_w$ under independent, uniformly random keys using $\Pi_s$. Moreover, as already argued once, $\adv$ has learned nothing from a potential seed dependency $(p, w)$, so these encryptions are everything $\adv$ had at its proposal to learn $s_w$.

We can use $\adv$'s ability to compute the seed $s_w$ of a safe node $w$ from encryptions of $s_w$ to construct an MIS-EAV adversary: We first guess a node $w$ whose seed $\adv$ may query first. Next, we give the MIS-EAV challenger $s_w$ and some other independent seed $s$. We simulate the SD-GSD game to $\adv$ and embed the encryptions from the MIS-EAV challenger when answering queries of the form $\operatorname{encrypt}(u, w)$ for any $u$. Now consider the behavior of $\adv$ depending on which seed the challenger chooses to encrypt:
\begin{itemize}
	\item If the challenger chooses to encrypt $s_w$, then $\adv$ will trigger the event $\qs \land \overline{\fdh}$ with the same probability as before. We can detect whether $\qs \land \overline{\fdh}$ gets triggered since all seeds in the simulation are known. If $\qs \land \overline{\fdh}$ occurs and we guessed $w$ correctly, the event will be triggered at $w$ and $\adv$ will query $s_w$, telling us that the challenger encrypted $s_w$.
	\item If the challenger chooses to encrypt $s$, then $\adv$ receives no information about $s_w$ and has a negligible probability of querying it.
\end{itemize}
Thus, the advantage of the MIS-EAV adversary is about $\pr{\qs \land \overline{\fdh}\,}/N$, where the factor $1/N$ arises from guessing $w$, and using that $\Pi_s$ is MIS-EAV secure we can bound this probability. Since we are only interested in checking whether the event was triggered for $w$, the adversary can abort when this is no longer possible ($w$ is corrupted, some other hidden seed is queried, etc.). The details of the proof can be found in Section~\ref{sec:eav-reduction-proof} of the appendix.

For a certain class of schemes, we can improve on Lemma~\ref{lemma:mis-eav-from-eav} and achieve a tight reduction. This allows us to get rid of the factor $\delta$ in Lemma~\ref{lemma:eav-reduction}. However, we do not use this in our main result and refer the interested reader to Section~\ref{sec:tighter-mis-eav-security} of the appendix.

\subsubsection{Reducing to the DDH problem}

\begin{lemma} \label{lemma:dh-reduction}
	Recall the assumptions, variables and events from the statement and proof of Theorem~\ref{theorem:sdgsd-security}. In particular, assume that the DDH problem is $(t, \eddh)$-hard relative to $\mathcal{G}$. Let $\eta$ arbitrary and let $\adv$ an SD-GSD adversary constructing a GSD graph of size at most $N(\eta)$ and indegree at most $\delta(\eta)$, making at most $\ms(\eta)$ queries to $\hgen$ or $\hdep$ and at most $\mdh(\eta)$ queries to $\hdh$, and running in time $\tilde{t}(\eta)$. Then
	\[
		\pr{\qs \land \fdh} \le N \cdot \eddh + \frac{\mdh \cdot N^2}{q}.
	\]
\end{lemma}

\paragraph{Intuition} We will bound the simpler event $\fdh$. This event tells us that there is some safe node $a$ in the GSD graph with encryption edges to nodes $u_1, \ldots, u_d$, where the query $\operatorname{encrypt}(a, u_i)$ returned the ciphertext $\langle g^{y_i}, \Pi_s.\enc_{k_i}(s_{u_i}) \rangle$ with $k_i = \hdh(g^{sk_{a} \cdot y_i})$, such that $g^{sk_a \cdot y_j}$ was the first hidden Diffie-Hellman key queried by $\adv$ for some $j$.
Moreover, at the time $g^{sk_a \cdot y_j}$ was queried, no hidden seed had yet been queried by $\adv$, implying that $\adv$ had not queried $\hgen(s_a)$ and thus had no information about $sk_a$ besides $pk_a$ (recall that $(pk_a, sk_a) = \dhies.\gen(1^\eta, \hgen(s_a))$).

It is interesting to note that our approach does not require that $\adv$ has not queried $\hdep$ for a hidden seed (i.e. that $\qdep$ was not triggered) as is implied by the event $\fdh$, because knowing $\hgen(s_a)$ is the only way to learn about $sk_a$. Regardless, we still want to have our definition of $\fdh$ include this information, as the bound on $\pr{\qs \land \overline{\fdh} \,}$ in Lemma~\ref{lemma:eav-reduction} relies on the fact that in the event of $\qs \land \overline{\fdh}$ happening,  $\qdh$ was not yet triggered when the event $\qs$ was triggered, i.e. when either the event $\qgen$ \emph{or} the event $\qdep$ was triggered.

The intuition is clear that this means that $\adv$ solved the Diffie-Hellman challenge $(g^{sk_a}, g^{y_j})$. What is not immediately clear is how to embed a \emph{given} Diffie-Hellman challenge $(g^x, g^y)$ from an instance of the DDH game and use $\adv$ to tell whether the key $k$ chosen by the challenger is the real key $g^{x \cdot y}$ or a uniformly random group element.
An intuitive strategy would be to embed the challenge by setting $pk_a = g^x$ and $g^{y_j} = g^y$, which involves guessing $u_j$, and simply checking whether for any of the queries $q_i$ to $\hdh$ by $\adv$, such that $q_i$ encodes a group element in $\mathbb{G}$, it holds that $q_i = k$. Now:
\begin{itemize}
	\item If $k = g^{x \cdot y}$, $\adv$ triggers $\fdh$ and we guessed $a$ and $u_j$ correctly, then indeed as described above $q_i = g^{sk_a \cdot y_j} = g^{x \cdot y} = k$ will hold for some $i$.
	\item If $k$ is a random group element, then $\adv$ has negligible probability of querying $k$, as no information about $k$ is ever leaked to $\adv$.
\end{itemize}
If we make sure not to change $\adv$'s view of the game in the case $k = g^{x \cdot y}$ in this process, we can achieve an advantage of about $\pr{\fdh} / N^2$, where one factor $1/N$ arises from guessing $a$ and another from guessing $u_j$. Unfortunately, this would yield no improvement over the result from \cite{ttkem}.

To avoid this issue, we can use the random self-reducibility of the DDH problem and avoid guessing $u_j$. Instead of embedding $g^y$ into a single encryption edge, we embed it into all $d$ encryption edges. To get a uniformly random exponent from $y$ we set $y_j = y + r_j \mod q$ with $r_j \from [q]$. Given $g^{x \cdot y_j}$, we can easily compute $g^{x \cdot y}$:
\[
	g^{x \cdot y_j} = g^{x \cdot (y + r_j)} = g^{x \cdot y}	\cdot g^{x \cdot r_j} \iff g^{x \cdot y} = g^{x \cdot y_j} \cdot \underbrace{((g^x)^{r_j})^{-1}}_{\eqqcolon \, R_j}.
\]
Now to determine whether $k$ is the real Diffie-Hellman key, we check whether $q_i \cdot R_j = k$ for some $i, j$. This yields an advantage of about $\pr{\fdh} / N$ (and a slightly larger runtime). The details of the proof can be found in Section~\ref{sec:dh-reduction-proof} of the appendix.


\newpage

\bibliographystyle{splncs04}
\bibliography{refs}

\newpage

\title{Supplementary material}
\author{}
\institute{}

\maketitle

\appendix

\section{Basic cryptographic definitions} \label{sec:preliminaries-appendix}

\subsection{Notation}

We will use the following notation throughout:
\begin{itemize}
	\item We write $x \from S$ to say that $x$ is sampled u.a.r.\ from the finite set $S$
	\item For $n \in \mathbb{N} \setminus \{0\}$, $[n] = \{1, \ldots, n\}$, and for $a, b \in \N$ s.t.\ $a \le b$, $[a, b] = \{a, a + 1, \ldots, b\}$
	\item If $\mathbb{G}$ is a cyclic group of order $q$ and $g$ a generator, then
	      \begin{itemize}
		      \item We write the group operation in $\mathbb{G}$ multiplicatively
		      \item $h^{-1}$ denotes the inverse of $h \in \mathbb{G}$
		      \item $\log_g(h)$ denotes the unique $x \in [q]$ such that $g^x = h$
	      \end{itemize}
	\item We write $b \from \adv$ to denote the event that an adversary $\adv$ outputs the bit $b$ when playing a game where it must output a bit in the end
	\item For $a, b \in \{0, 1\}^n$, $a \oplus b$ denotes the XOR of $a$ and $b$
	\item We will stick to using $\kappa$ as the security parameter of private-key encryption schemes and $\eta$ as the parameter of public-key encryption schemes
	\item For a function $f$ in the security parameter $\eta$ (or $\kappa$) we will often omit writing $\eta$ as an argument and simply write $f$ to refer to $f(\eta)$
	\item For an encryption scheme $\Pi$ that contains an algorithm $X$, we may refer to $\Pi$'s implementation of $X$ by $\Pi.X$. (E.g. if $\Pi$ is an encryption scheme we can refer to its key-generation algorithm by $\Pi.X$)
\end{itemize}

\subsection{Encryption schemes}

\subsubsection{Private-key encryption}

\begin{definition}[Private-key encryption {\cite[Definition 3.7]{introduction-to-modern-cryptography}}]
	Let $\kappa$ denote the security parameter. A \emph{private-key encryption scheme} $\Pi$ consists of three probabilistic polynomial-time algorithms $(\gen, \enc, \dec)$ such that:
	\begin{enumerate}[1.]
		\item The \emph{key-generation algorithm} $\gen$ takes as input $1^\kappa$ (in unary) and outputs a key $k$. We will write $k \from \gen(1^\kappa)$.
		\item The \emph{encryption algorithm} $\enc$ takes as input a key $k$ and a message $m \in \{0, 1\}^*$, or $m \in \{0, 1\}^{\le l(\kappa)}$ for some function $l$ if the message space is finite, and outputs a ciphertext $c$. We write this as $c \from \enc_k(m)$.
		\item The deterministic \emph{decryption algorithm} $\dec$ takes as input a key $k$ and a ciphertext $c$, and outputs a message $m$ or $\bot$ (denoting an error). We write this as $m = \dec_{k}(c)$.
	\end{enumerate}

	It is required that for every $\kappa$, every key $k$ output by $\gen$, and every message $m$, it holds that $\pr{\dec_k(\enc_k(m)) = m} = 1$ (where the probability is over the randomness of $\enc_k$).
\end{definition}

\subsubsection{Public-key encryption}

In the following definition we will be more explicit about the randomness used by the algorithm $\gen$, as we will require a way to provide the randomness as input.

\begin{definition}[Public-key encryption {\cite[Definition 12.1]{introduction-to-modern-cryptography}}] \label{def:public-key-encryption}
	Let $\eta$ denote the security parameter.
	A \emph{public-key encryption scheme} $\Pi$ consists of three probabilistic polynomial-time algorithms $(\gen, \enc, \dec)$ such that:
	\begin{enumerate}[1.]
		\item The \emph{key-generation algorithm} $\gen$ takes as input $1^\eta$ (in unary) and outputs a pair of keys $(pk, sk)$ (a public and private key). We will write $(pk, sk) \from \gen(1^\eta)$.

		      The public key defines a message space $\mathcal{M}_{pk}$.

		      The algorithm samples $\rho(\eta)$ uniformly random bits to make randomized decisions for some function $\rho$ polynomial in $\eta$. The sequence of random bits $r \in \{0, 1\}^{\rho(\eta)}$ to be used by the algorithm may also be provided as input. We write this as $(pk, sk) = \gen(1^\eta, r)$ to emphasize the fact that the output is deterministic. The distribution over key pairs output by sampling $r \from \{0, 1\}^{\rho(\eta)}$ and running $\gen(1^\eta, r)$ is identical to the distribution over key pairs output by running $\gen(1^\eta)$.


		\item The \emph{encryption algorithm} $\enc$ takes as input a public key $pk$ and a message $m \in \mathcal{M}_{pk}$, and outputs a ciphertext $c$. We write this as $c \from \enc_{pk}(m)$.
		\item The deterministic \emph{decryption algorithm} $\dec$ takes as input a private key $sk$ and a ciphertext $c$, and outputs a message $m$ or $\bot$ (denoting an error). We write this as $m = \dec_{sk}(c)$.
	\end{enumerate}

	It is required that for every $\eta$, every key $(pk, sk)$ output by $\gen$, and every message $m$, it holds that $\pr{\dec_{sk}(\enc_{pk}(m)) = m} = 1$ (where the probability is over the randomness of $\enc_{pk}$).
\end{definition}

\subsection{Security definitions}

\begin{definition}[The IND-CPA game]
	Let $\kappa$ denote the security parameter and let $\Pi$ a private-key encryption scheme. Define the game $\game{\Pi}{\kappa}{IND-CPA}(\adv)$ for an adversary $\adv$:
	\begin{enumerate}[1.]
		\item A key $k \from \gen(1^\kappa)$ is generated.
		\item The adversary $\adv$ is given oracle access to $\Pi.\enc_k$, and outputs a pair of messages $m_0, m_1$ of the same length.
		\item A bit $b \from \{0, 1\}$ is sampled and $\adv$ is given a ciphertext $c \from \enc_k(m_b)$. ($\adv$ continues to have oracle access to $\Pi.\enc_k$.)
		\item $\adv$ outputs a bit $b'$. The output of the game is defined to be $1$ if $b' = b$, and $0$ otherwise.
	\end{enumerate}
\end{definition}

\begin{definition}[IND-CPA security {\cite[Definition 3.21]{introduction-to-modern-cryptography}}]
	For functions $t, \epsilon$ in the security parameter $\kappa$, a private-key encryption scheme $\Pi$ is \emph{$(t, \epsilon)$-IND-CPA-secure} if for all $\kappa$, for any adversary $\adv$ running in time $t(\kappa)$ we have
	\begin{align*}
		\advantage{\Pi}{\kappa}{IND-CPA}(\adv) \coloneqq 2 \cdot \left(\pr{\game{\Pi}{\kappa}{IND-CPA}(\adv) = 1} - \frac{1}{2}\right) \le \epsilon(\kappa).
	\end{align*}
\end{definition}

We will make use of a weaker form of security called ``indistinguishability in the presence of an eavesdropper'' \cite{introduction-to-modern-cryptography} and will refer to it as ``EAV security''. It is identical to IND-CPA security with the sole exception that the adversary does not have access to an encryption oracle.

\begin{definition}[The EAV game]
	Let $\kappa$ denote the security parameter and let $\Pi$ a private-key encryption scheme. Define the game $\game{\Pi}{\kappa}{EAV}(\adv)$ for an adversary $\adv$:
	\begin{enumerate}[1.]
		\item A key $k \from \gen(1^\kappa)$ is generated.
		\item The adversary $\adv$ outputs a pair of messages $m_0, m_1$ of the same length.
		\item A bit $b \from \{0, 1\}$ is sampled and $\adv$ is given a ciphertext $c \from \enc_k(m_b)$.
		\item $\adv$ outputs a bit $b'$. The output of the game is defined to be $1$ if $b' = b$, and $0$ otherwise.
	\end{enumerate}
\end{definition}

\begin{definition}[EAV security {\cite[Definition 3.8]{introduction-to-modern-cryptography}}]
	A private-key encryption scheme $\Pi$ is \emph{$(t, \epsilon)$-EAV-secure} if for all $\kappa$, for any adversary $\adv$ running in time $t(\kappa)$ we have
	\begin{align*}
		\advantage{\Pi}{\kappa}{EAV}(\adv) \coloneqq 2 \cdot \left(\pr{\game{\Pi}{\kappa}{EAV}(\adv) = 1} - \frac{1}{2}\right) \le \epsilon(\kappa).
	\end{align*}
\end{definition}


\begin{lemma}
	Let $\Pi$ a private-key encryption scheme. If $\Pi$ is $(t, \epsilon)$-IND-CPA-secure, then $\Pi$ is $(t, \epsilon)$-EAV-secure.
\end{lemma}
\begin{proof}
	This follows immediately from the fact that any EAV adversary is also an IND-CPA adversary.
\end{proof}

When analyzing the advantage of an adversary we may make use of the following well known equality.

\begin{lemma}
	Let $X$ a Bernoulli random variable and $b \from \{0, 1\}$ (where $X$ and $b$ are not necessarily independent). Then for $x \in \{0, 1\}$
	\[
		2 \cdot \left(\pr{X = b} - \frac{1}{2}\right) = \pr{X = x \mid b = x} - \pr{X = x \mid b = 1 - x}.
	\]
	In particular, if $\adv$ is an adversary with output in $\{0, 1\}$ playing a game where a bit $b \from \{0, 1\}$ is sampled, then for $x \in \{0, 1\}$
	\begin{equation} \label{eq:advantage-equality}
		2 \cdot \left(\pr{b \from \adv} - \frac{1}{2}\right) = \pr{x \from \adv \mid b = x} - \pr{x \from \adv \mid b = 1 - x}.
	\end{equation}
\end{lemma}
\begin{proof}
	Let $x \in \{0, 1\}$. We have
	\begin{align*}
		2 \cdot \left(\pr{X = b} - \frac{1}{2}\right) & = 2 \cdot \left(\pr{X = x \mid b = x} \cdot \frac{1}{2} + \pr{X = 1 - x \mid b = 1 - x} \cdot \frac{1}{2} - \frac{1}{2}\right) \\
		                                              & = \pr{X = x \mid b = x} + \pr{X = 1 - x \mid b = 1 - x} - 1                                                                    \\
		                                              & = \pr{X = x \mid b = x} - (1 - \pr{X = 1 - x \mid b = 1 - x})                                                                  \\
		                                              & = \pr{X = x \mid b = x} - \pr{X = x \mid b = 1 - x}.
	\end{align*}
\end{proof}

\subsection{The Decisional Diffie-Hellman problem}

\begin{definition}[Group-generation algorithm {\cite[Section 9.3.2]{introduction-to-modern-cryptography}}]
	A \emph{group-generation algorithm} $\mathcal{G}$ is a probabilistic polynomial-time algorithm that takes as input $1^\eta$ and outputs $(\mathbb{G}, q, g)$, where $\mathbb{G}$ is (a description of) a cyclic group with order $q$ and $g \in \mathbb{G}$ is a generator. A group element is represented as a bit-string of length at most $\gamma(\eta)$. We write $(\mathbb{G}, q, g) \from \mathcal{G}(1^\eta)$.
\end{definition}

\begin{definition}[The Decisional Diffie-Hellman (DDH) problem {\cite[Section 9.3.2]{introduction-to-modern-cryptography}}]
	Let $\mathcal{G}$ a group-generation algorithm.
	Define the game $\game{\mathcal{G}}{\eta}{DDH}(\adv)$ for an adversary $\adv$:
	\begin{enumerate}[1.]
		\item $\mathcal{G}(1^\eta)$ is run to obtain $(\mathbb{G}, q, g)$, and exponents $x, y \from [q]$ and a bit $b \from \{0, 1\}$ are sampled.
		\item The adversary $\adv$ is given $\mathbb{G}$, $q$, $g$, $h_1 \coloneqq g^x, h_2 \coloneqq g^y$ and
		      \[
			      k = \begin{cases}
				      g^{x \cdot y} & b = 0 \\
				      \tilde{k}     & b = 1
			      \end{cases}
		      \]
		      where $\tilde{k} \from \mathbb{G}$.
		\item $\adv$ outputs a bit $b'$. The output of the game is defined to be $1$ if $b' = b$, and $0$ otherwise.
	\end{enumerate}
\end{definition}

\begin{definition}[Hardness of the DDH problem {\cite[Definition 9.64]{introduction-to-modern-cryptography}}]
	The DDH problem is \emph{$(t, \epsilon)$-hard relative to} $\mathcal{G}$ if for all $\eta$, for any adversary $\adv$ running in time $t(\eta)$ we have
	\begin{align*}
		\advantage{\mathcal{G}}{\eta}{DDH}(\adv) \coloneqq 2 \cdot \left(\pr{\game{\mathcal{G}}{\eta}{DDH}(\adv) = 1} - \frac{1}{2}\right) \le \epsilon(\eta).
	\end{align*}
\end{definition}

\subsection{The Random Oracle Model} \label{sec:rom}

We will work in the commonly used Random Oracle Model (ROM) to prove our results. We refer the reader to \cite[Chapter 6.5]{introduction-to-modern-cryptography} for an informal overview of the ROM and to \cite{rom} for the original work that introduced the model. The ROM introduces the concept of a \emph{random oracle}. If a function $H : A \to B$ is modelled as a random oracle, then certain assumptions are made about what an adversary $\adv$ knows about $H$ and how it interacts with it:
\begin{itemize}
	\item From $\adv$'s perspective, $H$ is a black-box function. The only way for $\adv$ to interact with $H$ is for it to provide a value $a \in A$ and get back $H(a)$, and this is the only way for $\adv$ to learn $H(a)$. We say that $\adv$ \emph{queries} $H(a)$ or that $\adv$ \emph{queries $H$ for $a$}. This well-defined interface of $\adv$ to $H$ implies that a reduction can extract the queries that $\adv$ makes to $H$.
	\item From $\adv$'s perspective, $H$ is a random variable, sampled u.a.r.\ from the set of all functions from $A$ to $B$. Thus, if $\adv$ queries $H$ for some $a \in A$ that it has not queried before, the value $H(a)$ is a random variable uniformly distributed in $B$ from $\adv$'s perspective.
\end{itemize}
We do not rely on the property known as ``programmability'' in this work.


\section{Tighter GSD security}

\subsection{Runtime in Theorem~\ref{theorem:sdgsd-security}} \label{sec:sdgsd-security-runtime}

For completeness, we provide a more precise expression of the runtime $\tilde{t}$ in Theorem~\ref{theorem:sdgsd-security}. For appropriately chosen constants we have
\[
	\tilde{t} = t \begin{aligned}[t]
		- & \mathcal{O}
		\begin{aligned}[t]
			\big(\rho \cdot t_{\mathrm{sample}} \cdot \ms & + (\gamma + \eta \cdot t_{\mathrm{sample}}) \cdot \mdh \\ + & N \cdot ((\rho + \eta) \cdot t_{\mathrm{sample}} + \mdh \cdot t_{\mathrm{op}} + t_{\dhies.\gen})  \\ + & N^2 \cdot t_{\dhies.\enc}\big), \\
		\end{aligned}
	\end{aligned}
\]
where the various variables denote the following
\begin{itemize}
	\item $t_{\mathrm{sample}}$: time to sample a uniform bit
	\item $t_{\dhies.\enc}$: time to encrypt $s \in \{0, 1\}^\rho$ with $\dhies$
	\item $t_{\dhies.\gen}$: runtime of $\dhies.\gen(1^\eta)$ (which is strictly greater than the runtime of $\dhies.\gen(1^\eta, r)$ for input randomness $r$)
	\item $t_{\mathrm{op}}$: time to perform the group operation in a group output by $\mathcal{G}(1^\eta)$
	\item $\gamma$: maximum length of any query to $\hdh$
\end{itemize}

\subsection{Proof of Lemma~\ref{lemma:mis-eav-from-eav}} \label{sec:mis-eav-from-eav-proof}

\begin{proof}[of Lemma~\ref{lemma:mis-eav-from-eav}] Note that since the message space is finite, the time to encrypt a message is bounded. As outlined before Lemma~\ref{lemma:mis-eav-from-eav}, the lemma follows from a simple hybrid argument. Let $q$ a function in $\kappa$, let $\kappa$ arbitrary and let $\adv$ an arbitrary MIS-EAV adversary running in time $\tilde{t}(\kappa)$ and making at most $q(\kappa)$ queries. Define the sequence of hybrid games $G_0, \ldots, G_q$ where in the game $G_i$ the first $i$ encryptions given to the adversary encrypt $m_1$ and all remaining encryptions encrypt $m_0$. We will write
	\[
		\pr{0 \from \adv \mid G_i}
	\]
	for the probability of $\adv$ outputting $0$ when playing the hybrid game $G_i$.

	Let $i \in [q]$. Construct an EAV adversary $\adv'$ that behaves as follows:
	\begin{enumerate}[1.]
		\item $\adv'$ runs $\adv$ and gets $q, m_0, m_1$.
		\item $\adv'$ outputs the messages $m_0, m_1$ and gets a ciphertext $c$ from the challenger.
		\item $\adv'$ gives the ciphertexts $c_1, \ldots, c_q$ to $\adv$ where
		      \[
			      c_j = \begin{cases}
				      \Pi.\enc_{k_j}(m_1) & i < j \\
				      c                   & i = j \\
				      \Pi.\enc_{k_j}(m_0) & i > j
			      \end{cases}
		      \]
		      and $k_j \from \Pi.\gen(1^\kappa) \; \forall j$.
		\item $\adv'$ outputs whatever bit $\adv$ outputs.
	\end{enumerate}
	Now consider the value of the bit $b$ sampled in $\game{\Pi}{\kappa}{EAV}(\adv')$. If $b = 0$, then the first $i - 1$ ciphertexts that $\adv$ received were encryptions of $m_1$ and the remaining ciphertexts were encryptions of $m_0$, where all encryptions were under keys sampled independently with $\Pi.\gen$. Thus, from the view of $\adv$ everything followed the same distribution as in the game $G_{i - 1}$ and
	\[
		\pr{0 \from \adv' \mid b = 0} = \pr{0 \from \adv \mid G_{i - 1}}.
	\]
	Analogously, in the case $b = 1$ the first $i$ ciphertexts received by $\adv$ were encryptions of $m_1$ and the rest encryptions of $m_0$, so
	\[
		\pr{0 \from \adv' \mid b = 1} = \pr{0 \from \adv \mid G_{i}}.
	\]
	Then
	\begin{align} \label{eq:eav-to-mis-eav-hybrid-distinguisher}
		\begin{split}
			\begin{split}
				\pr{0 \from \adv \mid G_{i - 1}} - & \pr{0 \from \adv \mid G_{i}} \\
				& = \pr{0 \from \adv' \mid b = 0} - \pr{0 \from \adv' \mid b = 1}
			\end{split} \\
			& \stackrel{\mathclap{\eqref{eq:advantage-equality}}}{=} \advantage{\Pi}{\kappa}{EAV}(\adv')                                         \\
			& \le \epsilon
		\end{split}
	\end{align}
	by $(t, \epsilon)$-EAV security of $\Pi$ since $\adv'$ runs in time $\tilde{t} + \mathcal{O}(q \cdot (t_{\gen} + t_{\enc})) = t$.

	Now let $b$ be the bit sampled in the MIS-EAV game. Notice that
	\[
		\pr{0 \from \adv \mid b = 0} = \pr{0 \from \adv \mid G_0}
	\]
	and
	\[
		\pr{0 \from \adv \mid b = 1} = \pr{0 \from \adv \mid G_q}.
	\]
	Then
	\begin{align*}
		\advantage{\Pi}{\kappa}{MIS\text{-}EAV}(\adv) & \stackrel{\mathclap{\eqref{eq:advantage-equality}}}{=} \; \pr{0 \from \adv \mid b = 0} - \pr{0 \from \adv \mid b = 1} \\
		                                       & = \pr{0 \from \adv \mid G_0} - \pr{0 \from \adv \mid G_q}                                                             \\
		                                       & = \sum_{i = 1}^{q} \pr{0 \from \adv \mid G_{i - 1}} - \pr{0 \from \adv \mid G_i}                                      \\
		                                       & \stackrel{\mathclap{\eqref{eq:eav-to-mis-eav-hybrid-distinguisher}}}{\le} q \cdot \epsilon.
	\end{align*}
\end{proof}

\subsection{Proof of Lemma~\ref{lemma:eav-reduction}} \label{sec:eav-reduction-proof}

\begin{proof}[of Lemma~\ref{lemma:eav-reduction}]
	As already motivated after Lemma~\ref{lemma:eav-reduction}, we construct an MIS-EAV adversary $\adv'$ to derive the bound. $\adv'$ behaves as follows:
	\begin{enumerate}[1.]
		\item $\adv'$ runs $\adv$ to get $n$ and $D$ and initializes the GSD graph, seeds and the set of edges and corrupted nodes as in step \ref{def:sd-gsd-game-step-init} of the SD-GSD game.
		\item \label{eav-reduction-mis-eav-adversary-step-init} $\adv'$ samples $w \from [n], s \from \{0, 1\}^\rho$ and gives $\delta$ and the messages $s_w, s$ to the challenger. Let $c_1, \ldots, c_\delta$ the encryptions it gets back.
		\item $\adv'$ faithfully simulates the SD-GSD game to $\adv$ with the following exception: Whenever $\adv$ makes a query of the form $\operatorname{encrypt}(u, w)$ for any $u$, $\adv'$ replies with $\langle g^x, c_i \rangle$ where $x \from [q]$ and $i$ is the index of the next ciphertext (from step \ref{eav-reduction-mis-eav-adversary-step-init} of the MIS-EAV game) not yet used.

		      All random oracle queries are simulated by sampling the output of the oracle u.a.r. for new queries and using the value first sampled for repeated queries.

		      During the simulation $\adv'$ also pays attention to the following:
		      \begin{itemize}
			      \item If any of the following events occur, $\adv'$ aborts the simulation and outputs 1:
			            \begin{itemize}
				            \item $\adv$ queries $\hdh$ for a hidden Diffie-Hellman key
				            \item $\adv$ queries $\hgen$ or $\hdep$ for a hidden seed that is not $s_w$
				            \item $\adv$ queries $\operatorname{corrupt}(u)$ for some node $u$ such that $w$ is no longer safe
			            \end{itemize}
			      \item If $\adv$ queries $\hgen(s_w)$ or $\hdep(s_w)$, $\adv'$ aborts the simulation and outputs 0. This is the only point at which $\adv'$ outputs 0.
		      \end{itemize}

		      If the simulation arrives at the point where $\adv$ outputs its guess (step \ref{def:sd-gsd-game-step-output} of the SD-GSD game), then $\adv'$ outputs 1.
	\end{enumerate}

	The advantage of $\adv'$ is given by
	\begin{equation} \label{eq:lemma-eav-reduction-advantage}
		\advantage{\Pi_s}{\eta}{MIS\text{-}EAV}(\adv') \stackrel{\eqref{eq:advantage-equality}}{=}  \pr{0 \from \adv' \mid b = 0} - \pr{0 \from \adv' \mid b = 1},
	\end{equation}
	where $b$ is the bit sampled by the MIS-EAV challenger.


	First, we will show that
	\begin{equation} \label{eq:lemma-eav-reduction-b=0}
		\pr{0 \from \adv' \mid b = 0} \ge \frac{\pr{\qs \land \overline{\fdh}\,}}{N}.
	\end{equation}
	Let $E = \qs \land \overline{\fdh}$. In the following, while showing \eqref{eq:lemma-eav-reduction-b=0} we will implicitly assume that $b = 0$ when referring to an execution of $\game{\Pi_s}{\eta}{MIS\text{-}EAV}(\adv')$.
	On a high level \eqref{eq:lemma-eav-reduction-b=0} holds because as long as the game has not been aborted the encryptions $\adv$ receives from $\adv'$ are indistinguishable from what it would get in the real SD-GSD game and we get a factor $\frac{1}{N}$ from guessing the node that triggered $E$. However, showing this requires a few steps.

	Consider a modification of the SD-GSD game $G_1$ where the game is aborted whenever one of the following events occurs, where for all these events $\adv'$ would also abort the simulation:
	\begin{itemize}
		\item $\adv$ queries $\hdh$ for a hidden Diffie-Hellman key
		\item $\adv$ queries $\hgen$ or $\hdep$ for a hidden seed
	\end{itemize}
	(Since we are not interested in the output of the game we can define \emph{aborting the game} as the game ending with output 0.) The game $G_1$ is something between the real SD-GSD game and what $\adv'$ simulates to $\adv$. The only difference in when $G_1$ aborts compared to the game simulated by $\adv'$ is that we are not paying attention to some specific node $w$ remaining safe. Aborting the game in this way does not alter the probability of $\adv$ triggering the event $E$ in $G_1$, since in either case when the game is aborted either $E$ or $\overline{E}$ is already known to hold:
	\begin{itemize}
		\item If $\adv$ queries $\hdh$ for a hidden Diffie-Hellman key, then it triggers $\qdh$ and $\qs$ has not been triggered before since this would have caused the game to be aborted. Thus, $\adv$ triggered $\fdh$ and $\qs \land \overline{\fdh}$ cannot hold in this execution of the game.
		\item If $\adv$ queries $\hgen$ or $\hdep$ for a hidden seed, then this triggers $\qs$. Moreover, $\overline{\fdh}$ also holds at this moment since the game would have aborted earlier if $\qdh$ had already been triggered. Thus, $\qs \land \overline{\fdh}$ holds.
	\end{itemize}
	Let $E_1$ the same event as $E$ in the game $G_1$. As argued above we have
	\begin{equation} \label{eq:lemma-eav-reduction-E=E_1}
		\pr{E_1} = \pr{E}.
	\end{equation}


	Now consider a game $G_2$ which is a modification of the game $G_1$ where at the beginning of the game $w_2 \from [n]$ is sampled and the game also aborts if $\adv$ queries $\operatorname{corrupt}(u)$ for some node $u$ such that $w_2$ is no longer safe, just as in the game simulated by $\adv'$. The game $G_2$ is again something between the game $G_1$ and what $\adv'$ simulates to $\adv$. We also modify $G_1$ such that it also samples $w_1 \from [n]$ at the beginning of the game. This does not change the fact that \eqref{eq:lemma-eav-reduction-E=E_1} holds as the sampling of $w_1$ has no effect on the execution of the game.

	Let $E_2$ and $E'$ the events corresponding to $E$ in the game $G_2$ and the game simulated by $\adv'$, respectively. We further introduce a new random variable $W$ to analyze each game where
	\[
		W = \begin{cases}
			0 & \overline{E}                       \\
			x & E \text{ was triggered at node } x
		\end{cases}
	\]
	(if $x$ is not unique we choose the node with the lowest identifier).
	Let $W_1$, $W_2$ and $W'$ be the corresponding random variables in game $G_1$, game $G_2$ and the game simulated by $\adv'$. Consider the probability $\pr{W_1 = w_1 \mid E_1}$. The node $w_1$ is sampled independently and does not affect the execution of the game. Therefore, in an execution where $E_1$ occurs and the GSD graph has size $n$ (so $W_1 \in [n]$), we correctly guess $W_1 = w_1$ with probability exactly $\frac{1}{n} \ge \frac{1}{N}$. Thus
	\[
		\pr{W_1 = w_1 \mid E_1} \ge \frac{1}{N}
	\]
	and combining this with \eqref{eq:lemma-eav-reduction-E=E_1} we get
	\begin{align} \label{eq:lemma-eav-reduction-detect-E1-G1}
		\begin{split}
			\pr{W_1 = w_1} & = \pr{W_1 = w_1 \land E_1} \\
			& = \pr{W_1 = w_1 \mid E_1} \cdot \pr{E_1} \\
			& \ge \frac{1}{N} \cdot \pr{E}.
		\end{split}
	\end{align}

	Analogously to the argument used to justify \eqref{eq:lemma-eav-reduction-E=E_1}, we can argue that
	\begin{equation} \label{eq:lemma-eav-reduction-G1-vs-G2}
		\pr{W_1 = w_1} = \pr{W_2 = w_2}.
	\end{equation}
	The only difference from $G_1$ to $G_2$ is that $G_2$ aborts when $w_2$ is no longer safe. But if $w_2$ is no longer safe then we know that $W_2 \neq w_2$ (if $W_2 = w_2$ the game would have already aborted when $w_2$'s seed was queried while it was safe). Thus, \eqref{eq:lemma-eav-reduction-G1-vs-G2} indeed holds.

	We now show an analogous result comparing the game $G_2$ to the game simulated by $\adv'$:
	\begin{equation} \label{eq:lemma-eav-reduction-G2-vs-simulation}
		\pr{W_2 = w_2} = \pr{W' = w}.
	\end{equation}
	Consider how $G_2$ differs from the game simulated by $\adv'$. Both games abort at exactly the same events (this is easy to see). They only differ in how $\adv'$ answers queries $\operatorname{encrypt}(u, w)$ for any $u$. In $G_2$ such a query is answered with a ciphertext $\langle g^x, c \rangle$ where $x \from [q], c \from \Pi_s.\enc_k(s_w)$ and $k = \hdh(pk_u^x)$. $\adv'$ answers such a query with $\langle g^{x'}, c' \rangle$ where $x' \from [q], c' \from \Pi_s.\enc_{k'}(s_w)$ and $k' \from \{0, 1\}^\eta$. Now notice that as long as the game $G_2$ is ongoing, $pk_u^{x}$ is a hidden Diffie-Hellman key and $\adv$ has not queried $pk_u^{x}$ to $\hdh$. If it had, then the game would have already aborted. Therefore, from $\adv$'s view $k$ follows the same distribution as $k'$. Thus, overall the game $G_2$ and the game simulated by $\adv'$ are indistinguishable to $\adv$ and \eqref{eq:lemma-eav-reduction-G2-vs-simulation} holds.

	Finally, notice that if the event $W' = w$ occurred, then $\adv'$ outputs $0$. Then we have
	\begin{align*}
		\pr{0 \from \adv' \mid b = 0} & \ge \pr{W' = w}                                                              \\
		                              & \stackrel{\eqref{eq:lemma-eav-reduction-G2-vs-simulation}}{=} \pr{W_2 = w_2} \\
		                              & \stackrel{\eqref{eq:lemma-eav-reduction-G1-vs-G2}}{=}  \pr{W_1 = w_1}        \\
		                              & \stackrel{\eqref{eq:lemma-eav-reduction-detect-E1-G1}}{\ge} \frac{\pr{E}}{N} \\
		                              & = \frac{\pr{\qs \land \overline{\fdh}\,}}{N},
	\end{align*}
	as promised.

	Second, returning to \eqref{eq:lemma-eav-reduction-advantage}, we can more easily show that $\pr{0 \from \adv' \mid b = 1}$ is negligible. In the SD-GSD game simulated to $\adv$ during an execution of $\game{\Pi_s}{\eta}{MIS\text{-}EAV}(\adv')$ with $b = 1$, the seed $s_w$ is a random variable independent of any information given to $\adv$:
	\begin{itemize}
		\item the game aborts when $w$ becomes unsafe, so $s_w$ cannot be learned by querying $\operatorname{corrupt}(w)$ or by querying $\hdep(s_p)$ for an unsafe node $p$ where $(p, w)$ is a seed dependency
		\item querying $\hdep(s_p)$ for a safe node $p$ where $(p, w)$ is a seed dependency results in the game being aborted and by virtue of $\hdep$ being a random oracle, from $\adv$'s perspective $s_w$ follows the same distribution regardless of whether there is a seed dependency $(p, w)$ or not
		\item with $b = 1$ queries $\operatorname{encrypt}(u, w)$ yield encryptions of $s$ instead of $s_w$
	\end{itemize}
	Therefore, for any seed $s'$ that $\adv$ queries to $\hgen$ or $\hdep$ we have
	\[
		\pr{s_w = s'} = \frac{1}{2^\rho}.
	\]
	Thus, by a union bound we have
	\begin{equation} \label{eq:lemma-eav-reduction-b=1}
		\pr{0 \from \adv' \mid b = 1} \le \frac{\ms}{2^\rho}.
	\end{equation}

	Combining \eqref{eq:lemma-eav-reduction-advantage}, \eqref{eq:lemma-eav-reduction-b=0} and \eqref{eq:lemma-eav-reduction-b=1} we get
	\begin{align} \label{eq:lemma-eav-reduction-advantage-lower-bound}
		\begin{split}
			\advantage{\Pi_s}{\eta}{MIS\text{-}EAV}(\adv') & = \pr{0 \from \adv' \mid b = 0} - \pr{0 \from \adv' \mid b = 1}           \\
			& \ge \frac{\pr{\qs \land \overline{\fdh}\,}}{N} - \frac{\ms}{2^\rho}.
		\end{split}
	\end{align}
	Furthermore, going through the details yields that $\adv'$ runs in time
	\begin{align*}
		\begin{split}
			t_{\adv'} \coloneqq \tilde{t} + \mathcal{O}\big(\rho \cdot t_{\mathrm{sample}} \cdot \ms & + (\gamma + \eta \cdot t_{\mathrm{sample}}) \cdot \mdh  \\
			& + N \cdot (\rho \cdot t_{\mathrm{sample}} + t_{\dhies.\gen})  \\
			& +  N^2 \cdot t_{\dhies.\enc}\big)
		\end{split}
	\end{align*}
	(note that $t_{\adv'}$ is a constant).
	Using that $\delta \le N, t_{\Pi_s.\gen} \le \mathcal{O}(\eta \cdot t_{\mathrm{sample}})$, $t_{\Pi_s.\enc} \le t_{\dhies.\enc}$ (as encrypting with $\dhies$ involves an encryption with $\Pi_s$), the definition of $\tilde{t}$, with appropriately chosen constants we have
	\[
		t_{\adv'} \le t - \mathcal{O}(\delta \cdot (t_{\Pi_s.\gen} + t_{\Pi_s.\enc})).
	\]
	By Lemma~\ref{lemma:mis-eav-from-eav} $\Pi_s$ is $(t - \mathcal{O}(\delta \cdot (t_{\Pi_s.\gen} + t_{\Pi_s.\enc})), \delta \cdot \eeav, \delta)$-MIS-EAV-secure, so
	\begin{equation} \label{eq:lemma-eav-reduction-advantage-upper-bound}
		\advantage{\Pi_s}{\eta}{MIS\text{-}EAV}(\adv') = \delta \cdot \eeav.
	\end{equation}

	Finally, if we now combine \eqref{eq:lemma-eav-reduction-advantage-lower-bound} and \eqref{eq:lemma-eav-reduction-advantage-upper-bound} we get
	\begin{align*}
		\frac{\pr{\qs \land \overline{\fdh}\,}}{N} - \frac{\ms}{2^\rho} & \le \delta \cdot \eeav                                       \\
		                                                                & \iff                                                         \\
		\pr{\qs \land \overline{\fdh}\,}                                & \le \delta \cdot N \cdot \eeav + \frac{\ms \cdot N}{2^\rho},
	\end{align*}
	as was to prove.
\end{proof}


\subsection{Tighter MIS-EAV security for certain schemes} \label{sec:tighter-mis-eav-security}

In our reduction from MIS-EAV security to EAV security (Lemma~\ref{lemma:mis-eav-from-eav}) we applied a general hybrid argument. It is also tempting to try a more direct approach. The EAV and MIS-EAV games seem less far apart than IND-CPA for single and multiple encryptions: All additional encryptions in the MIS-EAV game encrypt the same message, with the only difference being that each encryption is performed using a fresh key. If only we could take a single encryption $c \from \enc_k(m)$ and from it produce several encryptions $c_i \from \enc_{k_i}(m)$ for $k_i \from \gen(1^\kappa)$ (without knowing $k$ or $m$), then the additional encryptions would leak no new information to the adversary, and we would have a tight bound on MIS-EAV security from EAV security. There is a simple EAV secure scheme that achieves the above property: the one-time pad. Given an encryption $c = k \oplus m$, we can just sample $k' \from \{0, 1\}^\kappa$ and compute the ciphertext $c' = c \oplus k' = (k \oplus k') \oplus m$, an encryption of $m$ under the uniformly random key $k \oplus k'$. In the following, we formalize this property of a private-key encryption scheme and use it to prove the desired bound on MIS-EAV security.

\begin{definition}[Key-rerandomizability] \label{def:key-rerandomizability}
	Let $\kappa$ denote the security parameter and let $\Pi = (\gen, \enc, \dec)$ a private-key encryption scheme. $\Pi$ is \emph{key-rerandomizable} if there exists a probabilistic polynomial-time algorithm $\operatorname{ReRan}$ achieving the following: Let $\kappa, k \from \gen(1^\kappa)$, $m$ in the message space and $c \from \enc_k(m)$ be arbitrary but fixed.\footnote{Here we are quantifying over all possible keys $k$ and ciphertexts $c$ that can be output by $\gen(1^\kappa)$ and $\enc_k(m)$.} Then the distribution over ciphertexts as defined by computing $c' \from \operatorname{ReRan}(1^\kappa, c)$ is identical to the distribution over ciphertexts resulting from the process of first sampling $k' \from \gen(1^\kappa)$ and then computing a ciphertext $c' \from \enc_{k'}(m)$.
\end{definition}

\paragraph{Example} As outlined above, the one-time pad is an example of a key-rerandomizable encryption scheme.

\question{Is there a key-rerandomizable IND-CPA secure scheme? If yes, this would imply a key-rerandomizable AE scheme using the encrypt-then-authenticate paradigm, since a rerandomized tag can easily produced for the ciphertext by sampling a fresh MAC key.}

The key idea underlying the proof of the following Lemma was already provided at the beginning of this section.

\begin{lemma} \label{lemma:mis-eav-from-eav-with-key-rerandomizability}
	Let $\Pi$ a key-rerandomizable private-key encryption scheme with finite message space. Let $\operatorname{ReRan}$ the corresponding algorithm to rerandomize ciphertexts and $t_{\operatorname{ReRan}}$ an upper bound for the runtime of $\operatorname{ReRan}$. If $\Pi$ is $(t, \epsilon)$-EAV-secure, then for all $q \in \N$, $\Pi$ is $(\tilde{t}, \epsilon, q)$-MIS-EAV-secure with $\tilde{t} = t - \mathcal{O}(q \cdot t_{\operatorname{ReRan}})$.
\end{lemma}

\begin{proof}
	Note that since the message space and thus the ciphertext space is finite, the runtime of $\operatorname{ReRan}$ is indeed bounded. Let $\kappa$ arbitrary. Let $\adv$ an MIS-EAV adversary running in time $\tilde{t}(\kappa)$ and making at most $q(\kappa)$ queries. We construct an EAV adversary $\adv'$ that behaves as follows:
	\begin{enumerate}[1.]
		\item $\adv'$ runs $\adv$ to get the number of queries $q$ and messages $m_0, m_1$.
		\item $\adv'$ gives $m_0, m_1$ to the challenger and receives the ciphertext $c_1$.
		\item $\adv'$ computes ciphertexts $c_2 \from \operatorname{ReRan}(1^\kappa, c_1), \ldots, c_q \from \operatorname{ReRan}(1^\kappa, c_1)$ (with independent runs of $\operatorname{ReRan}$).
		\item $\adv'$ gives the ciphertexts $c_1, \ldots, c_q$ to $\adv$.
		\item $\adv'$ outputs whatever bit $\adv$ outputs.
	\end{enumerate}
	We apply the properties of $\operatorname{ReRan}$ given in Definition~\ref{def:key-rerandomizability} to show that the game simulated to $\adv$ is distributed identically to the MIS-EAV game. For this we need only show that the ciphertexts $c_1, \ldots, c_q$ given to $\adv$ in the simulation are distributed identically to the ciphertexts $c_1', \ldots, c_q'$ that $\adv$ would get in the real MIS-EAV game. It is immediate that $c_1$ is distributed identically to $c_1'$. Now let $i \in \{2, \ldots, q\}$. By Definition~\ref{def:key-rerandomizability} $\operatorname{ReRan}(c)$ outputs a ciphertext encrypting $m_b$ (where $b$ is the bit chosen by the EAV challenger) distributed identically to a ciphertext encrypting $m_b$ output by the MIS-EAV challenger. Thus, indeed for any $i$, $c_i$ is distributed identically to $c_i'$ and the claim holds. Therefore
	\begin{equation} \label{eq:lemma-key-rerandomizability-advantage-same}
		\advantage{\Pi}{\kappa}{MIS\text{-}EAV}(\adv) = \advantage{\Pi}{\kappa}{EAV}(\adv').
	\end{equation}

	Because $\adv'$ is an EAV adversary running in time $\tilde{t} + \mathcal{O}(q \cdot t_{\operatorname{ReRan}}) = t$ we know that
	\[
		\advantage{\Pi}{\kappa}{EAV}(\adv') \le \epsilon(\kappa),
	\]
	which together with \eqref{eq:lemma-key-rerandomizability-advantage-same} concludes the proof.
\end{proof}

By assuming a key-rerandomizable encryption scheme and applying Lemma~\ref{lemma:mis-eav-from-eav-with-key-rerandomizability} instead of the hybrid argument (Lemma~\ref{lemma:mis-eav-from-eav}) in the proof of Lemma~\ref{lemma:eav-reduction}, we can drop the $\delta$ factor in the bound. This also allows us to drop the $\delta$ factor in Theorem~\ref{theorem:sdgsd-security}.

\begin{corollary}
	Recall the setting of Theorem~\ref{theorem:sdgsd-security}. If the private-key encryption scheme $\Pi_s$ is additionally key-rerandomizable, then the bound in Lemma~\ref{lemma:eav-reduction} can be improved to
	\[
		\pr{\qs \land \overline{\fdh}\,} \le N \cdot \eeav + \frac{\ms \cdot N}{2^\rho}
	\]
	and the bound $\tilde{\epsilon}$ on the success probability of an SD-GSD adversary thus improved to
	\[
		\tilde{\epsilon} = 2 \cdot N \cdot (\eeav + \eddh) + \frac{2 \cdot \mdh \cdot N^2}{q} + \frac{\ms \cdot N}{2^{\rho - 1}}
	\]
	(with appropriate changes to the runtime $\tilde{t}$).
\end{corollary}

\subsection{Proof of Lemma~\ref{lemma:dh-reduction}} \label{sec:dh-reduction-proof}

\begin{proof}[of Lemma~\ref{lemma:dh-reduction}]
	As outlined after Lemma~\ref{lemma:dh-reduction} we use $\adv$ to construct a DDH adversary $\adv'$:
	\begin{enumerate}[1.]
		\item $\adv'$ gets $h_1, h_2$ and $k$ from the DDH challenger.
		\item $\adv'$ runs $\adv$ to get $n$ and $D$, samples $a \from [n]$ and initializes the GSD graph, seeds and the set of edges and corrupted nodes as in step \ref{def:sd-gsd-game-step-init} of the SD-GSD game, with the sole exception that $pk_a = h_1$ (as opposed to setting it to the public key output by $\dhies.\gen(1^\eta, \hgen(s_a))$).
		\item $\adv'$ faithfully simulates the SD-GSD game to $\adv$ with the following exception: For the $j$-th query $\operatorname{encrypt}(a, u_j)$ made by $\adv$, $\adv'$ replies with $\langle h_2 \cdot g^{r_j}, \Pi_s.\enc_{k_j}(s_{u_j}) \rangle$ where $r_j \from [q]$, $k_j \from \{0, 1\}^\eta$. $\adv'$ also computes and stores $R_j = \left(pk_a^{r_j}\right)^{-1}$.

		      All random oracle queries are simulated by sampling the output of the oracle u.a.r. for new queries and using the value first sampled for repeated queries.

		      During the simulation $\adv'$ also pays attention to the following:
		      \begin{itemize}
			      \item If any of the following events occur, $\adv'$ aborts the simulation and outputs 1:
			            \begin{itemize}
				            \item $\adv$ queries $\hdh$ for a hidden Diffie-Hellman key on an encryption edge $(u, v) \in E$ with $u \neq a$
				            \item $\adv$ queries $\hgen$ or $\hdep$ for a hidden seed
				            \item $\adv$ queries $\operatorname{corrupt}(u)$ for some node $u$ such that $a$ is no longer safe
			            \end{itemize}
			      \item If $\adv$ queries $q_i$ to $\hdh$ such that $q_i \cdot R_j = k$ for some $j$, $\adv'$ aborts the simulation and outputs 0. This is the only point at which $\adv'$ outputs 0.
		      \end{itemize}

		      If the simulation arrives to the point where $\adv$ outputs its guess (step \ref{def:sd-gsd-game-step-output} of the SD-GSD game), then $\adv'$ outputs 1.
	\end{enumerate}

	The advantage of $\adv'$ is given by
	\begin{equation} \label{eq:lemma-dh-reduction-advantage}
		\advantage{\mathcal{G}}{\eta}{DDH}(\adv') \stackrel{\eqref{eq:advantage-equality}}{=} \pr{0 \from \adv' \mid b = 0} - \pr{0 \from \adv' \mid b = 1},
	\end{equation}
	where $b$ is the bit sampled by the DDH challenger.

	First, we will show that
	\begin{equation} \label{eq:lemma-dh-reduction-b=0}
		\pr{0 \from \adv' \mid b = 0} \ge \frac{\pr{\fdh}}{N}.
	\end{equation}
	This part of the proof proceeds very similarly to the proof of Lemma~\ref{lemma:eav-reduction} and we will be a bit more concise. We focus on executions of $\game{\mathcal{G}}{\eta}{DDH}(\adv')$ with $b = 0$. Let the games $G_1, G_2$ be defined as in Lemma~\ref{lemma:eav-reduction}, where we denote the node sampled at the beginning of each game by $a_1, a_2$, respectively (as opposed to $w_1, w_2$). Let $E = \fdh$ and let $E_1, E_2$ and $E'$ be the analogous events in $G_1, G_2$ and the game simulated by $\adv'$ (note that in this latter game, the group elements $pk_a^{\log_g(h_2) + r_j}$ are also hidden Diffie-Hellman keys). Finally, we introduce the random variable
	\[
		A = \begin{cases}
			0 & \overline{\fdh}                                                                    \\
			x & \text{$\fdh$ holds and $\qdh$ was triggered on an encryption edge with source $x$}
		\end{cases}
	\]
	(if $x$ is not unique we choose the node with the lowest identifier) and let $A_1, A_2$ and $A'$ denote the corresponding random variables in game $G_1$, game $G_2$ and the game simulated by $\adv'$.

	Just as argued in Lemma~\ref{lemma:eav-reduction},
	\begin{equation} \label{eq:lemma-dh-reduction-E=E_1}
		\pr{E_1} = \pr{E}
	\end{equation}
	holds, since whenever $G_1$ aborts, it is already decided whether $\fdh$ holds:
	\begin{itemize}
		\item If the game was aborted when $\adv$ queried a hidden Diffie-Hellman key, then $\fdh$ holds.
		\item If the game was aborted when $\adv$ queried $\hgen$ or $\hdep$ for a hidden seed, $\fdh$ does not hold.
	\end{itemize}

	Next, the inequality
	\[
		\pr{A_1 = a_1 \mid E_1} \ge \frac{1}{N}
	\]
	and therefore also
	\begin{equation} \label{eq:lemma-dh-reduction-detect-E1-G1}
		\pr{A_1 = a_1} \ge \frac{1}{N} \cdot \pr{E}
	\end{equation}
	hold for the same reason that
	\[
		\pr{W_1 = w_1 \mid E_1} \ge \frac{1}{N}
	\]
	and \eqref{eq:lemma-eav-reduction-detect-E1-G1} held in Lemma~\ref{lemma:eav-reduction}.

	Then, the equality
	\begin{equation} \label{eq:lemma-dh-reduction-G1-vs-G2}
		\pr{A_1 = a_1} = \pr{A_2 = a_2}
	\end{equation}
	holds again due to the fact that when $G_2$ aborts because $a_2$ is no longer safe, we know that $A_2 \neq a_2$.

	Finally, we need to argue that
	\begin{equation} \label{eq:lemma-dh-reduction-G2-vs-simulation}
		\pr{A_2 = a_2} = \pr{A' = a}.
	\end{equation}
	Consider how $G_2$ differs from the game simulated by $\adv'$. As in Lemma~\ref{lemma:eav-reduction}, both games abort at exactly the same events (note that if $q_i \cdot R_j = k$ holds and $\adv'$ outputs $0$, then $q_i = k \cdot R_j^{-1} = k \cdot pk_a^{r_j} = h_1^{\log_g(h_2)} \cdot pk_a^{r_j} = pk_a^{\log_g(h_2) + r_j}$, a hidden Diffie-Hellman key). The game simulated by $\adv'$ differs in two aspects:
	\begin{enumerate}[(i)]
		\item $\adv'$ sets $pk_a$ to $h_1$ and not to the public key output by $\dhies.\gen(1^\eta, \hgen(s_a))$
		\item $\adv'$ answers queries $\operatorname{encrypt}(a, u)$ differently
	\end{enumerate}
	Note that as long as the game $G_2$ is ongoing, $\adv$ has not queried $\hgen$ for $s_a$ or $\hdh$ for a hidden Diffie-Hellman key. Both differences are therefore indistinguishable:
	\begin{enumerate}[(i)]
		\item By Definition~\ref{def:public-key-encryption}, the output of $\dhies.\gen(1^\eta, r)$ with $r \from \{0, 1\}^\rho$ follows the same distribution as the output of $\dhies.\gen(1^\eta)$. The former process is behind the distribution of $pk_a$ as viewed from $\adv$ in $G_2$, as $\adv$ has not queried $\hgen(s_a)$, and the latter process is behind the distribution of $pk_a$ in the game simulated by $\adv'$, as the DDH challenger generates a public key with the same distribution as $\dhies.\gen(1^\eta)$. Since both processes follow the same distribution, $pk_a$ follows the same in $G_2$ and the game simulated by $\adv'$ from $\adv$'s perspecive.
		\item In $G_2$ a query $\operatorname{encrypt}(a, u)$ is answered with $\langle g^z, c \rangle$ where $z \from [q], c \from \Pi_s.\enc_k(s_u)$ and $k = \hdh(pk_a^z)$. $\adv'$ answers such a query with $\langle g^{\log_g(h_1) + r}, c' \rangle$ where $r \from [q], c' \from \Pi_s.\enc_{k'}(s_u)$ and $k' \from \{0, 1\}^\eta$. First, $\log_g(h_1) + r$ follows the same distribution as $z$. Second, $pk_a^z$ is a hidden Diffie-Hellman key and from $\adv$'s view $k$ follows the same distribution as $k'$.
	\end{enumerate}
	Thus \eqref{eq:lemma-dh-reduction-G2-vs-simulation} indeed holds.

	Now, again analogous to Lemma~\ref{lemma:eav-reduction} if the event $A' = a$ occurred, then $\adv'$ outputs $0$ and
	\begin{align*}
		\pr{0 \from \adv' \mid b = 0} & \ge \pr{A' = a}                                                                             \\
		                              & \stackrel{\mathclap{\eqref{eq:lemma-dh-reduction-G2-vs-simulation}}}{=} \;\; \pr{A_2 = a_2} \\
		                              & \stackrel{\mathclap{\eqref{eq:lemma-dh-reduction-G1-vs-G2}}}{=} \;\; \pr{A_1 = a_1}         \\
		                              & \stackrel{\mathclap{\eqref{eq:lemma-dh-reduction-detect-E1-G1}}}{\ge} \;\; \frac{\pr{E}}{N} \\
		                              & = \frac{\pr{\fdh}}{N}.
	\end{align*}

	Second, we will show that $\pr{0 \from \adv' \mid b = 1}$ is negligible. When $b = 1$ in $\game{\mathcal{G}}{\eta}{DDH}(\adv')$, $k$ is a uniformly random group element independent of any information given to $\adv$, in particular of $q_i \cdot R_j$ for any $i, j$. Thus for any $i, j$,
	\[
		\pr{q_i \cdot R_j = k} = \frac{1}{q}.
	\]
	Thus, by a union bound and using that $i \in [\mdh], 1 \le j \le N - 1 \le N$ ($j$ is bounded by the maximum outdegree) we have
	\begin{equation} \label{eq:lemma-dh-reduction-b=1}
		\pr{0 \from \adv' \mid b = 1} \le \frac{\mdh \cdot N}{q}.
	\end{equation}

	Combining \eqref{eq:lemma-dh-reduction-advantage}, \eqref{eq:lemma-dh-reduction-b=0} and \eqref{eq:lemma-dh-reduction-b=1} we get
	\begin{align} \label{eq:lemma-dh-reduction-advantage-lower-bound}
		\begin{split}
			\advantage{\mathcal{G}}{\eta}{DDH}(\adv') \ge \frac{\pr{\fdh}}{N} - \frac{\mdh \cdot N}{q}.
		\end{split}
	\end{align}

	Furthermore, going through the details yields that $\adv'$ runs in time
	\begin{align*}
		\begin{split}
			t_{\adv'} \coloneqq \tilde{t} + \mathcal{O}\big(\rho \cdot t_{\mathrm{sample}} \cdot \ms & + (\gamma + \eta \cdot t_{\mathrm{sample}}) \cdot \mdh \\
			& + N \cdot ((\rho + \eta) \cdot t_{\mathrm{sample}} + \mdh \cdot t_{\mathrm{op}} + t_{\dhies.\gen})  \\
			& +  N^2 \cdot t_{\dhies.\enc}\big).
		\end{split}
	\end{align*}
	Then using the definition of $\tilde{t}$, with appropriately chosen constants we have $t_{\adv'} \le t$. So by virtue of the DDH problem being $(t, \eddh)$-hard relative to $\mathcal{G}$
	\[
		\advantage{\mathcal{G}}{\eta}{DDH}(\adv') \le \eddh
	\]
	and if we combine this with \eqref{eq:lemma-dh-reduction-advantage-lower-bound} we get
	\begin{align*}
		\frac{\pr{\fdh}}{N} - \frac{\mdh \cdot N}{q} & \le \eddh                                     \\
		                                             & \iff                                          \\
		\pr{\fdh}                                    & \le N \cdot \eddh + \frac{\mdh \cdot N^2}{q},
	\end{align*}
	concluding the proof.
\end{proof}





\end{document}
