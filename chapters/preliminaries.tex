\chapter{Preliminaries}

\section{Notation}

We will use the following notation throughout:
\begin{itemize}
	\item We write $x \from S$ to say that $x$ is sampled uniformly at random from the finite set $S$
	\item If $\mathbb{G}$ is a cyclic group and $g$ a generator, then
	      \begin{itemize}
		      \item We write the group operation in $\mathbb{G}$ multiplicatively
		      \item $h^{-1}$ denotes the inverse of $h \in \mathbb{G}$
		      \item $\log_g(h)$ denotes the unique $x \in [\abs{\mathbb{G}}]$ such that $g^x = h$
	      \end{itemize}
	\item We write $b \from \adv$ to denote the event that a probabilistic polynomial time adversary $\adv$ outputs the bit $b$ when playing a game where it must output a bit in the end
	\item For $a, b \in \{0, 1\}^n$, $a \oplus b$ denotes the XOR of $a$ and $b$
	\item For $n \in \mathbb{N} \setminus \{0\}$, $[n] = \{1, \ldots, n\}$
\end{itemize}


\section{Basic definitions}

The definitions presented in this section were taken from \cite{introduction-to-modern-cryptography}.

\subsection{Encryption schemes}

\subsubsection{Private-key encryption}

\begin{definition}[Private-key encryption {\cite[Definition 3.7]{introduction-to-modern-cryptography}}]
	A \emph{private-key encryption scheme} $\Pi$ consists of three probabilistic polynomial-time algorithms $(\gen, \enc, \dec)$ such that:
	\begin{enumerate}[1.]
		\item The \emph{key-generation algorithm} $\gen$ takes as input $1^n$ (in unary) where $n$ is the security parameter and outputs a key $k$. We will assume the security parameter to be fixed and write $k \from \gen()$.
		\item The \emph{encryption algorithm} $\enc$ takes as input a key $k$ and a plaintext message $m \in \{0, 1\}^*$ , or $m \in \{0, 1\}^{\le \eta}$ for some $\eta$ if the message space is finite, and outputs a ciphertext $c$. We write this as $c \from \enc_k(m)$.
		\item The \emph{decryption algorithm} $\dec$ takes as input a key $k$ and a ciphertext $c$, and outputs a message $m$ or $\bot$ (denoting an error). We write this as $m = \dec_{sk}(c)$.
	\end{enumerate}

	We may also refer to algorithm $X$ by $\Pi.X$ for $X \in \{\gen, \enc, \dec\}$.

	It is required that for every $n$, every key $k$ output by $\gen$, and every message $m$, it holds that $\dec_k(\enc_k(m)) = m$.
\end{definition}

\subsubsection{Public-key encryption}

\begin{definition}[Public-key encryption {\cite[Definition 12.1]{introduction-to-modern-cryptography}}]
	A \emph{public-key encryption scheme} $\Pi$ consists of three probabilistic polynomial-time algorithms $(\gen, \enc, \dec)$ such that:
	\begin{enumerate}[1.]
		\item The \emph{key-generation algorithm} $\gen$ takes as input $1^n$ (in unary) where $n$ is the security parameter and outputs a pair of keys $(pk, sk)$ (a public and private key). We will assume the security parameter to be fixed and write $(pk, sk) \from \gen()$.
		\item The \emph{encryption algorithm} $\enc$ takes as input a public key $pk$ and a plaintext message $m \in \mathcal{M}$ where $\mathcal{M}$ is the message space and outputs a ciphertext $c$. We write this as $c \from \enc_{pk}(m)$.
		\item The \emph{decryption algorithm} $\dec$ takes as input a private key $sk$ and a ciphertext $c$, and outputs a message $m$ or $\bot$ (denoting an error). We write this as $m = \dec_{sk}(c)$.
	\end{enumerate}

	We may also refer to algorithm $X$ by $\Pi.X$ for $X \in \{\gen, \enc, \dec\}$.

	It is required that for every $n$, every key $(pk, sk)$ output by $\gen$, and every message $m$, it holds that $\dec_{sk}(\enc_{pk}(m)) = m$.
\end{definition}

\subsection{Security definitions}

\begin{definition}[The IND-CPA game]
	Let $\Pi$ a private-key encryption scheme. Define the game $\mathrm{Game}_{\adv, \Pi}^{\mathrm{IND-CPA}}$ for an adversary $\adv$:
	\begin{enumerate}[1.]
		\item A key $k \from \gen()$ is generated.
		\item The adversary $\adv$ is given oracle access to $\Pi.\enc_k$ and outputs a pair of messages $m_0, m_1$ of the same length.
		\item A bit $b \from \{0, 1\}$ is sampled and $\adv$ is given a ciphertext $c \from \enc_k(m_b)$. ($\adv$ continues to have oracle access to $\Pi.\enc_k$.)
		\item $\adv$ outputs a bit $b'$. The output of the game is defined to be $1$ if $b' = b$, and $0$ otherwise.
	\end{enumerate}
\end{definition}

\begin{definition}[IND-CPA security {\cite[Definition 3.21]{introduction-to-modern-cryptography}}]
	A private-key encryption scheme $\Pi$ is \emph{$(t, \epsilon, q)$-IND-CPA secure} if for any adversary $\adv$ running in time $t$ we have
	\begin{align*}
		\mathrm{Adv}_{\Pi}^{\mathrm{IND-CPA}}(\adv) \coloneqq 2 \cdot \left(\pr{\mathrm{Game}_{\adv, \Pi}^{\mathrm{IND-CPA}} = 1} - \frac{1}{2}\right) \le \epsilon.
	\end{align*}
\end{definition}

When analyzing the advantage of an adversary we will may make use of the following well known equality.

\begin{lemma}
	Let $X$ a Bernoulli random variable and $b \from \{0, 1\}$ (where $X$ and $b$ are not necessarily independent). Then for $x \in \{0, 1\}$
	\[
		2 \cdot \left(\pr{X = b} - \frac{1}{2}\right) = \pr{X = x \mid b = x} - \pr{X = x \mid b = 1 - x}.
	\]
	In particular, if $\adv$ is an adversary with output in $\{0, 1\}$ playing a game where a bit $b \from \{0, 1\}$ is sampled, then for $x \in \{0, 1\}$
	\begin{equation} \label{eq:advantage-equality}
		2 \cdot \left(\pr{b \from \adv} - \frac{1}{2}\right) = \pr{x \from \adv \mid b = x} - \pr{x \from \adv \mid b = 1 - x}.
	\end{equation}
\end{lemma}
\begin{proof}
	Let $x \in \{0, 1\}$. We have
	\begin{align*}
		2 \cdot \left(\pr{X = b} - \frac{1}{2}\right) & = 2 \cdot \left(\pr{X = x \mid b = x} \cdot \frac{1}{2} + \pr{X = 1 - x \mid b = 1 - x} \cdot \frac{1}{2} - \frac{1}{2}\right) \\
		                                              & = \pr{X = x \mid b = x} + \pr{X = 1 - x \mid b = 1 - x} - 1                                                                    \\
		                                              & = \pr{X = x \mid b = x} - (1 - \pr{X = 1 - x \mid b = 1 - x})                                                                  \\
		                                              & = \pr{X = x \mid b = x} - \pr{X = x \mid b = 1 - x}.
	\end{align*}
\end{proof}


We will make use of a weaker form of security called ``indistinguishability in the presence of an eavesdropper'' (\cite{introduction-to-modern-cryptography}) and will refer to it as ``EAV security''. It is identical to IND-CPA security with the sole exception that the adversary does not have access to an encryption oracle.

\begin{definition}[The EAV game]
	Let $\Pi$ a private-key encryption scheme. Define the game $\mathrm{Game}_{\adv, \Pi}^{\mathrm{EAV}}$ for an adversary $\adv$:
	\begin{enumerate}[1.]
		\item A key $k \from \gen()$ is generated.
		\item The adversary $\adv$ outputs a pair of messages $m_0, m_1$ of the same length.
		\item A bit $b \from \{0, 1\}$ is sampled and $\adv$ is given a ciphertext $c \from \enc_k(m_b)$.
		\item $\adv$ outputs a bit $b'$. The output of the game is defined to be $1$ if $b' = b$, and $0$ otherwise.
	\end{enumerate}
\end{definition}

\begin{definition}[EAV security {\cite[Definition 3.8]{introduction-to-modern-cryptography}}]
	A private-key encryption scheme $\Pi$ is \emph{$(t, \epsilon)$-EAV secure} if for any adversary $\adv$ running in time $t$ we have
	\begin{align*}
		\mathrm{Adv}_{\Pi}^{\mathrm{EAV}}(\adv) \coloneqq 2 \cdot \left(\pr{\mathrm{Game}_{\adv, \Pi}^{\mathrm{EAV}} = 1} - \frac{1}{2}\right) \le \epsilon.
	\end{align*}
\end{definition}


\begin{lemma}
	Let $\Pi$ a private-key encryption scheme. If $\Pi$ is $(t, \epsilon)$-IND-CPA secure, then $\Pi$ is $(t, \epsilon)$-EAV secure.
\end{lemma}
\begin{proof}
	This follows immediately from the fact that any EAV adversary is also an IND-CPA adversary.
\end{proof}

\begin{definition}[The Decisional Diffie-Hellman (DDH) problem]
	Let $\mathbb{G}$ a cyclic group and $g$ a generator. Define the game $\mathrm{Game}_{\adv, (\mathbb{G}, g)}^{\mathrm{DDH}}$ for an adversary $\adv$:
	\begin{enumerate}[1.]
		\item Exponents $x, y \from [\abs{\mathbb{G}}]$ and a bit $b \from \{0, 1\}$ are sampled.
		\item The adversary $\adv$ is given $h_1 \coloneqq g^x, h_2 \coloneqq g^y$ and
		      \[
			      k = \begin{cases}
				      g^{x \cdot y} & b = 0 \\
				      \tilde{k}     & b = 1
			      \end{cases}
		      \]
		      where $\tilde{k} \from \mathbb{G}$.
		\item $\adv$ outputs a bit $b'$. The output of the game is defined to be $1$ if $b' = b$, and $0$ otherwise.
	\end{enumerate}
\end{definition}

\begin{definition}[Hardness of the DDH problem {\cite[Definition 9.64]{introduction-to-modern-cryptography}}]
	The DDH problem is \emph{$(t, \epsilon)$-hard in} $\mathbb{G}$ with the generator $g$ if for any adversary $\adv$ running in time $t$ we have
	\begin{align*}
		\mathrm{Adv}_{(\mathbb{G}, g)}^{\mathrm{DDH}}(\adv) \coloneqq 2 \cdot \left(\pr{\mathrm{Game}_{\adv, (\mathbb{G}, g)}^{\mathrm{DDH}} = 1} - \frac{1}{2}\right) \le \epsilon.
	\end{align*}
\end{definition}

\todo{Define DHIES.}

\subsection{The Random Oracle Model}

We will work in the commonly used Random Oracle Model (ROM) to prove our results. We refer the reader to \cite[Chapter 6.5]{introduction-to-modern-cryptography} for an informal overview of the ROM and to \cite{rom} for the original work that introduced the model. The ROM introduces the concept of a \emph{random oracle}. If a function $H : A \to B$ is modelled as a random oracle, then certain assumptions are made about what an adversary $\adv$ knows about $H$ and how it interacts with it:
\begin{itemize}
	\item From $\adv$'s perspective, $H$ is a black-box function. The only way for $\adv$ to interact with $H$ is for it to provide a value $a \in A$ and get back $H(a)$, and this is the only way for $\adv$ to learn $H(a)$. We say that $\adv$ \emph{queries} $H(a)$ or that $\adv$ \emph{queries $H$ for $a$}. This well-defined interface of $\adv$ to $H$ implies that a reduction can extract the queries that $\adv$ makes to $H$.
	\item From $\adv$'s perspective, $H$ is a random variable, uniformly sampled from the set of all functions from $A$ to $B$. Thus, if $\adv$ queries $H$ for some $a \in A$ that it has not queried before, the value $H(a)$ is a random variable uniformly distributed in $B$ from $\adv$'s perspective.
\end{itemize}
We do not rely on the property known as ``programmability'' in this work.
