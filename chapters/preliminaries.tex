\chapter{Preliminaries}

\section{Notation}

We will use the following notation throughout:
\begin{itemize}
	\item We write ``u.a.r.'' for ``uniformly at random''
	\item We write $x \from S$ to say that $x$ is sampled u.a.r.\ from the finite set $S$
	\item For $n \in \mathbb{N} \setminus \{0\}$, $[n] = \{1, \ldots, n\}$
	\item If $\mathbb{G}$ is a cyclic group of order $q$ and $g$ a generator, then
	      \begin{itemize}
		      \item We write the group operation in $\mathbb{G}$ multiplicatively
		      \item $h^{-1}$ denotes the inverse of $h \in \mathbb{G}$
		      \item $\log_g(h)$ denotes the unique $x \in [q]$ such that $g^x = h$
	      \end{itemize}
	\item We write $b \from \adv$ to denote the event that an adversary $\adv$ outputs the bit $b$ when playing a game where it must output a bit in the end
	\item For $a, b \in \{0, 1\}^n$, $a \oplus b$ denotes the XOR of $a$ and $b$
	\item $\log$ is short for $\log_2$
	\item We will stick to using $\kappa$ as the security parameter of private-key encryption schemes and $\eta$ as the parameter of public-key encryption schemes
	\item $\{0, 1\}^{\le l} = \bigcup_{i = 1}^{l} \{0, 1\}^i $
\end{itemize}


\section{Basic definitions}

The definitions presented in this section were taken from \cite{introduction-to-modern-cryptography}.

\subsection{Encryption schemes}

\subsubsection{Private-key encryption}

\begin{definition}[Private-key encryption {\cite[Definition 3.7]{introduction-to-modern-cryptography}}]
	Let $\kappa$ denote the security parameter. A \emph{private-key encryption scheme} $\Pi$ consists of three probabilistic polynomial-time algorithms $(\gen, \enc, \dec)$ such that:
	\begin{enumerate}[1.]
		\item The \emph{key-generation algorithm} $\gen$ takes as input $1^\kappa$ (in unary) and outputs a key $k$. We will write $k \from \gen(1^\kappa)$.
		\item The \emph{encryption algorithm} $\enc$ takes as input a key $k$ and a message $m \in \{0, 1\}^*$ , or $m \in \{0, 1\}^{\le l(\kappa)}$ for some function $l$ if the message space is finite, and outputs a ciphertext $c$. We write this as $c \from \enc_k(m)$.
		\item The deterministic \emph{decryption algorithm} $\dec$ takes as input a key $k$ and a ciphertext $c$, and outputs a message $m$ or $\bot$ (denoting an error). We write this as $m = \dec_{sk}(c)$.
	\end{enumerate}

	We may also refer to algorithm $X$ by $\Pi.X$ for $X \in \{\gen, \enc, \dec\}$.

	It is required that for every $\kappa$, every key $k$ output by $\gen$, and every message $m$, it holds that $\pr{\dec_k(\enc_k(m)) = m} = 1$ (where the probability is over the randomness of $\enc_k$).
\end{definition}

\subsubsection{Public-key encryption}

In the following definition we will be more explicit about the randomness used by the algorithm $\gen$, as we will require a way to provide the randomness as input later.

\begin{definition}[Public-key encryption {\cite[Definition 12.1]{introduction-to-modern-cryptography}}] \label{def:public-key-encryption}
	Let $\eta$ denote the security parameter.
	A \emph{public-key encryption scheme} $\Pi$ consists of three probabilistic polynomial-time algorithms $(\gen, \enc, \dec)$ such that:
	\begin{enumerate}[1.]
		\item The \emph{key-generation algorithm} $\gen$ takes as input $1^\eta$ (in unary) and outputs a pair of keys $(pk, sk)$ (a public and private key). We will write $(pk, sk) \from \gen(1^\eta)$.

		      The public key defines a message space $\mathcal{M}_{pk}$.

		      The algorithm samples up to $r(\eta)$ uniformly random bits to make randomized decisions for some function $r$ polynomial in $\eta$. The sequence of random bits $s \in \{0, 1\}^n$ with $n \ge r(\eta)$ to be used by the algorithm may also be provided as input. We write this as $(pk, sk) = \gen(1^\eta, s)$ to emphasize the fact that the output is deterministic. The algorithm must still run in time polynomial in $1^\eta$. The distribution over key pairs output by sampling $s \from \{0, 1\}^n$ and running $\gen(1^\eta, s)$ is identical to the distribution over key pairs output by running $\gen(1^\eta)$.


		\item The \emph{encryption algorithm} $\enc$ takes as input a public key $pk$ and a message $m \in \mathcal{M}_{pk}$, and outputs a ciphertext $c$. We write this as $c \from \enc_{pk}(m)$.
		\item The deterministic \emph{decryption algorithm} $\dec$ takes as input a private key $sk$ and a ciphertext $c$, and outputs a message $m$ or $\bot$ (denoting an error). We write this as $m = \dec_{sk}(c)$.
	\end{enumerate}

	We may also refer to algorithm $X$ by $\Pi.X$ for $X \in \{\gen, \enc, \dec\}$.

	It is required that for every $\eta$, every key $(pk, sk)$ output by $\gen$, and every message $m$, it holds that $\pr{\dec_{sk}(\enc_{pk}(m)) = m} = 1$ (where the probability is over the randomness of $\enc_pk$).
\end{definition}

\subsection{Security definitions}

\begin{definition}[The IND-CPA game]
	Let $\kappa$ denote the security parameter and let $\Pi$ a private-key encryption scheme. Define the game $\mathrm{Game}_{\adv, \Pi}^{\mathrm{IND-CPA}}(\kappa)$ for an adversary $\adv$:
	\begin{enumerate}[1.]
		\item A key $k \from \gen(1^\kappa)$ is generated.
		\item The adversary $\adv$ is given $1^\kappa$ and oracle access to $\Pi.\enc_k$, and outputs a pair of messages $m_0, m_1$ of the same length.
		\item A bit $b \from \{0, 1\}$ is sampled and $\adv$ is given a ciphertext $c \from \enc_k(m_b)$. ($\adv$ continues to have oracle access to $\Pi.\enc_k$.)
		\item $\adv$ outputs a bit $b'$. The output of the game is defined to be $1$ if $b' = b$, and $0$ otherwise.
	\end{enumerate}
\end{definition}

\begin{definition}[IND-CPA security {\cite[Definition 3.21]{introduction-to-modern-cryptography}}]
	A private-key encryption scheme $\Pi$ is \emph{$(t, \epsilon, q)$-IND-CPA secure} if for any adversary $\adv$ running in time $t$ we have
	\begin{align*}
		\mathrm{Adv}_{\Pi}^{\mathrm{IND-CPA}}(\adv) \coloneqq 2 \cdot \left(\pr{\mathrm{Game}_{\adv, \Pi}^{\mathrm{IND-CPA}} = 1} - \frac{1}{2}\right) \le \epsilon.
	\end{align*}
\end{definition}

We will make use of a weaker form of security called ``indistinguishability in the presence of an eavesdropper'' (\cite{introduction-to-modern-cryptography}) and will refer to it as ``EAV security''. It is identical to IND-CPA security with the sole exception that the adversary does not have access to an encryption oracle.

\begin{definition}[The EAV game]
	Let $\kappa$ denote the security parameter and $\Pi$ a private-key encryption scheme. Define the game $\mathrm{Game}_{\adv, \Pi}^{\mathrm{EAV}}$ for an adversary $\adv$:
	\begin{enumerate}[1.]
		\item A key $k \from \gen(1^\kappa)$ is generated.
		\item The adversary $\adv$ is given $1^\kappa$, and outputs a pair of messages $m_0, m_1$ of the same length.
		\item A bit $b \from \{0, 1\}$ is sampled and $\adv$ is given a ciphertext $c \from \enc_k(m_b)$.
		\item $\adv$ outputs a bit $b'$. The output of the game is defined to be $1$ if $b' = b$, and $0$ otherwise.
	\end{enumerate}
\end{definition}

\begin{definition}[EAV security {\cite[Definition 3.8]{introduction-to-modern-cryptography}}]
	A private-key encryption scheme $\Pi$ is \emph{$(t, \epsilon)$-EAV secure} if for any adversary $\adv$ running in time $t$ we have
	\begin{align*}
		\mathrm{Adv}_{\Pi}^{\mathrm{EAV}}(\adv) \coloneqq 2 \cdot \left(\pr{\mathrm{Game}_{\adv, \Pi}^{\mathrm{EAV}} = 1} - \frac{1}{2}\right) \le \epsilon.
	\end{align*}
\end{definition}


\begin{lemma}
	Let $\Pi$ a private-key encryption scheme. If $\Pi$ is $(t, \epsilon)$-IND-CPA-secure, then $\Pi$ is $(t, \epsilon)$-EAV-secure.
\end{lemma}
\begin{proof}
	This follows immediately from the fact that any EAV adversary is also an IND-CPA adversary.
\end{proof}

\begin{definition}[Group-generation algorithm {\cite[Section 9.3.2]{introduction-to-modern-cryptography}}]
	Let $\eta$ denote the security parameter. A \emph{group-generation algorithm} $\mathcal{G}$ is a deterministic polynomial-time algorithm that takes as input $1^\eta$ and outputs $(\mathbb{G}, q, g)$, where $\mathbb{G}$ is (a description of) a cyclic group, $q$ is the order of the group with $q \ge 2^\eta$ \todo{require this? probably useful for applying bound later} and $g \in \mathbb{G}$ is a generator. A group element is represented as a bit-string of length at most $\gamma(\eta)$. We write $(\mathbb{G}, q, g) \from \mathcal{G}(1^\eta)$.
\end{definition}

\begin{definition}[The Decisional Diffie-Hellman (DDH) problem]
	Let $\eta$ denote the security parameter and let $\mathcal{G}$ a group-generation algorithm.
	Define the game $\mathrm{Game}_{\adv, \mathcal{G}}^{\mathrm{DDH}}$ for an adversary $\adv$:
	\begin{enumerate}[1.]
		\item $\mathcal{G}(1^\eta)$ is run to obtain $(\mathbb{G}, q, g)$, and exponents $x, y \from [q]$ and a bit $b \from \{0, 1\}$ are sampled.
		\item The adversary $\adv$ is given $\mathbb{G}$, $q$, $g$, $h_1 \coloneqq g^x, h_2 \coloneqq g^y$ and
		      \[
			      k = \begin{cases}
				      g^{x \cdot y} & b = 0 \\
				      \tilde{k}     & b = 1
			      \end{cases}
		      \]
		      where $\tilde{k} \from \mathbb{G}$.
		\item $\adv$ outputs a bit $b'$. The output of the game is defined to be $1$ if $b' = b$, and $0$ otherwise.
	\end{enumerate}
\end{definition}

\begin{definition}[Hardness of the DDH problem {\cite[Definition 9.64]{introduction-to-modern-cryptography}}]
	The DDH problem is \emph{$(t, \epsilon)$-hard relative to} $\mathcal{G}$ if for any adversary $\adv$ running in time $t$ we have
	\begin{align*}
		\mathrm{Adv}_{\mathcal{G}}^{\mathrm{DDH}}(\adv) \coloneqq 2 \cdot \left(\pr{\mathrm{Game}_{\adv, \mathcal{G}}^{\mathrm{DDH}} = 1} - \frac{1}{2}\right) \le \epsilon.
	\end{align*}
\end{definition}

Note that by saying that ``the DDH problem is hard relative to $\mathcal{G}$'', we are simply saying that solving the Diffie-Hellman problem is hard in the (fixed) group $\mathbb{G}$ output by $\mathcal{G}(1^\eta)$. We use this phrasing as opposed to saying something along the lines of ``the DDH problem is hard in $\mathbb{G}$'', which better expresses what we mean, in order to stay closer to the formulation used in the asymptotic setting.

When analyzing the advantage of an adversary we may make use of the following well known equality.

\begin{lemma}
	Let $X$ a Bernoulli random variable and $b \from \{0, 1\}$ (where $X$ and $b$ are not necessarily independent). Then for $x \in \{0, 1\}$
	\[
		2 \cdot \left(\pr{X = b} - \frac{1}{2}\right) = \pr{X = x \mid b = x} - \pr{X = x \mid b = 1 - x}.
	\]
	In particular, if $\adv$ is an adversary with output in $\{0, 1\}$ playing a game where a bit $b \from \{0, 1\}$ is sampled, then for $x \in \{0, 1\}$
	\begin{equation} \label{eq:advantage-equality}
		2 \cdot \left(\pr{b \from \adv} - \frac{1}{2}\right) = \pr{x \from \adv \mid b = x} - \pr{x \from \adv \mid b = 1 - x}.
	\end{equation}
\end{lemma}
\begin{proof}
	Let $x \in \{0, 1\}$. We have
	\begin{align*}
		2 \cdot \left(\pr{X = b} - \frac{1}{2}\right) & = 2 \cdot \left(\pr{X = x \mid b = x} \cdot \frac{1}{2} + \pr{X = 1 - x \mid b = 1 - x} \cdot \frac{1}{2} - \frac{1}{2}\right) \\
		                                              & = \pr{X = x \mid b = x} + \pr{X = 1 - x \mid b = 1 - x} - 1                                                                    \\
		                                              & = \pr{X = x \mid b = x} - (1 - \pr{X = 1 - x \mid b = 1 - x})                                                                  \\
		                                              & = \pr{X = x \mid b = x} - \pr{X = x \mid b = 1 - x}.
	\end{align*}
\end{proof}

In the following definition we will refer to ``key-derivation functions''. This is only meant as a hint to the reader. We do not provide a definition here, as we will always model such a function as a random oracle (see Section~\vref{sec:rom}). \question{Good referenc to cite on KDFs?}

\begin{definition}[DHIES {\cite[Construction 12.19]{introduction-to-modern-cryptography}}]
	Let $\eta$ denote the security parameter. Let $\mathcal{G}$ a group-generation algorithm. Let $\Pi_s$ a private-key encryption scheme where the security parameter $\kappa$ is some function of $\eta$ and $\Pi_s.\gen(1^{\kappa(\eta)})$ samples a key u.a.r.\ from $\{0, 1\}^{\kappa(\eta)}$. Let $\mathcal{H} = \{ H_{\eta} \mid \eta \in \N \}$ a family of key-derivation functions with $H_{\eta} \colon \mathbb{G} \to \{0, 1\}^{\kappa(\eta)}$ where $\mathbb{G}$ is the group output by $\mathcal{G}(1^\eta)$. Define the algorithms $\gen, \enc$ and $\dec$ as follows:
	\begin{itemize}
		\item $\gen$: on input $1^\eta$ run $\mathcal{G}(1^\eta)$ to obtain $(\mathbb{G}, q, g)$. Sample $x \from [q]$ and set $h_1 \coloneqq g^x$. Let $H \coloneqq H_{\eta}$. Set $pk \coloneqq \langle \mathbb{G}, q, g, h_1, H \rangle$ and $sk \coloneqq \langle \mathbb{G}, q, g, x, H \rangle$, and output $(pk, sk)$.

		      The message space is the message space of $\Pi_s$ (with the security parameter set to $\kappa(\eta)$).
		\item $\enc$: on input a public key $\langle \mathbb{G}, q, g, h_1, H \rangle$ and a message $m$, sample $y \from [q]$, set $h_2 \coloneqq g^y, k \coloneqq H(h_1^y)$, compute $c' \from \Pi_s.\enc_k(m)$ and output the ciphertext $\langle h_2, c' \rangle$.
		\item $\dec$: on input a private key $\langle \mathbb{G}, q, g, x, H \rangle$ and a ciphertext $\langle h_2, c' \rangle$, compute $k \coloneqq H(h_2^x)$ and output $\Pi_s.\dec_k(c')$. If the ciphertext is not of the right form or $\Pi_s.\dec$ outputs $\bot$, output $\bot$.
	\end{itemize}
	The public-key encryption scheme $\dhies \coloneqq (\gen, \enc, \dec)$ is called the Diffie-Hellman Integrated Encryption Scheme (DHIES).

	When using the DHIES scheme later on, we will set $pk \coloneqq \langle h \rangle$ and $sk \coloneqq x$ in $\gen$, as the security parameter $\eta$ and thus $\mathbb{G}, q, g$ and $H$ are known in practice.
\end{definition}

The DHIES scheme is an instance of a so-called \emph{key-encapsulation mechanism} (\cite[Definition~12.9]{introduction-to-modern-cryptography}): a scheme that uses a public key to encapsulate a symmetric encryption key in a ciphertext and the corresponding private key to compute the encryption key again from the ciphertext. This can be combined with any arbitrary secure private-key encryption scheme to get a secure and efficient public-key encryption scheme by sending a message encrypted with the private-key encryption scheme along with an encapsulation of the encryption key.
Under the DDH assumption (i.e. the assumption that the DDH problem is hard relative to $\mathcal{G}$), using DHIES with an EAV secure private-key scheme gives an IND-CPA secure public-key encryption scheme in the ROM (see Section~\vref{sec:rom}), as proven in \cite[Theorem~12.12]{introduction-to-modern-cryptography}. Moreover, under the so-called ``gap-CDH'' assumption, also called the ``Strong Diffie-Hellman'' assumption in \cite{dhies}, using DHIES with an IND-CCA2 secure private-key encryption scheme gives an IND-CCA2 secure public-key encryption scheme (\cite[Theorem~12.22]{introduction-to-modern-cryptography}). (We do not provide defintions for many of the notions mentioned here as we will not make use of them in this work.)


\subsection{The Random Oracle Model} \label{sec:rom}

We will work in the commonly used Random Oracle Model (ROM) to prove our results. We refer the reader to \cite[Chapter 6.5]{introduction-to-modern-cryptography} for an informal overview of the ROM and to \cite{rom} for the original work that introduced the model. The ROM introduces the concept of a \emph{random oracle}. If a function $H : A \to B$ is modelled as a random oracle, then certain assumptions are made about what an adversary $\adv$ knows about $H$ and how it interacts with it:
\begin{itemize}
	\item From $\adv$'s perspective, $H$ is a black-box function. The only way for $\adv$ to interact with $H$ is for it to provide a value $a \in A$ and get back $H(a)$, and this is the only way for $\adv$ to learn $H(a)$. We say that $\adv$ \emph{queries} $H(a)$ or that $\adv$ \emph{queries $H$ for $a$}. This well-defined interface of $\adv$ to $H$ implies that a reduction can extract the queries that $\adv$ makes to $H$.
	\item From $\adv$'s perspective, $H$ is a random variable, sampled u.a.r.\ from the set of all functions from $A$ to $B$. Thus, if $\adv$ queries $H$ for some $a \in A$ that it has not queried before, the value $H(a)$ is a random variable uniformly distributed in $B$ from $\adv$'s perspective.
\end{itemize}
We do not rely on the property known as ``programmability'' in this work.
