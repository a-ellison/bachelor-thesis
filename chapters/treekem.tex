\chapter{Application to TreeKEM} \label{sec:application-to-treekem}

\section{Continuous Group Key Agreement}

\subsection{Syntax}

\todo{Explain model:
	\begin{itemize}
		\item users are honest nodes running the protocol algorithms and maintaining local state
		\item delivery server has a reliable and authenticated communication channel (message passing) to each individual user.
	\end{itemize}
}

\question{What assumptions are needed about the delivery server?
	\begin{itemize}
		\item cannot forge messages
		\item can choose not to deliver some messages
	\end{itemize}
}

\question{What correctness conditions are necessary in the definition?}

\todo{Call it more something like PC-CGKA? (PC: propose and commit)}

We assume that every user $u$ is identified by some value $id_u$.

\todo{Write security parameter?}

\begin{definition}[CGKA]
	A \emph{CGKA scheme} $\Sigma$ with key space $\mathcal{K}$ consists of the following algorithms:
	\begin{enumerate}[1.]
		\item[] \textsc{Initialization:}
			\begin{itemize}
				\item An algorithm $\gen$. Before joining any group, a user generates a pair of keys $(pk, sk) \from \gen(1^\eta)$, a public and private key. The public key is used to invite the user to the group and should therefore be made public. The value $sk$ is kept secret. The same key pair must not be reused to join multiple groups. A new key pair must be generated every time.
				\item An algorithm $\operatorname{CreateGroup}$. A user runs $\sigma \from \operatorname{CreateGroup}()$ to locally initialize a group with themselves as the only member and the state of the group stored in $\sigma$. We call $\sigma$ their \emph{group state}.
			\end{itemize}
		\item[] \textsc{Compute the group key:}
			\begin{itemize}
				\item An algorithm $\operatorname{Key}$. At any point in time, a member of a group with state $\sigma$ can compute the current \emph{group key} $k \from \operatorname{Key}(\sigma)$ with $k \in \mathcal{K}$. Any set of members with consistent group states (see Definition~\vref{def:consistent-group-state}) must compute the same key $k$.
			\end{itemize}
		\item[] \textsc{Proposal:}
			\begin{itemize}
				\item An algorithm $\operatorname{ProposeUpdate}$. If a member $u$ of a group with state $\sigma$ wishes update their key material, they may run $(\sigma, p) \from \operatorname{ProposeUpdate}(\sigma)$ to create an \emph{update proposal} $p$ to be shared with other members of the group and update their state such that they have processed $p$. The update proposal contains (possibly public) information for the other group members about $u$'s new key material such that other members know how to provide encrypted information in a commit (see below) such that $u$ is able to compute the group key.
				\item An algorithm $\operatorname{ProposeAdd}$. If a member of a group with state $\sigma$ wishes to add a new user $u$ with public key $pk_u$ to the group, they may run $(\sigma, p) \from \operatorname{ProposeAdd}(\sigma, id_u, pk_u)$ to create an \emph{add proposal} $p$ to be shared with other members of the group and update their state such that they have processed $p$.
				\item An algorithm $\operatorname{ProposeRemove}$. If a member of a group with state $\sigma$ wishes to remove another member $u$ from the group, they may run $(\sigma, p) \from \operatorname{ProposeAdd}(\sigma, id_u)$ to create a \emph{remove proposal} $p$ to be shared with other members of the group and update their state such that they have processed $p$.
			\end{itemize}
		\item[] \textsc{Commit:}
			\begin{itemize}
				\item An algorithm $\operatorname{CreateCommit}$. To apply a set of proposals $\pi$ to the group state, a member with state $\sigma$ may run $(\sigma', c, w_{_1}, \ldots, w_{_k}) \from \operatorname{CreateCommit}(\sigma, \pi)$, where $c$ is a \emph{commit} to be shared with other members, $\sigma'$ would be the new state of the member after applying the commit and each $w_{i}$ is a \emph{welcome message}, one for each new user added to the group in the commit with a corresponding add proposal in $\pi$. Welcome message $w_i$ contains the identifier $id_i$ of a user and the message should be shared with that user such that they can join the group. Besides updating the key material for all other members with an update proposal in $\pi$, the commit also updates the member's key material. Accordingly, $\pi$ should not contain an update proposal for the member creating the commit. The member may keep both $\sigma$ and $\sigma'$ until the group agrees on whether to apply the commit $c$ or not. If the commit is to be applied, the member sets their state to $\sigma'$ and discards $\sigma$. In this case, we say that the user \emph{applied} commit $c$. Otherwise, they discard $\sigma'$. Applying a commit results in a new group key. We will see the $\operatorname{CreateGroup}$ operation as applying a special type of commit.
			\end{itemize}
		\item[] \textsc{Process:}
			\begin{itemize}
				\item An algorithm $\operatorname{ProcessCommit}$. Upon receiving another member's commit $c$, a member with state $\sigma$ can set $\sigma \from \operatorname{ProcessCommit}(\sigma, c)$ to process $c$. We say that the member has \emph{applied} commit $c$. If the commit removed this member from the group, they should not be able to compute the group key from $\sigma$ and should delete $\sigma$.
				      As above, a user should only process a commit created in a consitent group state.
				\item An algorithm $\operatorname{ProcessWelcome}$. Upon receving a welcome message $w$ for a user with public key $pk$, the user with this public key can set $\sigma \from \operatorname{ProcessWelcome}(pk, sk, w)$, where $sk$ is the corresponding secret key output by $\gen$. The user must then discard $sk$.
			\end{itemize}
	\end{enumerate}
	For any object $X$ above (including $\mathcal{K}$) we will refer to it as $\Sigma.X$.

	The scheme must also specify
	\begin{itemize}
		\item the domain of the identifiers $id_u$
		\item an algorithm for determining the set of members of the group from a group state $\sigma$
	\end{itemize}

	Furthermore, we require a CGKA scheme to satisfy the correctness property described in \todo{Define correctness property.}
\end{definition}

\subsection{Correctness}

\todo{Separate syntax from intuition in definition above.}

\todo{Mention that public key corresponds to a key package in the RFC.}

\todo{Define what we mean by \emph{applying} a commit. Namely one of the following
	\begin{itemize}
		\item processing a commit
		\item creating a commit and subsequently moving to the new state of the commit
		\item the commit that was output along with their welcome message by a call to $\operatorname{CreateCommit}$ if they just joined the group
		\item the call to $\operatorname{CreateGroup}$ if they just created the group
	\end{itemize}
}

In the following, when talking about \emph{time} for a user that was a part of some group, we are referring to the sequence of group states they went through as members of the group.

\begin{definition}[Last commit]
	Let $u$ a user that at some point in time was a member of a group and had group state $\sigma$. We define their \emph{last commit} to be the most recent commit $c$ (up to arriving in state $\sigma$) that they applied .
\end{definition}

In the above definition, the user's last commit will always exist since they either joined the group through a welcome message or created the group themselves.

\begin{definition}[Consistent group states] \label{def:consistent-group-state}
	Let $u_0, u_1$ two users where each user was a member of a group at some point in time. Let $\sigma_0, \sigma_1$ the group states they were in and $c_0, c_1$ their last commits, respectively. The group states $\sigma_0$ and $\sigma_1$ are said to be \emph{consistent} if both users have the same last commit. (If a last commit is a call to $\operatorname{CreateGroup}$, it is the same as another last commit iff.\ both refer to the same call to $\operatorname{CreateGroup}$.)
\end{definition}

Users in different groups of course do not have consistent group states.


\subsection{Security}

\todo{Allow number of users to be determined adaptively?}

\begin{definition}[The CGKA game]
	Let $\Sigma$ a CGKA scheme. Define the game $\mathrm{Game}_{\adv, \Sigma}^{\mathrm{CGKA}}$ for an adversary $\adv$:
	\begin{enumerate}[1.]
		\item \label{def:cgka-game-step-1} The adversary $\adv$ outputs $n \in \mathbb{N}$. For each $i \in [n]$, initialize a user $i$ by creating an identifier $id_i$, generating $(pk_i, sk_i) \from \Sigma.\gen()$, preparing $U_i = \varnothing$, the set of unconfirmed commits at user $i$, and setting $\sigma_i = \varnothing$, where $\varnothing$ denotes the empty value. The state output by an algorithm of $\Sigma$ is never the empty value. $\adv$ is given $(pk_1, id_1), \ldots, (pk_n, id_n)$.

		      Set $P = C = W = 0$, where $P$ denotes the number of proposals, $C$ the number of commits and $W$ the number of welcome messages created by $\adv$.
		\item $\adv$ may adaptively do the following queries:
		      \begin{itemize}
			      \item $\operatorname{create-group}(i)$ for $i \in [n]$: set $\sigma_i \from \operatorname{CreateGroup}()$.
			      \item $\operatorname{propose-update}(i)$ for $i \in [n], \sigma_i \neq \varnothing$: run $(\sigma_i, p_{P + 1}) \from \operatorname{ProposeUpdate}(\sigma_i)$ to update user $i$'s state and get a proposal $p_{P + 1}$. $\adv$ is given $p_{P + 1}$. Set $P = P + 1$.
			      \item $\operatorname{propose-add}(i, j)$ for $i, j \in [n], \sigma_i \neq \varnothing, \sigma_j = \varnothing$: run $(\sigma_i, p_{P + 1}) \from \operatorname{ProposeAdd}(\sigma_i, id_j, pk_j)$ to update user $i$'s state and get a proposal $p_{P + 1}$. $\adv$ is given $p_{P + 1}$. Set $P = P + 1$.
			      \item $\operatorname{propose-remove}(i, j)$ for $i, j \in [n], \sigma_i \neq \varnothing, \sigma_j \neq \varnothing$: run $(\sigma_i, p_{P + 1}) \from \operatorname{ProposeRemove}(\sigma_i, id_j)$ to update user $i$'s state and get a proposal $p_{P + 1}$. $\adv$ is given $p_{P + 1}$. Set $P = P + 1$.
			      \item $\operatorname{create-commit}(i, I)$ for $i \in [n], \sigma_i \neq \varnothing, I \subseteq [P]$: run $(\sigma, c_{C + 1}, w_{W + 1}, \ldots, w_{W + k}) \from \operatorname{CreateCommit}(\sigma_i, \{p_j \mid j \in I\})$ to create the new state $\sigma$, commit $c_{C + 1}$ and corresponding welcome messages. $\adv$ is given $c_{C + 1}$ and $w_{W + 1}, \ldots, w_{W + k}$. Set $U_i = U_i \cup \{(C + 1, \sigma)\}$, $C = C + 1$ and $W = W + k$.
			      \item $\operatorname{confirm}(j, b)$ for $j$ s.t.~$(j, \sigma) \in U_i$ for some $i, \sigma$, $b \in \{0, 1\}$: Set $U_i = U_i \setminus \{(j, \sigma)\}$. If $b = 1$, set $\sigma_i = \sigma$.
			      \item $\operatorname{deliver-commit}(i, j)$ for $i \in [n], \sigma_i \neq \varnothing, j \in [C]$: run $\sigma \from \operatorname{ProcessCommit}(\sigma_i, c_j)$. If $c_j$ contains a remove proposal for user $i$, then set $\sigma_i = \varnothing$, generate a new pair $(pk_i, sk_i) \from \Sigma.\gen()$ and give $(i, pk_i)$ to $\adv$. Otherwise, set $\sigma_i = \sigma$.
			      \item $\operatorname{deliver-welcome}(i, j)$ for $i \in [n], \sigma_i = \varnothing, j \in [W]$: set $\sigma_i \from \operatorname{ProcessWelcome}(pk_j, sk_j, w_j)$ and set $sk_i = \varnothing$.
			      \item $\operatorname{corrupt}(i)$ for $i \in [n]$: $\adv$ is given $U_i$. If $\sigma_i = \varnothing$, $\adv$ is given $sk_i$. Otherwise, $\adv$ is given $\sigma_i$.
		      \end{itemize}
		\item $\adv$ picks $i \in \{0\} \cup [C]$. We call the $i$-th commit the \emph{challenge commit}, where the $0$-th commit refers to the initial $\mathrm{CreateGroup}$ operation. Let $\sigma$ the state output by the operation that created the $i$-th commit (the state output by $\operatorname{CreateCommit}$ if $i > 0$ or the state output by $\operatorname{CreateGroup}$ if $i = 0$). A bit $b \from \{0, 1\}$ is sampled and $\adv$ is given
		      \[
			      k = \begin{cases}
				      \Sigma.\mathrm{Key}(\sigma) & b = 0 \\
				      \tilde{k}                   & b = 1
			      \end{cases},
		      \]
		      where $\tilde{k} \from \Sigma.\mathcal{K}$. $\adv$ may continue to do queries as before.
		\item $\adv$ outputs a bit $b'$. The output of the game is defined to be $1$ if $b' = b$, and $0$ otherwise.
	\end{enumerate}

	We require an adversary playing the above game to adhere to the following:
	\begin{itemize}
		\item The challenge commit is safe (see Definition~\vref{def:safe-commit}) \question{Avoid collision in meaning of \emph{safe}?}
		\item For any query $\operatorname{deliver-commit}(i, j)$ where the commit $c_j$ was created by user $k$ while it was in state $\sigma_k'$, $\sigma_i$ and $\sigma_k'$ must be consistent
		\item $\operatorname{create-group}$ is queried exactly once \todo{More than once ok?}
		\item Every commit is processed at most once by any user
		\item A welcome message for user $i$ is processed by $i$ at most once and is never processed by a user $j$ with $i \neq j$
		\item A user never processes a commit that they created
		      \todo{Add the following other restrictions:
			      \begin{itemize}
				      \item committer never includes an update or remove proposal for itself
				      \item a proposal included in a commit must not already have been applied in the commit history of the user
				      \item user added that is already in the group
				      \item multiple update/add/remove proposals applying to the same user
				      \item user can only include a proposal it has already processed
				      \item update proposal included must be for a user in the group
			      \end{itemize}
		      }
	\end{itemize}

	\todo{Justify the restrictions to the adversary.}
\end{definition}

The concept of a safe user and safe commit is adapted from the so-called ``safe predicate'' in \cite{ttkem}, which again took inspiration from \cite{rtreekem}. As elaborated in the cited papers and also analogous to how we needed to define ``safe'' nodes in the SD-GSD game, we want to forbid the adversary to ask to be challenged on a commit for which it can trivially compute the group key through some corruption it performed.

To see what is needed for a commit to be safe, consider some commit $c$ with group key $k$ created by a user $i$ and let $j \neq i$ any user that $i$ would consider to be in the group after applying $c$. The commit $c$ (or associated welcome messages) provides encrypted information for user $j$ to compute the new group key using its current key material. Clearly, if this key material has been compromised by the adversary corrupting user $j$, the commit should not be safe. If the adversary has not corrupted user $j$ since the user last updated their key material, then we would not expect the adversary to be able to compute the group key introduced in $c$, even if user $j$ was corrupted before (recall post-compromise security). Moreover, corrupting user $j$ after they have again updated their key material should not allow the adversary to compute the group key of $c$ either (recall forward secrecy). We will later call this window between user $j$'s last and next update a \emph{critical window} during which the adversary must not have corrupted user $j$ for $c$ to be safe. Now, it is important to notice that the encrypted information in commit $c$ is for the key material that user $j$ had \emph{from user $i$'s view} when user $i$ created $c$, so we must be careful to keep exactly this key material of user $j$ inaccessible to the adversary through a corruption. The following definition formalizes this.

\begin{definition}[Safe user] \label{def:safe-user}
	\todo{Define safe user.}
\end{definition}

Continuing the discussion above, so far we have considered a necessary condition to keep the commit $c$ safe by restricting the corruptions made to a specific user $j$. If no user that $i$ considered to be in the group (including user $i$) was compromised in their critical window, we would expect that the adversary is not able to compute the group key of $c$. Indeed, this is how we define a safe commit.

\begin{definition}[Safe commit] \label{def:safe-commit}
	\todo{Define safe commit.}
\end{definition}


\begin{definition}[CGKA security]
	A CGKA scheme is \emph{$(t, \epsilon, q, u)$-CGKA secure} if for any adversary $\adv$ making at most $q$ queries to $\mathrm{create-commit}$ and asking for at most $u$ users in step \ref{def:cgka-game-step-1} we have
	\begin{align*}
		\mathrm{Adv}_{\Sigma}^{\mathrm{CGKA}}(\adv) \coloneqq 2 \cdot \abs*{\pr{\mathrm{Game}_{\adv, \Sigma}^{\mathrm{CGKA}} = 1} - \frac{1}{2}} \le \epsilon.
	\end{align*}
\end{definition}

\section{The TreeKEM Protocol}

\todo{Define TreeKEM as a CGKA.}

\question{How much detail to provide on TreeKEM details?}

\section{TreeKEM security from SD-GSD security}

\todo{Reduce TreeKEM security to GSD security.}

\begin{theorem}
	\todo{State and prove CGKA security of TreeKEM by applying Lemma~\ref{theorem:sdgsd-security}.}
\end{theorem}
